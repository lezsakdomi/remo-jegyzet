\topic{Struktúra alapú modellezés}

% vezérpélda lehetőségek
% - táblázatszerű adatok pl. felhasználókról (preferenciák, van-e jogsija, stb.), kocsikról (férőhely, légkondi stb.)
% - kiszolgáló rendszer top-down, bottom-up tervezése (mobil kliens, tömegközl. integráció, adatbázis stb.)
% - gráfként reprezentált fuvarok, kocsik, szakaszok, utasok, sofőrök stb.

\todo[inline]{kifejteni azt, hogy lehet logikai és fizikai felépítés alapján is modellezni.
	
	- Fizikai/architektúrális egységek --> moduláris struktúra hoz létre, újrahasznosítható részekkel.
	
	- Funkcionális egységek --> dedikált struktúrát hoz létre, pl. távirányító -- redundanciamentes, azért, hogy a költség alacsony legyen. Nem újrahasznosíthatók a részei.
	
	Pl. dedikált eszközt készítünk, amikor van egy rendszermenedzsment problémám és írok rá egy szkriptet. A másik megoldás lenne, hogy behozunk egy rendszermenedzsment eszközt és annak az újrahasznosítható komponenseit alkalmazzuk.
	
	Jó lenne építészeti példát behozni.}

Hogyan épül fel egy rendszer? Milyen részekre bontható és ezek között milyen kapcsolat van? Ahhoz, hogy a rendszer egészével kapcsolatos összetett problémákat megválaszoljunk, fontos, hogy tudjuk ezekre a kérdésekre a választ. Ebben a fejezetben az egyes rendszerek \fogalomragozva{struktura}{struktúrájának} jellemzésével foglalkozunk. Bemutatjuk a strukturális modellezés motivációját, a leggyakrabban alkalmazott formalizmusokat és azok matematikai alapjait.

%Modellezés során egy rendszert jellemezhetünk annak statikus, ill. dinamikus jellemzői alapján. A statikus jellemzés esetén a rendszer felépítését, \fogalomragozva{struktura}{struktúráját} írjuk le, míg dinamikus jellemzés esetén a rendszer \fogalomragozva{viselkedes}{viselkedésével} foglalkozunk.

\section{Strukturális modell}

\begin{definicio}
A \fogalom{strukturalis-modell} a rendszer felépítésére vonatkozó tudás.
\end{definicio}

A strukturális modell alkalmas arra, hogy az alábbi kérdéseikre nyújtson valamilyen (nem feltétlenül kimerítő) választ:

\begin{itemize}
\item Milyen elemekből áll a rendszer?
\item Hogyan kapcsolódnak egymáshoz az elemek?
\item Milyen tulajdonságúak az elemek?
\end{itemize}

Egy rendszer strukturális modellje a rendszer dekompozíciójával állítható elő.

\begin{definicio}
	A \fogalom{dekompozicio} (\fogalom{faktoring}) egy összetett probléma vagy rendszer kisebb részekre bontása, amelyek könnyebben érthetők, fejleszthetők és karbantarthatók.
\end{definicio}


A rendszer így kapott egyes komponensei (részrendszerei) gyakran további dekompozícióval még kisebb részekre bonthatóak. Természetesen a dekompozíció során ügyelnünk kell arra, hogy az egyes részekből visszaállítható legyen az eredeti rendszer, különben a kapott strukturális modellünk hiányos.

\begin{definicio}
	Egy dekompozíció \fogalomragozva{helyes-dekompozicio}{helyes}, ha a dekompozícióval kapott rendszer minden elemének megfeleltethető az eredeti rendszer valamelyik eleme, és az eredeti rendszer minden eleméhez hozzárendelhető a dekompozícióval kapott rendszer egy vagy több eleme.
\end{definicio}

\begin{definicio}
	\Fogalomangolul{top-down} modellezés során a rendszert felülről lefelé építjük. A modellezés alaplépése a \fogalom{dekompozicio}.
\end{definicio}

A top-down modellezés jellemzői:

\begin{itemize}
\item[$\oplus$] Részrendszer tervezésekor a szerepe már ismert
\item[$\ominus$] ,,Félidőben'' még nincsenek működő részek
\item[$\ominus$] Részek problémái, igényei későn derülnek ki
\end{itemize}

\begin{definicio}
	\Fogalomangolul{bottom-up} modellezés során a rendszert alulról felfelé építjük. A modellezés alaplépése a \fogalom{kompozicio}: az egész rendszer összeszerkesztése külön modellezett vagy fejlesztett részrendszerekből.
\end{definicio}

A bottom-up modellezés jellemzői:

\begin{itemize}
\item[$\oplus$] Részrendszer önmagában kipróbálható, tesztelhető
\item[$\oplus$] Részleges készültségnél is összeépíthető valami
\item[$\ominus$] Nem látszik előre a rész szerepe az egészben
\end{itemize}

\section{Gráfok}
\label{sec:graf}

Az alábbiakban összefoglaljuk a gráfokkal kapcsolatos fogalmakat és definíciókat. Nem mindenhol célunk a teljes matematikai precizitás, ill. az egyes fogalmakhoz tartozó különböző definíciók felsorolása (pl. a \fogalom{graf} fogalomra több különböző, némileg eltérő jelentésű definíció létezik).


\begin{definicio}
A \fogalom{graf} egy olyan $G = (V, E)$ struktúra, ahol $V$ halmaz a csomópontok, $E$ az élek halmaza. Az élek csomópontok között futnak, \fogalom{iranyitatlan-graf} esetén $E$ csomópontok rendezetlen $\{v_1, v_2\}$ párjaiból áll (tehát nem különböztetjük meg a $\{v_1, v_2\}$ és a $\{v_2, v_1\}$) párokat, míg \fogalom{iranyitott-graf} esetén csomópontok rendezett $(v_1, v_2)$ párjaiból.
\end{definicio}

Bár a gráf meglehetősen egyszerű fogalom, rendkívül sok szakterület modelljei jól ábrázolhatók benne. Ennek egyik fő oka, hogy az emberek gondolkodás közben a világot gyakran úgy írják le, mint ,,dolgok'' (csomópontok), amelyek egymással valamilyen ,,viszonyban'', ,,kapcsolatban'' vannak (élek).

\begin{pelda}
Irányítatlan gráffal ábrázolhatjuk emberek között az \emph{ismerőse} relációt, ilyen pl. a Facebook \emph{friendship} relációja.
\end{pelda}

\begin{pelda}
Irányított gráffal ábrázolhatjuk emberek között a \emph{követi} relációt, ilyen pl. a Twitter \emph{follows} relációja. Szintén irányított gráffal ábrázolhatók az interneten egymásra mutató weboldalak.
\end{pelda}

Természetesen irányítatlan gráfok ábrázolhatók irányítottként is, ha minden $\{v_1, v_2\}$ irányítatlan élet lecserélünk. (Érdekesség: megfelelő transzformációval irányított gráfok is ábrázolhatók irányítatlan gráfként~\cite{rodriguez2008mapping}.)

Gyakran nincs szükségünk a teljes gráfra.

\begin{definicio}
Egy $G = (V, E)$ gráfnak $G' = (V', E')$ \fogalomragozva{reszgraf}{részgráfja}, ha $G'$ megkapható úgy, hogy $G$-ből elhagyunk néhány csomópontot és azok éleit, ill. néhány további élet. Formálisan: ha $V' \subseteq V$ és $E' \subseteq V' \times V'$.
\end{definicio}	

\subsection{Kiterjesztések}

Sokszor olyan rendszert szeretnénk modellezni, amihez a gráf struktúra nem rendelkezik elég leíróerővel, azaz a rendszer gráf-modellje túlságosan nagy, bonyolult, és nehezen átlátható. Ilyenkor hasznos lehet a gráfok valamely kiterjesztett változatát használni, mint modellezőeszköz.

\subsubsection{Címkézett gráfok}

\Fogalom{cimkezett-graf} használata esetén a gráf elemeit (csomópontjait és/vagy éleit) \fogalomragozva{cimke}{címkékkel} láthatjuk el. A címkézés célja lehet \fogalom{egyedi-azonosito} hozzárendelése vagy bizonyos tulajdonság leírása (pl. csomópontok kapacitása, élek típusa).
Ha csak a gráf éleihez (csúcsaihoz) rendelünk címkéket, \fogalomragozva{elcimkezett-graf}{élcímkézett} (\fogalomragozva{csucscimkezett-graf}{csúcscímkézett}) gráfról beszélünk.

\subsubsection{Típusos gráfok}

A \fogalomragozva{tipusos-graf}{típusos gráfok} a címkézett gráfok speciális esetei, ahol a gráf elemei 
 \fogalomragozva{tipus}{típusokkal} rendelkeznek.
 
\subsubsection{Tulajdonsággráfok}

\Fogalom{tulajdonsaggraf} esetén az egyes csomópontokat és éleket tulajdonságokkal látjuk el. A tulajdonságok kulcs-érték párok, pl. "neve = Gipsz Jakab", "életkora = 20", "színe = zöld".

\subsubsection{További kiterjesztések}

A gráf adatmodellnek többféle kiterjesztése is létezik. A fentiek mellett lehetséges további kiterjesztés például a \fogalom{sulyozott-graf}, amely egy speciális címkézett gráf, ahol a gráf elemeit súlyokkal látjuk el (pl. kapacitást vagy költséget rendelünk hozzájuk); a \fogalom{multigraf}, ahol az egyes csomópontok között \fogalomragozva{tobbszoros-el}{többszörös élek} is futhatnak; vagy a hipergráf, ahol egy él (ún. \fogalom{hiperel}) több csomópont között is futhat.


\subsection{Körmentes gráfok}

A strukturális modellezés szempontjából kiemelt szerepet játszanak azok a gráfok, amelyekben nem található bizonyos értelemben vett kört alkotó élsorozat. A kör definiálásához először szükség van a séta definíciójára.

\begin{definicio}
	\Fogalom{seta}: szomszédos csúcsok és élek váltakozó sorozata.
\end{definicio}

Ha egy séta során egy gráfelemet sem érintünk többször, útról beszélünk.

\begin{definicio}
\Fogalom{ut}: olyan séta, amely nem metszi önmagát, valamint első és utolsó csúcsa különbözik.
\end{definicio}

Körről beszélünk, ha egy út során ugyanoda érünk vissza, ahonnan elindultunk.

\begin{definicio}
\Fogalom{kor}: olyan séta, amely nem metszi önmagát, valamint első és utolsó csúcsa megegyezik.
\end{definicio}

Ezek ismeretében már beszélhetünk körmentes gráfokról.

\begin{definicio}
	A körmentes, összefüggő gráfokat \fogalomragozva{fa-graf}{fának} nevezzük.
\end{definicio}

\begin{definicio}
	A körmentes gráfokat \fogalomragozva{erdo}{erdőnek} nevezzük. Az \fogalom{erdo} gráfot egyes források \emph{ligetnek} nevezik.
\end{definicio}

A fák esetén gyakran kiemelt szerepet tulajdonítunk egy csomópontnak.

\begin{definicio}
	\Fogalom{gyoker-csomopont}: a fa egy megkülönböztetett csomópontja.
\end{definicio}

\begin{definicio}
	\Fogalom{gyokeres-fa}: olyan fa, ami rendelkezik gyökér csomóponttal.
\end{definicio}

\begin{definicio}
	\Fogalom{gyokeres-szintezett-fa}: a fa csomópontjaihoz hozzárendeljük a gyökértől vett távolságukat.
\end{definicio}

\subsection{Műveletek gráfokon}

\paragraph{Elérhetőségvizsgálat}

Egy $v_i$ csomópontból egy $v_j$ csomópont elérhető, ha létezik $v_i$-ből $v_j$-be irányított út.

Ez elérhetőségvizsgálatra gyakran elemi bejáróalgoritmusokat használunk:

\begin{itemize}
\item \rovidites{BFS}
\item \rovidites{DFS}
\end{itemize}

\paragraph{Útkereső algoritmusok}

\begin{itemize}
\item \fogalom{Dijsktra-algoritmus}
\item \fogalom{A-csillag-kereses} (ejtsd: ,,á-csillag keresés'')
\item \fogalom{Bellman-Ford-algoritmus}
\item \fogalom{Floyd-algoritmus}
\end{itemize}

A legrövidebb út keresése kipróbálható az alábbi példán: \url{http://gist.neo4j.org/?6d2a1a1c6325043d09a0}.

\paragraph{Transzformációk}

\begin{itemize}
\item \Fogalom{szures-eltipusra}: gyakran csak bizonyos éltípus(ok)ra van szükségünk. Egy $G = (V, E)$ gráf adott $c_1, \dots, c_n$ élcímkékre szűrt részgráfja $G' = (V, E)$, ahol $V$ a csomópontok változatlan halmaza, míg $E'$ azon $e$ élek halmaza, ahol $e$ címkéje eleme a $c_1, \dots, c_n$ halmaznak.
\end{itemize}



\section{Tulajdonságmodellezés}

Egy modellelem (gráfok esetén tipikusan a modell csomópontjai) tulajdonságait \fogalomragozva{jellemzo}{jellemzőkkel} vagy más néven \fogalomragozva{attributum}{attribútumokkal} jellemezzük. Különböző modellelemekre az egyes jellemzők más-más értéket vehetnek fel (de nem szükségszerűen térnek el). A jellemzők nem is feltétlenül értelmezettek minden modellelemre (ún. parciális függvényként írhatóak le a matematika nyelvén).

A jellemzők között gyakran kitüntetünk egyet:

\begin{definicio}
A \fogalom{tipus} egy kitüntetett jellemző, amely tipikusan állandó egy modellelem életciklusa során. A többi jellemzőt \fogalomragozva{tulajdonsag}{tulajdonságnak} hívjuk.
\end{definicio}

Azonos típussal rendelkező modellelemek közös tulajdonságokkal bírnak.

\subsection{Táblázatos ábrázolás}

A tulajdonságokat gyakran ábrázoljuk táblázatos formában, ún. \fogalomragozva{relacio}{relációkban}. A reláció pontos halmazelméleti definíciójával és a relációkon értelmezett műveleteket definiáló \fogalomragozva{relacioalgebra}{relációalgebrával} bővebben az \adatb c. tárgy foglalkozik. Most egy egyszerűbb, kevésbé formális definíció alkalmazunk:

\begin{definicio}
A \fogalom{relacio} a modellelemeket és azok jellemzőit tárolja. A táblázat egy sora a modell egy elemének, egy oszlopa egy jellemzőjének felel meg.
\end{definicio}

\begin{pelda}
Vegyük az alábbi táblázatot.
\begin{verbatim}
| név            | fénykard színe | nem    | holdak száma | keringési idő |
|----------------|----------------|--------|--------------|---------------|
| Alderaan       |                |        | 1            | 364           |
| Coruscant      |                |        | 4            | 368           |
| Darth Vader    | piros          | férfi  |              |               |
| Jabba          |                | hímnős |              |               |
| Leia Organa    |                | nő     |              |               |
| Luke Skywalker | zöld           | férfi  |              |               |
| Mace Windu     | lila           | férfi  |              |               |
| Tatooine       |                |        | 3            | 304           |
| Yoda           | zöld           | férfi  |              |               |
\end{verbatim}
\end{pelda}
Látható, hogy a táblázatnak nem minden celláját töltöttük ki, mivel az egyes jellemzők nincsenek mindenhol értelmezve. Ezeket a cellákat gyakran az \textsf{NA} (not applicable, not available) rövidítéssel vagy (különösen az adatbázisokban) \textsf{null} értékkel jellemezzük.

\subsubsection{Műveletek}

A relációkon most két alapműveletet definiálunk, ezek a \fogalom{szures} (relációalgebrában \fogalom{szelekcio}) és a \fogalom{vetites} (relációalgebrában \fogalom{projekcio}). Ezekkel a műveletekkel a modellünknek egy-egy nézetét állíthatjuk elő.

\paragraph{Szűrt nézet}

Szűrés során egy feltétel mentén bizonyos sorokat elhagyunk a relációból:

Szűrés a $\mathsf{fénykard színe\: zöld}$ feltételre:

\begin{verbatim}
| név            | fénykard színe | nem   | holdak száma | keringési idő |
|----------------|----------------|-------|--------------|---------------|
| Luke Skywalker | zöld           | férfi |              |               |
| Yoda           | zöld           | férfi |              |               |
\end{verbatim}

Szűrés a $\mathsf{holdak\: száma} > 2$ feltételre:

\begin{verbatim}
| név       | fénykard színe | nem | holdak száma | keringési idő |
|-----------|----------------|-----|--------------|---------------|
| Coruscant |                |     | 4            | 368           |
| Tatooine  |                |     | 3            | 304           |
\end{verbatim}

\paragraph{Vetített nézet}

Vetítés során a modell egyes jellemzőit elhagyjuk a relációból. Fontos, hogy az esetleg létrejövő üres sorokat is elhagyjuk.

Vetítés a $\{\mathsf{holdak\: száma, keringési\: idő\}}$ jellemzőkre.

\begin{verbatim}
| holdak száma | keringési idő |
|--------------|---------------|
| 1            | 364           |
| 4            | 368           |
| 3            | 304           |
\end{verbatim}

Vetítés a $\{\mathsf{név, fénykard\: színe\}}$ jellemzőkre.

\begin{verbatim}
| név            | fénykard színe |
|----------------|----------------|
| Alderaan       |                |
| Coruscant      |                |
| Darth Vader    | piros          |
| Jabba          |                |
| Leia Organa    |                |
| Luke Skywalker | zöld           |
| Mace Windu     | lila           |
| Tatooine       |                |
| Yoda           | zöld           |
\end{verbatim}

\subsection{Típusgráf}

\begin{definicio}
A \fogalom{tipusgraf} egy olyan gráf, amelyben minden csomóponttípushoz egy típuscsomópont, minden éltípushoz egy típusél tartozik.
\end{definicio}

Az elsőre talán furcsának ható definíció tulajdonképpen egy egyszerű fogalmat vezet be: a típusgráfban a csomópontok olyan típusokat jelölnek, amit a példánygráfban csomópont vehet fel (pl. ,,person'', ,,warrior'', ,,planet''), az ezek között futó élek pedig olyan típusokat, amiket a példánygráfban (adott típusú csomópontok között futó élek) vehetnek fel. Példa: ld. diasor, Star Wars szereplők, bolygók és kapcsolataik.

A típusgráf egy rendszer \fogalomragozva{metamodell}{metamodelljének} típusait és kapcsolatait ábrázolja.

A típusgráfot és példánygráfot egy gráfon ábrázolva a típus-példány viszonyok is megjeleníthetők: a példánygráf csomópontjaiból (pl. ,,Mace Windu'') a típusgráf csomópontjaira (pl. ,,Jedi'') \fogalomangolul{peldanya} (\fogalom{peldanya}) élek mutatnak. Szintén \fogalomangolul{peldanya} viszony áll fenn a példánygráf élei és a típusgráf élei között -- ennek ábrázolása azonban gráfként nehézkes.

Az \fogalom{peldanya} viszony helyett/mellett gyakran annak inverzét alkalmazzák, a \fogalomangolul{tipusa} (\fogalom{tipusa}) viszony is. Gráfban ábrázolva az \fogalomangolul{peldanya} és \fogalomangolul{tipusa} élek adott csomópontok között egymással ellentétes irányúak: pl. \emph{,,Mace Windu'' instance of ,,Jedi''} és \emph{,,Jedi'' type of ,,Mace Windu''}.

%%%%%%%%%%%%%%%%%%%%%%%%%%%%%%%%%%%%%%%%%%%%%%%%%%%%%%%%%%%%%%%%%%%%%%%%%%%%%%%%%%%%%%%%%%%%%%%%%%%%

\section{Gyakorlati alkalmazások (kieg.)}

Az alábbiakban bemutatunk néhány, a strukturális modellezés témaköréhez kapcsolódó gyakorlati technológiát, specifikációt és eszközt. Az itt felsorolt fogalmak nem részei a számonkérésnek, gyakran előkerülnek viszont a későbbi tanulmányok és munkák során, ezért mindenképpen érdemes legalább névről ismerni őket.

\subsection{Modellezési nyelvek}

A gyakorlatban rengeteg különböző modellezési nyelvet használnak, ezek közül mutatunk be most néhány elterjedtebbet.

\subsubsection{UML (Unified Modeling Language)}

Az UML egy általános célú modellezési nyelv az~\cite{UML}. Az UML három fő diagramtípust definiál:

\begin{itemize}
	\item \emph{Structure Diagram:} strukturális modellek leírására. A \emph{Class Diagram} az osztályok (metamodell), míg az \emph{Object diagram} a példányok (modell) leíárásra alkalmas. A \emph{Composite Structure Diagram} egy rendszer struktúráját és a rendszer komponenseinek kapcsolatát mutatja be.
	\item \emph{Behaviour Diagram:} viselkedési modellek leírására, pl. a \emph{State Machine Diagram} segítségével állapotgépekm az \emph{Activity Diagramon} folyamatok ábrázolhatók.
	\item \emph{Interaction Diagram:} a \emph{Behaviour Diagram} altípusa. Ennek megfelelően szintén a viselkedés leírása a célja, de a hangsúly a vezérlés- és adatáramlás bemutatásán van. Ilyen pl. a \emph{Sequence Diagram} (szekvenciadiagram), amely az egyes objektumok közötti interakciót mutatja be üzenetek formájában.
\end{itemize}

\remofigscale{struktura-alapu-modellezes/UML-diagram}{UML diagramok típusai és a közöttük lévő viszony osztálydiagramként ábrázolva~\cite{wiki:ISE}}{0.6}

Az UML nyelvvel részletesen foglalkozik a \emph{Szoftvertechnológia} c. tárgy.

\subsubsection{AADL}

Az \rovidites{AADL} eredetileg repülőipari célokra fejlesztett architektúraleíró nyelv~\cite{AADL}.

\subsubsection{SysML}

A \rovidites{SysML} egy UML-alapú általános modellezési nyelv rendszertervezési célokra~\cite{SysML}. A SysML az UML egy részhalmazát bővíti ki és új diagramtípusokat is bevezet, pl. \emph{Requirement Diagram} követelmények és viszonyaik ábrázolására alkalmas.

\subsubsection{EMF}

Az Eclipse fejlesztőkörnyezet~\cite{eclipse} saját modellezési keretrendszerrel rendelkezik, ez az \rovidites{EMF}. Az EMF metamodellező nyelve, az Ecore lehetővé teszi saját, ún. szakterület-specifikus nyelv (domain-specific language, DSL) definiálását. Az EMF mára több területen is \emph{de facto} modellezési keretrendszer, sikere nagyban hozzájárul az Eclipse népszerűségéhez.

Az Eclipse-szel és az EMF-fel foglalkozik az \emph{Eclipse-alapú technológiák} c. szabadon választható tárgy.

\subsubsection{AUTOSAR}

Az \rovidites{AUTOSAR}~\cite{autosar} nagy autóipari gyártók és beszállítóik által fejlesztett szabványos architektúranyelv, amely az egyes hardver- és szoftverkomponensek együttműködése magas szinten definiálható. Az AUTOSAR konzorciumnak a Méréstechnika és Információs Rendszerek Tanszék is tagja~\cite{autosar-attendees}.

\subsection{Struktúramodellezési technikák}

\subsubsection{Refaktoring}

A dekompozícióhoz, azaz faktoringhoz szorosan kapcsolódik a \fogalom{refaktoring} (\fogalomangolul{refaktoring}) fogalma~\cite{fowler2012refactoring}. Refaktoring során létező rendszerek, programkódok (vagy modellek) olyan átalakítását értjük, amelynek sorána megfigyelhető működés változatlan marad, de a kapott programkód (modell) olvashatóbb, érthetőbb, karbantarthatóbb lesz.

\subsection{Struktúramodellező eszközök és vizualizáció}

Gráfok automatikus megjelenítésére alkalmas pl. a GraphViz\footnote{\url{http://www.graphviz.org/}} programcsomag. Gráfok feldolgozására gyakran alkalmazzák az igraph\footnote{\url{http://igraph.org/}} programcsomagot. Manapság több gráfadatbázis-kezelő rendszer is elterjedt, pl. a Neo4j\footnote{\url{http://neo4j.com/}} és a Titan\footnote{\url{http://thinkaurelius.github.io/titan/}} rendszerek.

Gráfok manuális rajzolására szintén több eszköz elterjedt. Egyszerűen használható online felületet biztosít a draw.io\footnote{\url{http://draw.io/}} és az Arrow Tools\footnote{\url{http://www.apcjones.com/arrows/}}. Komolyabb célokra a yEd\footnote{\url{https://www.yworks.com/products/yed}} eszközt érdemes használni. Sok információt tartalmazó gráf esetén érdemes lehet vektorgrafikus rajzoló-, ill. prezentáló eszközök, pl. a Microsoft Visio vagy Microsoft PowerPoint alkalmazás.

%%%%%%%%%%%%%%%%%%%%%%%%%%%%%%%%%%%%%%%%%%%%%%%%%%%%%%%%%%%%%%%%%%%%%%%%%%%%%%%%%%%%%%%%%%%%%%%%%%%%

\section{Elméleti kitekintés (kieg.)}

A struktúramodellezésnek komoly matematikai eszköztára is van. Az alábbiakban ezekből mutatunk be néhány részletet.

\subsubsection{Tranzitivitás}

A \fogalom{tranzitiv-lezaras} fogalom megértését segítheti, ha ismerjük a relációkon értelmezett tulajdonságokat, köztük a \fogalomragozva{tranzitiv}{tranzitivitást}.

Egy $S$ halmazon értelmezett $r$ (kétváltozós) reláció \fogalom{tranzitiv}, ha bármely $a,b,c \in A$ esetén $r(a, b)$ és $r(b, c)$ teljesülése esetén $r(a, c)$ is teljesül.

\begin{itemize}
	\item Az egyenlőség ($=$) és a kisebb ($<$) reláció tranzitívak, mert
	\begin{itemize}
		\item $a = b$ és $b = c$ esetén $a = c$,
		\item $a < b$ és $b < c$ esetén $a < c$.
	\end{itemize}
	\item A nemegyenlő reláció ($\neq$) nem tranzitív, mert 
	\begin{itemize}
		\item $a \neq b$ és $b \neq c$ esetén $a \neq c$ nem mindig áll fenn, például
		\item $1 \neq 2$ és $2 \neq 1$ esetén $1 \neq 1$ nem teljesül.
	\end{itemize}
	\item Személyek közötti $\mathit{őse}$ reláció tranzitív, mert $\mathit{őse}(a, b)$ és $\mathit{őse}(b, c)$ esetén $\mathit{őse}(a, c)$ is fennáll. 
	\item Személyek közötti $\mathit{ismerőse}$ reláció nem tranzitív, mert $\mathit{ismerőse}(a, b)$ és $\mathit{ismerőse}(b, c)$ esetén nem garantált, hogy $\mathit{ismerőse}(a, c)$ fennáll.
\end{itemize}

\subsubsection{További tulajdonságok}

A Rendszermodellezés tárgynak nem témája, de érdemes megismerni a relációk további tulajdonságait: \fogalom{reflexivitas}, \fogalom{szimmetria} stb.~\cite{relaciok}. 

\subsection{Ontológia}

Az ontológia egy nem egyértelmű, nehezen definiálható kifejezés. Az informatikában általában az alábbihoz hasonló módon definiálják:

\begin{definicio}
Az \fogalom{ontologia} egy olyan taxonómia, amely tartalmazza a benne szereplő fogalmak közötti viszonyokat. 
\end{definicio}

%%%%%%%%%%%%%%%%%%%%%%%%%%%%%%%%%%%%%%%%%%%%%%%%%%%%%%%%%%%%%%%%%%%%%%%%%%%%%%%%%%%%%%%%%%%%%%%%%%%%

\section{Ajánlott irodalom}

Olvasmányos, jól érthető összefoglalót ad az UML nyelvről Martin Fowler ,,UML Distilled'' c. munkája~\cite{fowler1997uml}.

A tulajdonsággráfokról egy jól érthető tudományos cikk Marko Rodriguez és Peter Neubauer munkája~\cite{Rodriguez2010}. Rodriguez a Titan elosztott gráfadatbázis-kezelő rendszer egyik fő fejlesztője~\cite{Titan}, míg Neubauer a Neo4j gráfadatbázis-kezelőt fejlesztő cég alapítója~\cite{Neo4j}. Az elosztott gráfadatbázis alkalmazását, elméleti és gyakorlati kihívásait kiváló prezentációkban mutatja be~\cite{RodriguezSlides2012,RodriguezSlides2013}.

Barabási-Albert László magyar fizikus nemzetközileg elismert kutatója a komplex \fogalomragozva{halozat}{hálózatok} elméletének. Barabási ,,Behálózva'' c. könyve közérthető stílusban mutatja be a hálózatok elemzésének elméleti kihívásait a kutatási eredmények gyakorlati jelentőségét~\cite{behalozva}. A szerzővel több interjú is készült~\cite{barabasi1,barabasi2,barabasi3}.

Az osztályok és prototípusok közötti elvi különbséget mutatja be Antero Taivalsaari, a Nokia Research fejlesztőjének 1996-os cikke~\cite{taivalsaari1996classes}.
