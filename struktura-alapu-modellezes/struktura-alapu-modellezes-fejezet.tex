\topic{Struktúra alapú modellezés}

\graphicspath{ {./struktura-alapu-modellezes/figures/} }

\newcommand{\yedscale}{0.7}

% vezérpélda lehetőségek
% - táblázatszerű adatok pl. felhasználókról (preferenciák, van-e jogsija, stb.), kocsikról (férőhely, légkondi stb.)
% - kiszolgáló rendszer top-down, bottom-up tervezése (mobil kliens, tömegközl. integráció, adatbázis stb.)
% - gráfként reprezentált fuvarok, kocsik, szakaszok, utasok, sofőrök stb.

%\begin{megjegyzes}
%	kifejteni azt, hogy lehet logikai és fizikai felépítés alapján is modellezni.
%	
%	- Fizikai/architektúrális egységek --> moduláris struktúra hoz létre, újrahasznosítható részekkel.
%	
%	- Funkcionális egységek --> dedikált struktúrát hoz létre, pl. távirányító -- redundanciamentes, azért, hogy a költség alacsony legyen. Nem újrahasznosíthatók a részei.
%	
%	Pl. dedikált eszközt készítünk, amikor van egy rendszermenedzsment problémám és írok rá egy szkriptet. A másik megoldás lenne, hogy behozunk egy rendszermenedzsment eszközt és annak az újrahasznosítható komponenseit alkalmazzuk.
%	
%	Jó lenne építészeti példát behozni.
%\end{megjegyzes}

Hogyan épül fel egy rendszer? Milyen részekre bontható és ezek között milyen kapcsolat van? Ahhoz, hogy a rendszer egészével kapcsolatos összetett problémákat megválaszoljunk, fontos, hogy tudjuk ezekre a kérdésekre a választ. Ebben a fejezetben az egyes rendszerek \fogalomragozva{struktura}{struktúrájának} jellemzésével foglalkozunk. Bemutatjuk a strukturális modellezés motivációját, a leggyakrabban alkalmazott formalizmusokat és azok matematikai alapjait.

%Modellezés során egy rendszert jellemezhetünk annak statikus, ill. dinamikus jellemzői alapján. A statikus jellemzés esetén a rendszer felépítését, \fogalomragozva{struktura}{struktúráját} írjuk le, míg dinamikus jellemzés esetén a rendszer \fogalomragozva{viselkedes}{viselkedésével} foglalkozunk.

%%%%%%%%%%%%%%%%%%%%%%%%%%%%%%%%%%%%%%%%%%%%%%%%%%%%%%%%%%%%%%%%%%%%%%%%%%%%%%%%

\section{Motiváció: a struktúra formái}
\label{sec:motivacio}

Mind a természetben előforduló, mind a mesterséges rendszerekben fellelhetők bizonyos szabályszerűségek. Egyes szabályszerűségek a rendszer elemei közötti kapcsolatokat, míg mások magukat az elemeket jellemzik.

\clearpage

\subsection{Hálózatok}

%\fogalomragozva{graf}{gráfok}

Egy rendszert gyakran úgy jellemezhetünk a legjobban, ha bizonyos elemeit megkülönböztetjük és leírjuk az ezek közötti \fogalomragozva{kapcsolat}{kapcsolatot}.

\begin{pelda}
	Egy nagyváros közlekedési hálózatában szövevényes rendszere az út- és sínhálózatnak, a több százezernyi járműnek és az ezeken utazó embereknek. A közlekedők mellett az infrastruktúra is folyamatosan változik a különböző fejlesztések, átalakítások és karbantartások miatt.
	
	Egy ilyen rendszer precíz modellezése lehetetlen vállalkozás lenne, ezért helyette tipikusan olyan absztrakciókkal dolgozunk, amik az adott probléma megoldásához szükséges információkat tartalmazzák. Például az útvonalkereső alkalmazások ismerik a helyi tömegközlekedés járatait és segítséget nyújtanak abban, hogy az indulási pontunktól a célállomásig közlekedő járműveket és az azok közötti átszállási lehetőségeket listázzák.
	
	Vizsgáljuk meg például Budapest metróhálózatát. Az egyszerűség kedvéért a példában  a hálózatnak csak a Nagykörúton és az azon belüli részével foglalkozunk (\ref{fig:budapest-metrohalozat-nagykorut-terkep}. ábra).
\end{pelda}

	A metróhálózat könnyen leképezhető egy gráfra, ahol a gráf minden \fogalomragozva{csomopont}{csomópontja} egy-egy metróállomásokat jelöl. A csomópontok címkéje a metrómegálló neve. Két csomópont között akkor fut \fogalom{el}, ha a két megálló közvetlenül össze van kötve metróval. A metróhálózatot modellező \fogalomragozva{iranyitatlan-graf}{irányítatlan}, \fogalomragozva{cimkezett-graf}{címkézett}, \fogalom{egyszeru-graf} \aref{fig:budapest-metrohalozat-nagykorut-egyszeru}. ábrán látható.

	\begin{figure}[H]
		\begin{subfigure}[b]{0.5\textwidth}
			\centering
			\includegraphics[width=0.9\textwidth]{budapest-metrohalozat-nagykorut-terkep}
			\caption{Budapest metróhálózata a Nagykörúton belül. TODO: forrást megjelölni}
			\label{fig:budapest-metrohalozat-nagykorut-terkep}
		\end{subfigure}
		~ %add desired spacing between images, e. g. ~, \quad, \qquad, \hfill etc. 
		%(or a blank line to force the subfigure onto a new line)
		\begin{subfigure}[b]{0.5\textwidth}
			\includegraphics[width=\textwidth]{budapest-metrohalozat-nagykorut-egyszeru}
			\caption{Budapest metróhálózata a Nagykörúton belül, egyszerű gráfként ábrázolva}
			\label{fig:budapest-metrohalozat-nagykorut-egyszeru}
		\end{subfigure}
		\caption{}\label{fig:animals}
	\end{figure}

%	\remofigscaleframed{budapest-metrohalozat-nagykorut-egyszeru}{Budapest metróhálózata a Nagykörúton belül, egyszerű gráfként ábrázolva}{\yedscale}

%	\begin{kisdefinicio}
%		A \fogalom{graf} egy olyan $G = (V, E)$ struktúra, ahol $V$ halmaz a csomópontok, $E$ az élek halmaza. Az élek csomópontok között futnak, \fogalom{iranyitatlan-graf} esetén az $E$ halmaz csomópontok rendezetlen $\{v_1, v_2\}$ párjaiból áll (tehát nem különböztetjük meg a $\{v_1, v_2\}$ és a $\{v_2, v_1\}$) párokat, míg \fogalom{iranyitott-graf} esetén csomópontok rendezett $(v_1, v_2)$ párjaiból.
%%	\end{kisdefinicio}
%	
%%	\begin{kisdefinicio}
%		\fogalomragozva{cimkezett-graf}{Címkézett gráf} esetén a gráf elemeit (csomópontjait és/vagy éleit) \fogalomragozva{cimke}{címkékkel} láthatjuk el. A címkézés célja lehet \fogalom{egyedi-azonosito} hozzárendelése vagy bizonyos tulajdonság leírása (pl. a csomópontok kapacitása, élek típusa).
%%	\end{kisdefinicio}
%	
%%	\begin{kisdefinicio}
%		Egy gráf \fogalomragozva{egyszeru-graf}{egyszerű}, ha nem tartalmaz többszörös és hurokélet.
%	\end{kisdefinicio}

	A modellünk segítségével választ kaphatunk például a következő kérdésekre:
	
	\begin{enumerate}
		\item Milyen megállók érhetők el a Vörösmarty térről indulva?
		
			Vizsgáljuk meg, hogy a Vörösmarty teret reprezentáló csomópontból kiindulva milyen csomópontok érhetők el. Ehhez elemi bejáróalgoritmusokat használunk, pl. \fogalomragozva{BFS}{szélességi keresést} (\rovidites{BFS}) vagy \fogalomragozva{DFS}{mélységi keresést} (\rovidites{DFS}).
		\item Hány megállót kell utaznunk a Kossuth Lajos tér és a Kálvin tér között?

			A legrövidebb utat \fogalomragozva{BFS}{szélességi kereséssel} határozhatjuk meg.
	\end{enumerate}

	Vannak azonban olyan metróközlekedéssel kapcsolatos kérdések, amelyekhez a modell nem tartalmaz elég információt:
	
	\begin{enumerate}
		\item Milyen megállók érhetők el a Fővám térről indulva \emph{legfeljebb egy átszállással}?
		\item A menetrend szerint \emph{hány percig tart} az út a Kossuth Lajos tér és az Astoria között?
	\end{enumerate}

	Ezeknek a kérdéseknek megválaszolásához egészítsük ki a gráfot! Az első kérdéshez szükséges, hogy az egyes metróvonalakat meg tudjuk különböztetni, amit például az élek címkézésével tehetünk meg. \Aref{fig:budapest-metrohalozat-nagykorut-elcimkezett}. ábrán színekkel jelöltük a különböző élcímkéket. Induljunk ki a Fővám térről: átszállás nélkül az Astoria és a Rákóczi tér megállókat érhetjük el, míg egy átszállással elérhetjük a 3-as metró vonalán található megállókat is.

	\begin{figure}[H]
		\begin{subfigure}[b]{0.5\textwidth}
			\includegraphics[width=\textwidth]{budapest-metrohalozat-nagykorut-elcimkezett}
			\caption{A gráf élcímkékkel kiegészítve.}
			\label{fig:budapest-metrohalozat-nagykorut-elcimkezett}
		\end{subfigure}
		~ %add desired spacing between images, e. g. ~, \quad, \qquad, \hfill etc. 
		%(or a blank line to force the subfigure onto a new line)
		\begin{subfigure}[b]{0.5\textwidth}
			\includegraphics[width=\textwidth]{budapest-metrohalozat-nagykorut-elsulyozott}
			\caption{A gráf élsúlyokkal kiegészítve.}
			\label{fig:budapest-metrohalozat-nagykorut-elsulyozott}
		\end{subfigure}
		\caption{A metróhálózatot ábrázoló gráf kiterjesztései.}\label{fig:animals}
	\end{figure}
	
%	\begin{kisdefinicio}
%		Ha precízen szeretnénk jellemezni a gráfot, a következő terminológiát használhatjuk: ha csak a csomópontokhoz rendelünk címkéket, \fogalomragozva{csucscimkezett-graf}{csúcscímkézett} gráfról beszélünk, míg ha csak a gráf éleihez rendelünk címkéket, \fogalomragozva{elcimkezett-graf}{élcímkézett} gráfról beszélünk. 
%	\end{kisdefinicio}

	A második kérdés megválaszolásához az egyes megállók közötti útidőt kell jellemeznünk. Ehhez vegyünk fel \fogalomragozva{elsuly}{élsúlyokat} a gráfba. \Aref{fig:budapest-metrohalozat-nagykorut-elsulyozott}. ábrán élsúlyokkal jelöltük az egyes megállók között menetidőt. Ezek ismeretében meghatározható a Kossuth Lajos tér és a Kálvin tér közötti út menetrend szerinti időtartama. Ez a modell arra is alkalmas, hogy meghatározzuk a legrövidebb utat a két csomópont között, például a \fogalom{Dijsktra-algoritmus} segítségével.
		
%	\begin{kisdefinicio}
%		A \fogalom{sulyozott-graf} olyan $(G, w)$ pár, ahol $G = (V, E)$ gráf, $w: E \rightarrow \mathbb{R}$ egy súlyfüggvény.
%	\end{kisdefinicio}
	
	A metróhoz hasonlóan sok rendszer jól modellezhető hálózattal: az élőlények táplálkozási lánca, az egyes személyek közösségi hálója, az úthálózat,  telekommunikációs hálózatok stb.
	

\subsection{Hierarchikus rendszerek}

\begin{pelda}
	Szervezési okokból Budapestet 23 kerületre osztották, amelyek további városrészekből állnak. %\footnote{\url{https://hu.wikipedia.org/wiki/Budapest_v\%C3\%A1rosr\%C3\%A9szeinek_list\%C3\%A1ja}} A metróhálózat megálló különböző városrészekben találhatók. 
	Melyik városrészben van az Opera metrómegállója? Melyik városrészben van a legtöbb metrómegálló? Ezekhez hasonló kérdésekre úgy tudunk hatékonyan válaszolni, ha készítünk egy \fogalomragozva{hierarchikus-modell}{hierarchikus modellt} a problémáról.
\end{pelda}

\remofigscale{budapest-keruletek-varosreszek-metromegallok}{Budapest kerületei, városrészei és metrómegállói (részleges modell)}{0.5}

Készítsünk modellt, amely ábrázolja Budapest, a kerületek, a városrészek és a metrómegállók viszonyát. \Az+\refstruc{fig:budapest-keruletek-varosreszek-metromegallok} modellje négy szintet tartalmaz:

\begin{enumerate}
	\item Az első szinten a hierarchia legfelső eleme, Budapest szerepel.
	\item A második szinten a város kerületei találhatók.
	\item A harmadik szinten az egyes városrészek vannak.
	\item A negyedik szinten a metrómegállók találhatók.
\end{enumerate}

Látható, hogy a hierarchikus modellt is ábrázolhatjuk gráfként. A csomópontok a modell különböző szintű elemeit reprezentálják, míg az élek a \fogalomragozva{tartalmazas}{része} viszonyt fejezik ki, például az Opera megálló a VI.~kerület része. A gráf \fogalomragozva{gyoker-csomopont}{gyökér csomópontja} a hierarchiában legmagasabban szereplő elem, Budapest.

Amennyiben egy rendszert hierarchikusan részekre bontunk, nem fordulhat elő, hogy egy elem tartalmazza a szülő elemét, ezért a hierarchikus modelleket reprezentáló gráfok körmentesek. A gyökér elemet leszámítva minden elemnek van szülője, tehát a gráf összefüggő is, így a hierarchikus modellek \fogalomragozva{fa-graf}{fa gráffal} ábrázolhatók.

%\begin{kisdefinicio}
%	%azok a gráfok, amelyekben nem található bizonyos értelemben vett kört alkotó élsorozat. A kör definiálásához először szükség van a séta definíciójára.
%	A \fogalom{seta} szomszédos csúcsok és élek váltakozó sorozata, mely csúccsal kezdődik és csúcsban végződik. %Ha egy séta során egy gráfelemet sem érintünk többször, útról beszélünk, tehát
%	Az \fogalom{ut} olyan séta, amely nem metszi önmagát, valamint első és utolsó csúcsa különbözik. A \fogalom{kor} olyan séta, amely nem metszi önmagát, valamint első és utolsó csúcsa megegyezik. %Körről beszélünk, ha egy út során ugyanoda érünk vissza, ahonnan elindultunk.	
%	%Ezek ismeretében már beszélhetünk körmentes gráfokról.
%	A körmentes, összefüggő gráfokat \fogalomragozva{fa-graf}{fának} nevezzük. (A körmentes gráfokat \fogalomragozva{erdo}{erdőnek} nevezzük.) %Az \fogalom{erdo} gráfot egyes források \emph{ligetnek} nevezik.
%	A fák esetén gyakran kiemelt szerepet tulajdonítunk egy csomópontnak: a \fogalom{gyoker-csomopont} a fa egy megkülönböztetett csomópontja. A \fogalom{gyokeres-fa} olyan fa, ami rendelkezik gyökér csomóponttal. \Fogalom{gyokeres-szintezett-fa} esetén a fa csomópontjaihoz hozzárendeljük a gyökértől vett távolságukat is.
%
%	A strukturális modellezés szempontjából kiemelt szerepet játszanak az irányított körmentes gráfok.
%	Egy gráf \rovidites{DAG} (\roviditesangolul{DAG}), ha nem tartalmaz irányított kört.	
%\end{kisdefinicio}


%\begin{megjegyzes}
%	A fa struktúrában a gráf élei \emph{implicit módon} jelölik a tartalmazási hierarchiát.
%	
%	A tartalmazási hierarchia ábrázolható a bennfoglalási viszonyok \emph{explicit megjelenítésével} is. Ebben az esetben egy olyan diagramot rajzolunk, amelyben az egyes téglalapok közötti tartalmazás reprezentálja a modellben szereplő tartalmazási viszonyt.
%	
%	\begin{center}
%		\includegraphics[scale=0.45]{budapest-explicit-tartalmazas}
%	\end{center}
%\end{megjegyzes}

Láthattuk tehát, hogy mind a modellelemek közötti kapcsolat, mind a modellhierarchia hatékonyan ábrázolható gráfként. \Aref{fig:budapest-metrohalozat-hierarchiaval}. ábrán látható modell a metróhálózatot és a területek hiearchiáját is tartalmazza.

\remofigscale{budapest-metrohalozat-hierarchiaval}{Budapest metróhálózata és a városrészek, kerületek hierarchiája}{\yedscale}

\subsection{Tulajdonságok}

\begin{pelda}
	Melyik a legnagyobb teljesítményű jármű? Milyen jellemzőket kell rögzíteni egy újonnan beszerzett buszról? Ezek a kérdések a modellünk elemeinek tulajdonságaival kapcsolatban merülnek fel.
\end{pelda}

Készítsünk egy modellt, ami a különböző járműveket tartalmazza. Ezt a modellt gráf helyett inkább táblázatos formában érdemes ábrázolni, így jobban áttekinthető lesz.

\begin{table}[H]
	\centering
	\begin{tabular}{|l|l|l|l|l|r|l|l|}
		\hline
		név                   & típus    & tömeg & hossz & lökettérfogat & \#motorok & telj. & nyomtáv \\ \hline\hline
		Volvo 7700            & busz     & 18 t  & 12 m  & 9000 cm$^3$   &               & 320 LE       &         \\ \hline
		Siemens Combino Supra & villamos & 70 t  & 54 m  &               & 8             & 800 kW       & 1800 mm \\ \hline
		XY & villamos & a t  & b m  &               & 8             & yy kW       & 1800 mm \\ \hline
		Alstom Metropolis     & metró    &       &       &               &               &              & 1435 mm \\ \hline
	\end{tabular}
	\caption{Járművek tulajdonságai.}
	\label{tab:jarmuvek}
\end{table}

A modellünk elemei itt a táblázat sorai. A modellelemek tulajdonságait \fogalomragozva{jellemzo}{jellemzőkkel} vagy más néven \fogalomragozva{attributum}{attribútumokkal} írjuk le. Mint \aref{tab:jarmuvek}. táblázatban látható, különböző modellelemekre az egyes jellemzők más-más értéket vehetnek fel. Látható, hogy a táblázatnak nem minden celláját töltöttük ki, mivel az egyes jellemzők nincsenek mindenhol értelmezve. Ezeket a cellákat gyakran az \rovidites{NA} (\roviditesangolul{NA}) rövidítéssel vagy \textsf{null} értékkel jellemezzük. Megfigyelhetjük azonban, hogy az azonos \fogalomragozva{tipus}{típussal} rendelkező modellelemek azonos tulajdonságokra vesznek fel értéket.

%A jellemzők nem is feltétlenül értelmezettek minden modellelemre (ún. \fogalomragozva{parcialis-fuggveny}{parciális függvényként} írhatóak le a matematika nyelvén).


%%%%%%%%%%%%%%%%%%%%%%%%%%%%%%%%%%%%%%%%%%%%%%%%%%%%%%%%%%%%%%%%%%%%%%%%%%%%%%%%

\section{Struktúramodellezés}

Ahogy \az+\refstruc{sec:motivacio} példáiban láttuk, a struktúramodellezés célja, hogy a rendszer felépítését jellemezze, beleértve az egyes elemek típusát, a közöttük lévő kapcsolatokat és az elemek tulajdonságait.

\begin{definicio}
	A \fogalom{strukturalis-modell} a rendszer felépítésére vonatkozó tudás.
\end{definicio}

%A jellemzők között gyakran kitüntetünk egyet, a típust. Azonos típussal rendelkező modellelemek közös tulajdonságokkal bírnak.
%
%\begin{definicio}
%	A \fogalom{tipus} egy kitüntetett jellemző, amely tipikusan állandó egy modellelem életciklusa során. A többi jellemzőt \fogalomragozva{tulajdonsag}{tulajdonságnak} hívjuk.
%\end{definicio}

A struktúramodellezés során meghatározzuk a rendszerünkben található elemek típusait és a típusok közötti kapcsolatokat. A típusok alapján meghatározhatjuk az egyes elemekhez tartozó tulajdonságokat is. Egy rendszer strukturális modellje tehát alkalmas arra, hogy az alábbi kérdésekre (nem feltétlenül kimerítő) választ nyújtson:

\begin{itemize}
	\item Milyen elemekből áll a rendszer?
	\item Hogyan kapcsolódnak egymáshoz az elemek?
	\item Milyen tulajdonságúak a rendszer elemei?
\end{itemize}

\subsection{Dekompozíció}

Egy rendszer strukturális modellje a rendszer dekompozíciójával állítható elő.

\begin{definicio}
	A \fogalom{dekompozicio} (\fogalom{faktoring}) egy összetett probléma vagy rendszer kisebb részekre bontása, amelyek könnyebben érthetők, fejleszthetők és karbantarthatók.
\end{definicio}

A rendszer így kapott egyes komponensei (részrendszerei) gyakran további dekompozícióval még kisebb részekre bonthatóak. Természetesen a dekompozíció során ügyelnünk kell arra, hogy az egyes részekből visszaállítható legyen az eredeti rendszer, különben a kapott strukturális modellünk hiányos.

\begin{definicio}
	Egy dekompozíció \fogalomragozva{helyes-dekompozicio}{helyes}, ha a dekompozícióval kapott rendszer minden elemének megfeleltethető az eredeti rendszer valamelyik eleme, és az eredeti rendszer minden eleméhez hozzárendelhető a dekompozícióval kapott rendszer egy vagy több eleme.
\end{definicio}

%\subsubsection{Hierarchia modellezése}

A gyakorlatban a hierarchia modellezésére két fő megközelítést szoktak alkalmazni: a \fogalomangolul{top-down} és a \fogalomangolul{bottom-up} modellezést.

\begin{definicio}
	\Fogalomangolul{top-down} modellezés során a rendszert felülről lefelé építjük. A modellezés alaplépése a \fogalom{dekompozicio}.
\end{definicio}

A top-down modellezés jellemzői:

\begin{itemize}
	\item[$\oplus$] Részrendszer tervezésekor a szerepe már ismert
	\item[$\ominus$] ,,Félidőben'' még nincsenek működő részek
	\item[$\ominus$] Részek problémái, igényei későn derülnek ki
\end{itemize}

\begin{definicio}
	\Fogalomangolul{bottom-up} modellezés során a rendszert alulról felfelé építjük. A modellezés alaplépése a \fogalom{kompozicio}: az egész rendszer összeszerkesztése külön modellezett vagy fejlesztett részrendszerekből.
\end{definicio}

A bottom-up modellezés jellemzői:

\begin{itemize}
	\item[$\oplus$] A rendszer részei önmagukban kipróbálhatók, tesztelhetők
	\item[$\oplus$] Részleges készültségnél könnyebben előállítható a rendszer prototípusa
	\item[$\ominus$] Nem látszik előre a rész szerepe az egészben
\end{itemize}



\subsection{Típusgráf}

\begin{definicio}
	A \fogalom{tipusgraf} egy olyan gráf, amelyben minden csomóponttípushoz egy típuscsomópont, minden éltípushoz egy típusél tartozik.
\end{definicio}

A típusgráfban tehát a csomópontok olyan típusokat jelölnek, amit a példánygráfban csomópont vehet fel ..., az ezek között futó élek pedig olyan típusokat, amiket a példánygráfban (adott típusú csomópontok között futó élek) vehetnek fel. 

A típusgráf egy rendszer \fogalomragozva{metamodell}{metamodelljének} típusait és kapcsolatait ábrázolja.

A típusgráfot és példánygráfot egy gráfon ábrázolva a típus-példány viszonyok is megjeleníthetők: a példánygráf csomópontjaiból a típusgráf csomópontjaira \fogalomangolul{peldanya} (\fogalom{peldanya}) élek mutatnak. Szintén \fogalomangolul{peldanya} viszony áll fenn a példánygráf élei és a típusgráf élei között -- ennek ábrázolása azonban gráfként nehézkes.

Az \fogalom{peldanya} viszony helyett gyakran annak inverzét, a \fogalom{tipusa} (\fogalomangolul{tipusa}) viszonyt. Gráfban ábrázolva az \fogalom{peldanya} és \fogalom{tipusa} élek adott csomópontok között egymással ellentétes irányúak: 



\subsection{Jellemzők}

A tulajdonságokat gyakran ábrázoljuk táblázatos formában, ún. \fogalomragozva{relacio}{relációkban}. A reláció pontos halmazelméleti definíciójával és a relációkon értelmezett műveleteket definiáló \fogalomragozva{relacioalgebra}{relációalgebrával} bővebben az \adatb tárgy foglalkozik. Most egy egyszerűbb, kevésbé formális definíció alkalmazunk:

\begin{definicio}
A \fogalom{relacio} a modellelemeket és azok jellemzőit tárolja. A reláció egy eleme a modell egy elemének, egy attribútum egy jellemzőjének felel meg. A relációkat gyakran táblázatos formában ábrázoljuk.
\end{definicio}












\section{Nézetek}

A struktúramodellekből különböző \fogalomragozva{nezet}{nézeteket} állíthatunk elő a \fogalom{szures} és a \fogalom{vetites} műveletekkel.

\paragraph{Szűrt nézet}

\begin{definicio}
	\Fogalom{szures} során egy feltétel mentén bizonyos sorokat elhagyunk a relációból. \todo{formalisan is irjuk le}
\end{definicio}

\begin{megjegyzes}
	A relációalgebrában a \fogalom{szures} művelet neve \fogalom{szelekcio}.
\end{megjegyzes}


\paragraph{Vetített nézet}

\begin{definicio}
	\Fogalom{vetites} során a modell egyes jellemzőit elhagyjuk a relációból. Fontos, hogy az esetleg létrejövő üres sorokat is elhagyjuk. \todo{formalisan is irjuk le: lehet olyan vetites is, ahol nem hagyunk el jellemzoket}
\end{definicio}

\begin{megjegyzes}
	A relációalgebrában a \fogalom{vetites} művelet neve \fogalom{projekcio}.
\end{megjegyzes}

%%%%%%%%%%%%%%%%%%%%%%%%%%%%%%%%%%%%%%%%%%%%%%%%%%%%%%%%%%%%%%%%%%%%%%%%%%%%%%%%%%%%%%%%%%%%%%%%%%%%

\section{Összefoglalás}

A fejezetben bemutattuk \fogalom{struktura-alapu-modellezes} motivációját, a használt formalizmusukat és azok alkalmazásait. Ismertettük \fogalomragozva{tipus}{típusok} fontosságát és a típusrendszer ábrázolásának lehetőségeit.

A következő fejezetekben a \fogalomragozva{viselkedes-alapu-modellezes}{viselkedés alapú modellezést} és annak formalizmusait mutatjuk be.

%%%%%%%%%%%%%%%%%%%%%%%%%%%%%%%%%%%%%%%%%%%%%%%%%%%%%%%%%%%%%%%%%%%%%%%%%%%%%%%%%%%%%%%%%%%%%%%%%%%%

\section{Gyakorlati alkalmazások (kieg.)}

Az alábbiakban bemutatunk néhány, a strukturális modellezés témaköréhez kapcsolódó gyakorlati technológiát, specifikációt és eszközt. Az itt felsorolt fogalmak nem részei a számonkérésnek, gyakran előkerülnek viszont a későbbi tanulmányok és munkák során, ezért mindenképpen érdemes legalább névről ismerni őket.

\subsection{Modellezési nyelvek}

A gyakorlatban sokféle modellezési nyelvet használnak, ezek közül mutatunk be most néhány elterjedtebbet.

\subsubsection{UML}

Az \roviditesangolulkifejtve{UML} egy általános célú modellezési nyelv az~\cite{UML}. Az UML három fő diagramtípust definiál:

\begin{itemize}
	\item \emph{Structure Diagram:} strukturális modellek leírására. A \emph{Class Diagram} az osztályok (metamodell), míg az \emph{Object diagram} a példányok (modell) leírására alkalmas. A \emph{Composite Structure Diagram} egy rendszer struktúráját és a rendszer komponenseinek kapcsolatát mutatja be.
	\item \emph{Behaviour Diagram:} viselkedési modellek leírására, pl. a \emph{State Machine Diagram} segítségével állapotgépek az \emph{Activity Diagramon} folyamatok ábrázolhatók. A \emph{Behaviour Diagramek} között megkülönböztejük az \emph{Interaction Diagrameket}. Ezeknek szintén a viselkedés leírása a célja, de a hangsúly a vezérlés- és adatáramlás bemutatásán van. Ilyen pl. a \emph{Sequence Diagram} (szekvenciadiagram), amely az egyes objektumok közötti interakciót mutatja be üzenetek formájában.
\end{itemize}

\remofigscale{UML-diagram}{UML diagramok típusai és a közöttük lévő viszony osztálydiagramként ábrázolva~\cite{wiki:ISE}}{0.6}

Az UML nyelvvel részletesen foglalkozik a \szofttech tárgy.

\subsubsection{AADL}

Az \roviditesangolulkifejtve{AADL} eredetileg repülőipari célokra fejlesztett architektúraleíró nyelv~\cite{AADL}.

\subsubsection{SysML}

A \roviditesangolulkifejtve{SysML} egy UML-alapú általános modellezési nyelv rendszertervezési célokra~\cite{SysML}. A SysML az UML egy részhalmazát bővíti ki és új diagramtípusokat is bevezet, pl. \emph{Requirement Diagram} követelmények és viszonyaik ábrázolására alkalmas.

\subsubsection{EMF}

Az Eclipse fejlesztőkörnyezet~\cite{eclipse} saját modellezési keretrendszerrel rendelkezik, ez az \roviditesangolulkifejtve{EMF}. Az EMF metamodellező nyelve, az Ecore lehetővé teszi saját, ún. \roviditesteljesenkifejtve{DSL} definiálását. Az EMF mára több területen is \emph{de facto} modellezési keretrendszer, sikere nagyban hozzájárul az Eclipse népszerűségéhez.

Az Eclipse-szel és az EMF-fel foglalkozik az \eat szabadon választható tárgy.

\subsubsection{RDF}

\roviditesangolulkifejtve{RDF}

TODO erről is lesz szakirányos tárgy

\subsubsection{AUTOSAR}

Az \roviditesangolulkifejtve{AUTOSAR}~\cite{autosar} nagy autóipari gyártók és beszállítóik által fejlesztett szabványos architektúranyelv, amely az egyes hardver- és szoftverkomponensek együttműködése magas szinten definiálható. Az AUTOSAR konzorciumnak a \emph{Méréstechnika és Információs Rendszerek Tanszék} is tagja~\cite{autosar-attendees}.

\subsection{Struktúramodellezési technikák}



\subsubsection{Tervezési minták}

Az \fogalom{objektum-orientalt} tervezés során gyakran előforduló problémákra különböző \fogalomragozva{tervezesi-minta}{tervezési minták} (\fogalomragozva{tervezesi-minta}{design patterns}) léteznek. A tervezési minták között külön szerepet kapnak a rendszer struktúráját leíró \fogalomragozva{szerkezeti-minta}{szerkezeti minták} (\fogalomragozva{szerkezeti-minta}{structural patterns}). A tervezési mintákkal bővebben a \sznikak tárgy foglalkozik.

\subsubsection{Refaktoring}

A \fogalomragozva{dekompozicio}{dekompozícióhoz}, azaz \fogalomragozva{faktoring}{faktoringhoz} szorosan kapcsolódik a \fogalom{refaktoring} (\fogalomangolul{refaktoring}) fogalma~\cite{fowler2012refactoring}. Refaktoring során létező rendszereket definiáló programkódok vagy modellek átalakítását értjük. A refaktoring lényege, hogy az átalakítás során a rendszer megfigyelhető működése változatlan marad, de a kapott programkód vagy modell érthetőbb, karbantarthatóbb lesz.

\subsection{Struktúramodellező eszközök és vizualizáció}

Gráfok automatikus megjelenítésére alkalmas pl. a GraphViz\footnote{\url{http://www.graphviz.org/}} programcsomag. Gráfok feldolgozására gyakran alkalmazzák az igraph\footnote{\url{http://igraph.org/}} programcsomagot. Manapság több gráfadatbázis-kezelő rendszer is elterjedt, pl. a Neo4j\footnote{\url{http://neo4j.com/}} és a Titan\footnote{\url{http://thinkaurelius.github.io/titan/}} rendszerek.

Gráfok manuális rajzolására szintén több eszköz elterjedt. Egyszerűen használható online felületet biztosít a draw.io\footnote{\url{http://draw.io/}} és az Arrow Tools\footnote{\url{http://www.apcjones.com/arrows/}}. Összetettebb ábrákhoz a yEd\footnote{\url{https://www.yworks.com/products/yed}} eszközt érdemes használni. Sok információt tartalmazó gráf esetén érdemes lehet vektorgrafikus rajzoló-, ill. prezentáló eszközök, pl. a Microsoft Visio vagy Microsoft PowerPoint alkalmazás.

%%%%%%%%%%%%%%%%%%%%%%%%%%%%%%%%%%%%%%%%%%%%%%%%%%%%%%%%%%%%%%%%%%%%%%%%%%%%%%%%%%%%%%%%%%%%%%%%%%%%

\section{Elméleti kitekintés (kieg.)}

A struktúramodellezésnek komoly matematikai eszköztára is van. Az alábbiakban ezekből mutatunk be néhány részletet.

\subsection{Hipergráfok}

A gráf adatmodell kiterjesztése a \fogalom{hipergraf}. A hipergráfban a \fogalomragozva{hiperel}{hiperélek} több csomópont között is futhatnak. Megkülönböztetünk irányított és irányítatlan hipergráfokat.

Az irányítatlan hipergráfok egyszerűen leképezhetők \fogalomragozva{paros-graf}{páros gráffá}: a gráf egyik halmazában a hipergráf csomópontjainak, a másik halmazában a hiperéleknek veszünk fel egy pontot.

\fogalom{kenyszer} + \rovidites{CSP}

\todo[inline]{ábra}

\subsection{Bináris relációk tulajdonságai}

A \fogalom{tranzitiv-lezaras} fogalom jelentésének megértését segíti, ha ismerjük a kétváltozós relációkon értelmezett tulajdonságokat~\cite{wiki:relacio}.

%http://math.stackexchange.com/questions/65102/if-a-relation-is-symmetric-and-transitive-will-it-be-reflexive

\begin{definicio}
	Egy $S$ halmazon értelmezett $r$ kétváltozós reláció \fogalomragozva{reflexivitas}{reflexív}, ha bármely $a \in S$-re $r(a, a)$ teljesül.
\end{definicio}

\begin{definicio}
	Egy $S$ halmazon értelmezett $r$ kétváltozós reláció \fogalomragozva{szimmetria}{szimmetrikus}, ha bármely $a,b \in S$-re $r(a, b)$ teljesülése esetén $r(b, a)$ is teljesül. A nem szimmetrikus relációkat \fogalomragozva{aszimmetria}{aszimmetrikusnak} nevezzük.
\end{definicio}

\begin{definicio}
	Egy $S$ halmazon értelmezett $r$ kétváltozós reláció \fogalomragozva{tranzitivitas}{tranzititív}, ha bármely $a,b,c \in S$-re $r(a, b)$ és $r(b, c)$ teljesülése esetén $r(a, c)$ is teljesül.
\end{definicio}

\begin{pelda}
	\remofigscaleframed{tranzitivitas}{Az \emph{egyenlő} ($=$) és a \emph{kisebb} ($<$) relációk gráfon ábrázolva}{\yedscale}

	Tranzitív relációk:
	\begin{itemize}
	\item Az \emph{egyenlő} ($=$) és a \emph{kisebb} ($<$) relációk tranzitívak, mert
	\begin{itemize}
		\item $a = b$ és $b = c$ esetén $a = c$,
		\item $d < e$ és $e < f$ esetén $e < f$.
	\end{itemize}
	\Az+\ref{fig:tranzitivitas}. ábrán ábrázoltuk a fenti relációkat. Az $=$ reláció szimmetrikus, ezért irányítatlan gráffal, a $<$ reláció aszimmetrikus, ezért irányított gráffal reprezentálható.
	\end{itemize}
	
	Nem tranzitív relációk:
	\begin{itemize}
	\item A \emph{nemegyenlő} reláció ($\neq$) nem tranzitív, mert 
	\begin{itemize}
		\item $a \neq b$ és $b \neq c$ esetén $a \neq c$ nem mindig áll fenn, például
		\item $1 \neq 2$ és $2 \neq 1$ esetén $1 \neq 1$ nem teljesül.
	\end{itemize}
	\item Személyek közötti $\mathit{őse}$ reláció tranzitív, mert $\mathit{őse}(a, b)$ és $\mathit{őse}(b, c)$ esetén $\mathit{őse}(a, c)$ is fennáll. 
	\item Személyek közötti $\mathit{ismerőse}$ reláció nem tranzitív, mert $\mathit{ismerőse}(a, b)$ és $\mathit{ismerőse}(b, c)$ esetén nem garantált, hogy $\mathit{ismerőse}(a, c)$ fennáll.
	\end{itemize}
\end{pelda}


A Rendszermodellezés tárgynak nem témája, de érdemes megismerni a relációk további tulajdonságait: \fogalom{reflexivitas}, \fogalom{szimmetria} stb.~\cite{wiki:relacio}. 

\subsection{Ontológia}

Az ontológia kifejezésre sok különböző definíció létezik. Az informatikában általában az alábbihoz hasonló módon definiálják:

\begin{definicio}
Az \fogalom{ontologia} egy olyan \fogalom{taxonomia}, amely tartalmazza a benne szereplő fogalmak közötti viszonyokat. 
\end{definicio}

\todo{RDF}

%%%%%%%%%%%%%%%%%%%%%%%%%%%%%%%%%%%%%%%%%%%%%%%%%%%%%%%%%%%%%%%%%%%%%%%%%%%%%%%%%%%%%%%%%%%%%%%%%%%%

\section{Ajánlott irodalom}

A gráfelmélettel behatóan foglalkozik a \bszketto tantárgy és Fleiner Tamás jegyzete~\cite{FleinerJegyzet}. Különböző gráfalgoritmusokkal -- pl. \fogalom{legrovidebb-ut} és minimális összsúlyú \fogalom{feszitofa} keresésére -- az \algel tárgy foglalkozik. További keresőalgoritmusok a \mestersegesintelligencia tárgyban szerepelnek.

Olvasmányos, jól érthető összefoglalót ad az UML nyelvről Martin Fowler ,,UML Distilled'' című könyve~\cite{fowler1997uml}.

A tulajdonsággráfokról egy jól érthető tudományos cikk Marko Rodriguez és Peter Neubauer munkája~\cite{Rodriguez2010}. Rodriguez a Titan elosztott gráfadatbázis-kezelő rendszer egyik fő fejlesztője~\cite{Titan}, míg Neubauer a Neo4j gráfadatbázis-kezelőt fejlesztő cég alapítója~\cite{Neo4j}. Az elosztott gráfadatbázis alkalmazását, elméleti és gyakorlati kihívásait kiváló prezentációkban mutatja be~\cite{RodriguezSlides2012,RodriguezSlides2013}.

Barabási-Albert László magyar fizikus nemzetközileg elismert kutatója a komplex \fogalomragozva{halozat}{hálózatok} elméletének. Barabási ,,Behálózva'' című könyve közérthető stílusban mutatja be a hálózatok elemzésének elméleti kihívásait a kutatási eredmények gyakorlati jelentőségét~\cite{behalozva}. A szerzővel több interjú is készült~\cite{barabasi1,barabasi2,barabasi3}.

%Az osztályok és prototípusok közötti elvi különbséget mutatja be Antero Taivalsaari, a Nokia Research fejlesztőjének 1996-os cikke~\cite{taivalsaari1996classes}.

%\roviditesangolulkifejtve{SQL}
%\fogalom{azonosito} (\fogalomragozva{elsodleges-kulcs}{elsődleges kulcsok})
