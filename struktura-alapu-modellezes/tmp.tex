\subsection{Struktúramodellek}
\section{Gráfok}
\label{sec:graf}

Az alábbiakban összefoglaljuk a gráfokkal kapcsolatos fogalmakat és definíciókat. Nem mindenhol célunk a teljes matematikai precizitás, ill. az egyes fogalmakhoz tartozó különböző definíciók felsorolása (pl. a \fogalom{graf} fogalomra több különböző, némileg eltérő jelentésű definíció létezik).

Bár a gráf meglehetősen egyszerű fogalom, rendkívül sok szakterület modelljei jól ábrázolhatók benne. Ennek egyik fő oka, hogy az emberek gondolkodás közben a világot gyakran úgy írják le, mint ,,dolgok'' (csomópontok), amelyek egymással valamilyen ,,viszonyban'', ,,kapcsolatban'' vannak (élek).

\begin{pelda}
Irányítatlan gráffal ábrázolhatjuk emberek között az \emph{ismerőse} relációt, ilyen pl. a Facebook \emph{friendship} relációja.
\end{pelda}

\begin{pelda}
Irányított gráffal ábrázolhatjuk emberek között a \emph{követi} relációt, ilyen pl. a Twitter \emph{follows} relációja. Szintén irányított gráffal ábrázolhatók az interneten egymásra mutató weboldalak.
\end{pelda}

Természetesen irányítatlan gráfok ábrázolhatók irányítottként is, ha minden $\{v_1, v_2\}$ irányítatlan élet lecserélünk. (Érdekesség: megfelelő transzformációval irányított gráfok is ábrázolhatók irányítatlan gráfként~\cite{rodriguez2008mapping}.)
\begin{kisdefinicio}
Egy $G = (V, E)$ gráfnak $G' = (V', E')$ \fogalomragozva{reszgraf}{részgráfja}, ha $G'$ megkapható úgy, hogy $G$-ből elhagyunk néhány csomópontot és azok éleit, ill. néhány további élet. Formálisan: ha $V' \subseteq V$ és $E' \subseteq V' \times V'$.
\end{kisdefinicio}


\subsection{Műveletek gráfokon}

\paragraph{Elérhetőségvizsgálat}

Egy $v_i$ csomópontból egy $v_j$ csomópont elérhető, ha létezik $v_i$-ből $v_j$-be irányított út.

\paragraph{Útkereső algoritmusok}

%\fogalom{Dijsktra-algoritmus}, \fogalom{Bellman-Ford-algoritmus}, \fogalom{Floyd-algoritmus}, \fogalom{A-csillag-kereses}
%
%A legrövidebb út keresése kipróbálható az alábbi példán: \url{http://gist.neo4j.org/?6d2a1a1c6325043d09a0}.

\paragraph{Transzformációk}

\begin{definicio}
\Fogalom{szures-eltipusra}: gyakran csak bizonyos éltípus(ok)ra van szükségünk. Egy $G = (V, E)$ gráf adott $c_1, \dots, c_n$ élcímkékre szűrt részgráfja $G' = (V, E)$, ahol $V$ a csomópontok változatlan halmaza, míg $E'$ azon $e$ élek halmaza, ahol $e$ címkéje eleme a $c_1, \dots, c_n$ halmaznak.
\end{definicio}

\fogalom{graftranszformacio}
