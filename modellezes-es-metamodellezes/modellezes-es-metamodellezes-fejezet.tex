\topic{Modellezés és metamodellezés}

Ebben a fejezetben a modellezés alapfogalmaival fogunk megismerkedni. Az itt bevezetett fogalmak újra és újra megjelennek majd a későbbi fejezetekben, ahol részletesen ki fogunk térni az adott területen történő értelmezésükre.

%Az alábbi dokumentum a rendszermodellezés tárgy első előadásához kapcsolódó segédanyag, mely tartalmazza az elhangzott definíciókat és a hozzájuk tartozó egyszerűbb magyarázatokat.

\section{Modellezés}

Mi értelme van modellezni? Fejben szinte mindig modellezünk, bármilyen probléma kerül elénk. Nincs ez másképp egy szoftver fejlesztésekor sem. Lássuk hát, miről beszélünk, amikor a modell szót használjuk.
% Legfőbb szerepe a kommunikáció:

\begin{definicio}
	\Fogalom{modell}: egy valós vagy hipotetikus világ (a ,,rendszer'') egy részének egyszerűsített képe, amely a rendszert helyettesíti bizonyos megfontolásokban.
\end{definicio}

A modell tehát egy többnyire bonyolult dolog egyszerűsített reprezentációja, amiben csak a számunkra lényeges vonásokat szerepeltetjük.
Egy adott probléma modellé történő leképezésének általában két fontos előnye van: %\emph{áttekinthetőbb}, \emph{kisebb (véges)}.

\begin{itemize}
	\item A modell a problémánál \emph{kisebb}, hiszen a problémához nem (vagy lazábban) kapcsolódó, elhanyagolható információk nem jelennek meg benne.
	\item A modell a problémánál \emph{áttekinthetőbb}, hiszen csak az adott megfontolásban érdekes, releváns információkat és kapcsolatokat kell vizsgálni.
\end{itemize}

\begin{pelda}
	Modellekre sok példát láthatunk a hétköznapokban is. Nem csak gyerekek körében népszerű játék a modellvasút. Itt valóban modellről beszélhetünk, hiszen a játékvonatok számos tekintetben hűen reprezentálják a valódi vonatokat, azonban például ``szemet hunyunk'' a méretükkel, tömegükkel, a bennük lévő villanymotor paramétereivel, és még sok egyéb hasonló tulajdonságukkal kapcsolatban.
	
	Az informatika egyik leggazdagabb modellforrása a matematika. Amikor gráfelméletet tanulunk, valójában rengeteg, bizonyos szempontból hasonló probléma közös modelljét vizsgáljuk. Valóban, a matematika egyik célja az ilyen modellek azonosítása, és minél hatékonyabb eszköztárak kidolgozása a rajtuk megfogalmazott problémák megoldására. A gráfoknál maradva (gráfokról bővebben a \ref{sec:graf} fejezetben és a \emph{Bevezetés a számításelméletbe} című tárgyban lehet tájékozódni) egy város úthálózata jól reprezentálható egy élsúlyokkal ellátott gráffal, ahol a csomópontok a kereszteződések, az élek az útszakaszok, az élsúlyok pedig a szakaszok hosszait jelölik. Ez a modell kiválóan alkalmas legrövidebb útvonalak tervezésére (mi kellene még a \emph{leggyorsabb} útvonal tervezéséhez?), de figyelmen kívül hagy számos egyéb paramétert, például az utakon lévő kanyarulatokat, a legnagyobb megengedett sebességet, stb.
\end{pelda}

\begin{megjegyzes}
	Az, hogy a modell kisebb, nem mindig ``kényelmi'' szempont. Gyakran lehet olyan problémával találkozni, aminek a mérete bizonyos szempontból végtelen (például végtelen sok állapota van, vagy folytonos változók vannak benne), viszont egy alkalmas \emph{véges} modellen a számunkra releváns tulajdonságok továbbra is jól vizsgálhatók maradnak. Egy autót fizikai szempontból modellezhet a pillanatnyi sebességvektora, ami három valós szám, tehát a modellünk így végtelen. Ha viszont feltételezzük, hogy a sebesség nagysága nem lehet $200$ $km/h$-nál több, és két tizedesjegy pontossággal adjuk meg a vektor koordinátáit, akkor máris véges modellt kapunk. Természetesen ez a modell kicsit torzítani fog a valósághoz képest, de számos problémánál ez a hiba elhanyagolhatóan kicsi lesz (ráadásul végtelen modellen lehet, hogy nem is tudnánk megoldást adni).
\end{megjegyzes}

Fontos kérdés, hogy hogyan lehet ábrázolni egy modellt. Maga a modell ugyanis többnyire egy hipotetikus struktúra, nem kézzel fogható, és nem is mindig célszerű teljes részletességgel ábrázolni.

\begin{definicio}
	\Fogalom{diagram}: A modell egy nézete, amely a modell bizonyos aspektusait grafikusan ábrázolja.
\end{definicio}

\begin{megjegyzes}
Fontos megjegyezni, hogy nem minden modell, ami modellnek látszik.

\begin{itemize}
	\item \emph{A modell nem a valóság}: az általunk definiált modellen bizonyos állítások igazak lehetnek, amik a valóságban nem állják meg a helyüket.
	\item \emph{A modell nem a diagram}: A diagram csak egy ábrázolás módja a modellnek, amivel olvashatóvá tesszük. 
\end{itemize}
\end{megjegyzes}

\section{Modellezési nyelvek}

A modellek leírásához nem elég, ha diagramokat rajzolunk. Egy modell precíz megadásához szükség van egy \fogalomragozva{modellezesi nyelv}{modellezési nyelvre}. Egy modellezési nyelv lehet \emph{szöveges} (pl. Verilog, VHDL, Java, stb.) vagy \emph{grafikus} (pl. webes konyhatervező, UML diagramok, stb.). Gyakori, hogy egy modellezési nyelv szöveges és grafikus jelölésrendszert is definiál, ugyanis mindkettőnek megvannak a saját előnyei és hátrányai: Jellemzően modellt építeni könnyebb szöveges nyelv használatával, míg a modell olvasása grafikus nyelvek (diagramok) segítségével egyszerűbb.

A modellezési nyelvek legfőbb célja a kommunikáció gép-gép, ember-gép, vagy akár ember-ember között is. Egy nyelv megadásához két dolgot kell definiálni: a \emph{szintaxist} és a \emph{szemantikát}.

%Segítségükkel modelleket építhetünk, amik lehetnek szöveges (pl.: Verilog, VHDL, Java stb.), vagy grafikusak is (pl.: webes konyhatervező, UML diagrammok, UPPAAL stb.). Céljuk a kommunikáció gép-gép, ember-gép vagy akár ember-ember között is. A nyelveknek két fő aspektusa van:

\begin{definicio}
	\Fogalom{szintaxis}: Egy szabályrendszer, ami meghatározza, hogy a modell milyen elemekből épül fel, ezeknek mi a (szöveges vagy grafikus) reprezentációja, és a különböző elemek milyen módon kombinálhatók.
\end{definicio}

\begin{definicio}
	\Fogalom{szemantika}: A szintaxis által megadott nyelvi elemek jelentését definiáló szabályrendszer.
\end{definicio}

%\begin{itemize}
%	\item \fogalom{szintaxis}: hogyan írom a le a modellt? Egy szabályrendszer, ami meghatározza, hogy a nyelv milyen elemekből épül fel és azok hogyan kombinálhatóak.
%	\item \fogalom{szemantika}: mit jelent a modell? A szintaxis által definált nyelvi elemek jelentését definiáló szabályrendszer.
%\end{itemize}

A szintaxis tehát definiálja, hogy hogyan írhatjuk le a modellt az adott nyelv segítségével, a szemantika pedig megadja, hogy az így leírt modell pontosan mit jelent.

\begin{pelda}
	A C nyelv egy szöveges nyelv, melyben elemeknek tekinthetjük az \code{if}, \code{for}, \code{switch}, \code{változónevek}, \code{metódusnevek}, \code{típusok} stb. részeket. Ezek azonban nem tetszőleges sorrendben állhatnak egymás mellett, hanem egy jól meghatározott szabályrendszer szerint. Ez a szintaxis. Azt, hogy az \code{if} kulcsszóval elágazást, a \code{for} kulcsszóval pedig egy ciklust definiálunk és nem pedig valami teljesen más dolgot, a szemantika határozza meg. Vagyis a szemantika jelentéssel tölti meg a nyelvi elemeket.

	A Yakindu eszköz bizonyos szinten egy grafikus nyelvet határoz meg, ahol például nyelvi elemek a dobozok és nyilak (az állapottérképek Yakindu által definiált modellezési nyelvét a \ref{sec:allapot-alapu-modellezes} fejezet mutatja be). A szintaxis meghatározza, hogy a dobozokból csak nyilakkal lehet összekötni más dobozokat. A szemantikus jelentésük pedig az, hogy a dobozok állapotokat, a nyilak pedig állapotátmeneteket jelölnek (tehát megmutatják, hogy az alkalmazásunk melyik állapotból melyik állapotba kerülhet).
\end{pelda}

\begin{megjegyzes}
	Szintaktikailag leírhatunk olyan kifejezéseket a nyelvben, aminek szemantikailag nincs értelme.	
\end{megjegyzes}


\section{Metamodellezés}

\begin{definicio}
	A \fogalom{metamodell} egy modellezési nyelv modellje. Meghatározza, hogy a nyelvnek milyen típusú elemei vannak, milyen kapcsolatban állhatnak egymással, és a típusoknak mi a viszonya egymáshoz.
\end{definicio}

Tipikus, a későbbi tanulmányokban előkerülő illusztrációk:

\begin{itemize}
	\item Egyed-kapcsolat, EK modell (entity-relationship, ER)
	\item UML~\cite{UML} objektum diagram (modell) $\leftrightarrow$ osztálydiagram (metamodell)
	\item Adatbázis tábla (modell) $\leftrightarrow$ relációs adatbázisséma
	\item XML dokumentum (modell) $\leftrightarrow$ XML séma (metamodell)
\end{itemize}

\section{Alapfogalmak}

\begin{definicio}
	Rendszer és külvilág: a külvilág a rendszerre ható tényezők összessége. A rendszer pedig a komponensei és a közöttük lévő kapcsolatok. (Fehér doboz: ismerjük a rendszer belső működését; fekete doboz: nem ismerjük a belső működés, csak a kimenetei alapján vizsgálhatjuk a működését.)
\end{definicio}

\todo{kulvilag micsoda? environment}

\begin{definicio}
	\Fogalom{finomitas}: a modell gazdagítása részletekkel úgy, hogy az eredeti modell absztrakció megmaradjon. (Nem irreleváns adatokkal bővítjük, hanem pontosítjuk a modellünket)
\end{definicio}

\begin{definicio}
	\Fogalom{absztrakcio}: a finomítás inverz művelete. (Nem a releváns dolgokról találjuk ki, hogy most már nem relevánsak, hanem a releváns részleteket egyszerűsítjük)
\end{definicio}

