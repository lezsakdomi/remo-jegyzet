% !TeX spellcheck = hu_HU
\topic{Benchmarking}\label{cha:benchmarking}

\graphicspath{ {./benchmarking/figures/} }

Melyik a legjobb ár-érték arányú processzort egy PC-be? Melyik a leggyorsabb relációs adatbázis-kezelő? Melyik kollekciókat kezelő függvénykönyvtár használja a legkevesebb memóriát? Gyakran szükséges, hogy ezekhez hasonló kérdésekre megbízható választ adjunk. A kérdések megválaszolásában általában komoly segítséget nyújt valamilyen metrika vizsgálata. Például számítsuk ki a másodpercenként elvégzett lebegőpontos utasítások számának és az árnak a hányadosát, mérjünk le adott mennyiségű lekérdezéshez szükséges időt egy előre definiált adathalmazon, vagy hozzunk létre egy megadott kollekciót különböző függvénykönyvtárakkal és vizsgáljuk meg a memóriafogyasztásukat. Megfelelő mérések elvégzésével jó képet kaphatunk egy rendszer adott jellemzőiről.

\begin{megjegyzes}
	Benchmarkokat abban az esetben érdemes alkalmazni, ha egy rendszer objektív jellemzőire vagyunk kíváncsiak. Szubjektív jellemzők esetén (pl. mennyire használható egy programozási nyelv, mennyire alkalmas egy adatmodell bizonyos feladatokra) a kérdés megválaszolása jóval összetettebb lehet. Ezekben az esetekben általában \fogalomragozva{felhasznaloi-tanulmany}{felhasználói tanulmányok} (\fogalomangolul{felhasznaloi-tanulmany}) végzése célravezető.
	
	Az informatika a klasszikus mérnöki tudományokkal -- pl. építészet, gépészet, vegyészet -- szemben még fiatal területnek számít. Ennek egyik fő jele, hogy gyakran szubjektív kérdésekről (melyik a legjobb programozási nyelv, melyik a legjobb operációs rendszer vagy melyik a legjobb adatmodell) is véget nem érő ,,vallási viták'' zajlanak.\footnote{Érdekes olvasmány a témában Luis Solano ,,Why Does Programming Suck?'' c. cikke: \url{https://medium.com/@luisobo/why-does-programming-suck-6b253ebfc607}} Ezeket érdemes elkerülni.
	
	Az elmúlt évtizedek intenzív kutatómunkája ellenére az informatikai projektekben továbbra is kiemelkedően magas a sikertelen, elvetett (cancelled) vagy költségtervet túllépő (over budget) projektek száma~\cite{HBR:ITProjects}. Az informatikai projektek menedzsment háttere iránt érdeklődőknek javasoljuk a mára ,,klasszikusnak'' számító műveket, mint a \emph{Peopleware}~\cite{demarco2013peopleware} és a \emph{The Mythical Man-Month}~\cite{brooks1995mythical}.
	
%	No wonder in this field there are holy wars about the most insignificant things
\end{megjegyzes}



\section{Alapfogalmak}


Motiváció

- Szoftver/hardver eszközök teljesítményének összehasonlítása

- Döntéstámogatás 

o Melyiket előnyösebb megvenni/telepíteni?

o Mekkora terhelésre elég a meglévő rendszer?

- Teljesítménytesztelés 

o Kell‐e még a teljesítményen javítani és hol 
(fejlesztésnél)?

o Optimális‐e egy konkrét beállítás?

o Van‐e egy beállításnak teljesítményre gyakorolt hatása 
(érzékenység vizsgálat)?

Fejlesztés alatt lévő rendszer

- Teljesítmény felmérése

o Modell alapján becsült értékek

o Megvalósított rendszer értékei „éles” helyzetben

- Tervezői és menedzsment döntések

o Melyik részébe fektessünk több energiát (≈ pénzt)? 

- Mi alapján döntsünk?

o Jelenlegi erősségek és gyengeségek

- Mi számít erősségnek vagy gyengeségnek?

o Mire képesek a hasonló rendszerek, versenytársak?

Elkészített rendszer

- Versenyképesség felmérése

o Versenytársak eredményei

- Marketing stratégiák

o Miért minket válasszanak?

- Milyen adatokra hivatkozzunk?

o A miénk a legjobb rendszer egy esettanulmány szerint

• És a többi esetben?

o Ki rangsorolta az elérhető rendszereket?

o Mi alapján?

\begin{definicio}
	A \fogalom{benchmarkolas}
	\begin{itemize}
		\item egy \emph{program} (programok, vagy más műveletek) \emph{futtatása},
		\item \emph{szabványos tesztekkel} vagy bemenetekkel,
		\item egy objektum \emph{relatív teljesítményének} felmérése érdekében.
	\end{itemize}
\end{definicio}

A Wikipédia definíciója~\cite{wiki:benchmark} (kiemelések a jegyzet szerzőitől):

\begin{quote}
	In \textbf{computing}, a benchmark is the \textbf{act of running} a computer program, a set of programs, or other operations, in order to \textbf{assess the relative performance} of an object, normally by running a number of \textbf{standard tests} and trials against it.
\end{quote}

A benchmarkokkal szemben többféle elvárást támasztunk:

\begin{definicio}
	\Fogalom{ismetelhetoseg} (\fogalomangolul{ismetelhetoseg}): egy elemen többször megismételt mérés/művelet eredményeinek változékonysága (szóródási faktor).
\end{definicio}

\begin{definicio}
	\Fogalom{reprodukalhatosag} (\fogalomangolul{reprodukalhatosag}): a mérési rendszer változékonysága, amelyet a műveletek viselkedéseinek különbsége okoz.
\end{definicio}

\begin{definicio}
	\Fogalom{erthetoseg} (\fogalomangolul{erthetoseg}): átlag felhasználó számára értelmezhető legyen az eredmény
\end{definicio}

Szabványok/megállapodások betartása

\fogalom{relevancia} (\fogalomangolul{relevancia})

o Tényleg azt az alkalmazást mérjük, amit kell

o Terhelésgenerálás jellege közelítse a \fogalom{valodi} terhelést

o Minimalizáljuk a zavaró tényezőket

• Memória cache tartalma

• Betöltött/kilapozott memória területek

• Diszk cache tartalma

A \emph{Benchmarking Handbook}~\cite{DBLP:books/mk/Gray93}.



\roviditesangolulkifejtve{OLTP}

\begin{megjegyzes}
Az OLTP rendszerekkel szemben az \roviditesangolulkifejtve{OLAP} rendszerekben (általában) nagyobb mennyiségű adaton összetettebb, analitikus jellegű lekérdezéseket futtatnak. \rovidites{OLAP} rendszerek teljesítménymérésére tervezték a \mbox{TPC-DS} benchmarkot.
\end{megjegyzes}

\section{Példa benchmarkok\kieg}

\roviditesangolulkifejtve{SPEC}\footnote{\url{https://www.spec.org/benchmarks.html}}

\subsection{Hardver benchmarkok}

PassMark\footnote{\url{https://www.passmark.com/}}

PassMark CPU benchmark\footnote{\url{https://www.cpubenchmark.net/}}

Böngészőmotorok: Octane\footnote{\url{https://developers.google.com/octane/}},
Kraken\footnote{\url{http://krakenbenchmark.mozilla.org/}}.

Okostelefonok: Antutu\footnote{\url{http://www.antutu.com/en/index.shtml}}


\subsection{Szoftver benchmarkok}

Relációs adatbázis-kezelő rendszerek teljesítményének méréséhez a \roviditesangolulkifejtve{TPC} szervezet definiált különféle benchmarkokat. Például a \rovidites{TPCC}\footnote{\url{http://www.tpc.org/tpcc/}} célja tranzakciós  rendszerek teljesítménymérése.

\subsection{Gráftranszformáció benchmarkok}

A kutatócsoportunkban készült el az egyik első gráftranszformációk teljesítményét vizsgáló benchmark~\cite{vlhcc05_vsv}. A területen azóta is aktív kutatási munkát végzünk~\cite{DBLP:conf/staf/SzarnyasSRV15}.

