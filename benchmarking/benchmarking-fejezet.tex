% !TeX spellcheck = hu_HU
\topic{Benchmarking}\label{cha:benchmarking}

\graphicspath{ {./benchmarking/figures/} }

%\todo[inline]{ez a fejezet elég jó, leginkább ábra kellene bele}

\section{Alapfogalmak}

\begin{megjegyzes}
	Az informatika a klasszikus mérnöki tudományokkal -- pl. építészet, gépészet, vegyészet -- szemben még fiatal területnek számít, kiforratlan alapelvekkel. Ennek egyik jele, hogy gyakran szubjektív kérdésekről (melyik a legjobb programozási nyelv, a legjobb operációs rendszer vagy a legjobb adatmodell) is véget nem érő ,,vallási viták'' zajlanak. Ezeket érdemes távolról elkerülni.
	
	Az elmúlt több mint 50 év intenzív kutatómunkája ellenére az informatikai projektekben továbbra is kiemelkedően magas a sikertelen, elvetett (cancelled) vagy költségtervet túllépő (over budget) projektek száma.
	
	A műszaki projektek iránt érdeklődőknek javasolt ,,klasszikus''-nak számító műveket, mint a \emph{Peopleware}~\cite{demarco2013peopleware} és a \emph{Mythical Man Month}~\cite{brooks1995mythical}.
	
%	No wonder in this field there are holy wars about the most insignificant things
\end{megjegyzes}

Motiváció

- Szoftver/hardver eszközök teljesítményének összehasonlítása

- Döntéstámogatás 

o Melyiket előnyösebb megvenni/telepíteni?

o Mekkora terhelésre elég a meglévő rendszer?

- Teljesítménytesztelés 

o Kell‐e még a teljesítményen javítani és hol 
(fejlesztésnél)?

o Optimális‐e egy konkrét beállítás?

o Van‐e egy beállításnak teljesítményre gyakorolt hatása 
(érzékenység vizsgálat)?

Fejlesztés alatt lévő rendszer

- Teljesítmény felmérése

o Modell alapján becsült értékek

o Megvalósított rendszer értékei „éles” helyzetben

- Tervezői és menedzsment döntések

o Melyik részébe fektessünk több energiát (≈ pénzt)? 

- Mi alapján döntsünk?

o Jelenlegi erősségek és gyengeségek

- Mi számít erősségnek vagy gyengeségnek?

o Mire képesek a hasonló rendszerek, versenytársak?

Elkészített rendszer

- Versenyképesség felmérése

o Versenytársak eredményei

- Marketing stratégiák

o Miért minket válasszanak?

- Milyen adatokra hivatkozzunk?

o A miénk a legjobb rendszer egy esettanulmány szerint

• És a többi esetben?

o Ki rangsorolta az elérhető rendszereket?

o Mi alapján?

\begin{definicio}
	A \fogalom{benchmarkolas}
	\begin{itemize}
		\item egy \emph{program} (programok, vagy más műveletek) \emph{futtatása},
		\item \emph{szabványos tesztekkel} vagy bemenetekkel,
		\item egy objektum \emph{relatív teljesítményének} felmérése érdekében.
	\end{itemize}
\end{definicio}

A Wikipédia definíciója~\cite{wiki:benchmark} (kiemelések a jegyzet szerzőitől):

\begin{quote}
	In \textbf{computing}, a benchmark is the \textbf{act of running} a computer program, a set of programs, or other operations, in order to \textbf{assess the relative performance} of an object, normally by running a number of \textbf{standard tests} and trials against it.
\end{quote}

Elvárások

\begin{itemize}
	\item \fogalom{ismetelhetoseg} (\fogalomangolul{ismetelhetoseg})
	o Egy elemen többször megismételt mérés/művelet eredményeinek változékonysága (szóródási faktor)
	\item \fogalom{reprodukalhatosag} (\fogalomangolul{reprodukalhatosag})
	o A mérési rendszer változékonysága, amelyet a műveletek viselkedéseinek különbsége okoz
\end{itemize}

Általánosított felhasználói eset

o Átlag felhasználó számára értelmezhető legyen az
eredmény
Szabványok/megállapodások betartása
\fogalom{relevancia} (\fogalomangolul{relevancia})
o Tényleg azt az alkalmazást mérjük, amit kell
o Terhelésgenerálás jellege közelítse a \fogalom{valodi} terhelést
o Minimalizáljuk a zavaró tényezőket
• Memória cache tartalma
• Betöltött/kilapozott memória területek
• Diszk cache tartalma

~\cite{DBLP:books/mk/Gray93}


\section{Benchmarkok\kieg}

PassMark

\rovidites{SPEC} (\roviditesangolulkifejtve{SPEC})

\rovidites{TPC} (\roviditesangolulkifejtve{TPC})

\rovidites{TPCC}
