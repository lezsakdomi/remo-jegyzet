\topic{Felderítő adatelemzés}

\graphicspath{ {./felderito-adatelemzes/figures/} }

Felderítő adatelemzés \ldots~\cite{tukey77}.

\section{Valószínűségszámítási alapfogalmak}

\begin{definicio}
\begin{itemize}
\item \Fogalom{valoszinusegi-valtozo}: $X$
\item \Fogalom{varhato-ertek}, \fogalom{atlag}:
$$\mu = \mathbb{E}X = \sum_{i=1}^{n} p_i x_i$$

\item \Fogalom{szorasnegyzet}:

$$\sigma^2 = \mathbb{E}\left(X-\mu\right)^2 = \sum_{i=1}^{n} p_i (x_i - \mu)^2$$

\item \Fogalom{szoras}:

$$\sigma = \sqrt{\mathbb{E}\left(X-\mu\right)^2} = \sqrt{\sum_{i=1}^{n} p_i (x_i - \mu)^2}$$
\end{itemize}
\end{definicio}

\section{Statisztikai alapfogalmak}

\begin{definicio}
	\begin{itemize}
		\item \fogalomragozva{megfigyeles}{Megfigyelések}: $t$ darab, $x_1, \dots, x_t$
		\item \Fogalom{tapasztalati-atlag}:

		$$m = \bar{x} = \frac{x_1 + \dots + x_t}{t}$$

		\item \Fogalom{korrigalt-tapasztalati-szoras}:

		$$s = \sqrt\frac{\left(x_1-m\right)^2 + \dots + \left(x_t-m\right)^2}{t-1} = \sqrt\frac{\sum_{i=1}^{t}\left(x_i - m\right)^2}{t-1}$$

		Figyeljük meg, hogy a korrigált tapasztalati értékeknél $t$ helyett $(t-1)$-gyel osztunk. Ennek oka, hogy $t$-vel osztva a kapott érték általában alábecsli a teljes populáció szórását. Belátható, hogy $(t-1)$-gyel osztva a valódi szórást jobban közelítő értéket kapunk. Ezt nevezzük Bessel-féle korrekciónak (\url{https://en.wikipedia.org/wiki/Bessel's\_correction}).
		\end{itemize}
	\end{definicio}

\section{Kísérlettervezés}

A centrális határeloszlás tételéből (CHT) következőik, hogy tetszőleges eloszlású jellemző (véges $m$ várható értékkel és $s$ szórással) tapasztalati átlaga $t \rightarrow \infty$ esetén normális eloszlású, $\mu = m$ várható értékkel és $\sigma = \frac{s}{\sqrt{t}}$ szórással. 

Ökölszabály: ismert szórásnál $t > 30$, ismeretlen szórásnál $t > 100$ után kezd elfogadható lenni a közelítés.

A normális eloszlású változó 
\begin{itemize}
	\item az esetek 68\%-ában legfeljebb $1\sigma$ messze kerül $\mu$-től,
	\item az esetek 95\%-ában legfeljebb $2\sigma$ messze kerül $\mu$-től,
	\item az esetek 99,7\%-ában legfeljebb $3\sigma$ messze kerül $\mu$-től.
\end{itemize}

\remofigscale{felderito-adatelemzes/gaussian-distribution}{Konfidenciaintervallumok}{0.4}
