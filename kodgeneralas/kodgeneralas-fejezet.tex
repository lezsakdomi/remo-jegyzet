% !TeX spellcheck = hu_HU
\topic{Kódgenerálás}

A \fogalom{kodgeneralas} célja \fogalomragozva{modell}{modellből} futtatható \fogalom{forraskod} generálása. ;-)

\section{Fogalmak}


\begin{definicio}
	Az \fogalom{absztrakt-szintaxis} ...
\end{definicio}

\begin{definicio}
	A \fogalom{konkret-szintaxis} ...
\end{definicio}



\begin{definicio}
	A \fogalom{nyelvtan} (\fogalomangolul{nyelvtan})...
\end{definicio}

\fogalomragozva{regularis-kifejezes}{reguláris kifejezések}

\fogalom{fordito}

\fogalom{nezeti-modell}



\fogalom{sablon}

A reguláris kifejezésekkel és a nyelvtanok elméletével bővebben az \algel BSc és a \nyau MSc tárgyak foglalkoznak.

\section{Eszközök\kieg}

ANTLR eszköz~\cite{Parr:2013:DAR:2501720}

Xtext~\cite{Xtext}

\section{Összefoglalás}

\fogalom{produktivitas}++
