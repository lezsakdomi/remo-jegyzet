% !TeX spellcheck = hu_HU
\topic{Modellező eszközök és kódgenerálási módszerek}
\graphicspath{ {./kodgeneralas/figures/} }

A fejezet célja, hogy megismertessük a korszerű modellező eszközökkel és kódgenerálási technikákkal kapcsolatos fogalmakat, alapvető felépítésésüket és működési elvüket.
A fejezet gyakorlatias megközelítésben tárgyalja a témát, minded pontban bemutatjuk, hogyan és mekkora munkával készíthetünk saját, vagy egészíthetünk ki meglévő modellező eszközt.

\section{Modellező eszközök felépítése}
Egy \emph{modelező eszközök} és a felhasználó \emph{tervező mérnök} közötti interakciókat az \aref{fig:modellezoeszkoz-felepites}.~ábra szemlélteti. A modellező eszköz alapvető feladata, hogy karbantartson egy \emph{modellt}, és különböző \emph{olvasási (bejárási, vagy lekérdezési)} és \emph{módosítási} műveletekkel láthatóvá és szerkeszthetővé tegye azt a \emph{tervezőmérnöknek}. Ezért a tervezőeszköz különböző \emph{nézeteket} \emph{származtat} a modellekből, amit a mérnök \emph{megjeleníthet} és meghatározott \emph{szerkesztési műveletekkel} változtathat. Egy modellhez több nézet is tartozhat, ami különböző részleteit emeli ki a modellnek. A modellező eszköz feladata, hogy ezeket a módosításokat visszavezesse a modellbe.

\remofigscalefixed{modellezoeszkoz-felepites}{Modellező eszközökök alapvető felépítése}{0.45}

\begin{pelda}
Vegyünk egy a Yakindu tervező eszközből: 
\end{pelda}

\begin{pelda}TODO
A séma elég álltalános ahhoz hogy egy számítógépes játékot is bemutassunk rajta: a modell célja, hogy tárolja a játékban szereplő objektumokat. A mögöttes modellt a játékprogram kezeli, menti és tölti be, és meghatározza azokat a műveleteket aminek mentén változhat. A játékos a játék során ezt a modellt szerkeszti. Hogy a játékos számára is játszható legyen a világ, a program különböző nézeteket készít
\end{pelda}

\begin{definicio}
	Az \fogalom{absztrakt-szintaxis} ...
\end{definicio}

\begin{definicio}
	A \fogalom{konkret-szintaxis} ...
\end{definicio}

\section{Grafikus és szöveges nyelvek}

\begin{definicio}
	A \fogalom{nyelvtan} (\fogalomangolul{nyelvtan})...
\end{definicio}

\section{Kódgenerálási Módszerek}

A \fogalom{kodgeneralas} célja \fogalomragozva{modell}{modellből} futtatható \fogalom{forraskod} generálása. ;-)


\fogalomragozva{regularis-kifejezes}{reguláris kifejezések}

\fogalom{fordito}

\fogalom{nezeti-modell}

\fogalom{sablon}

A reguláris kifejezésekkel és a nyelvtanok elméletével bővebben az \algel BSc és a \nyau MSc tárgyak foglalkoznak.

\section{Eszközök\kieg}

ANTLR eszköz~\cite{Parr:2013:DAR:2501720}

Xtext~\cite{Xtext}

\section{Összefoglalás}

\fogalom{produktivitas}++
