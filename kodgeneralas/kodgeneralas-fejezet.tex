% !TeX spellcheck = hu_HU
\topic{Modellező eszközök és kódgenerálási módszerek}
\graphicspath{ {./kodgeneralas/figures/} }

A fejezet célja, hogy megismertessük a korszerű modellező eszközökkel és kódgenerálási technikákkal kapcsolatos fogalmakat, alapvető felépítésésüket és működési elvüket.
A fejezet gyakorlatias megközelítésben tárgyalja a témát, minded pontban bemutatjuk, hogyan és munkával készíthetünk saját, vagy egészíthetünk ki meglévő modellező eszközt.

\section{Modellező eszközök felépítése}
A modellező eszközök célja, hogy különböző modellezési nyelvekhez szerkesztőfelületet nyújtson, és a modellekre épülő automatizált műveletekkel támogassa a fejlesztési folyamatot.

A \emph{modelező eszköz} és a felhasználó \emph{tervező mérnök} közötti interakciókat az \aref{fig:modellezoeszkoz-felepites}.~ábra szemlélteti. A modellező eszköz alapvető feladata, hogy karbantartson egy \emph{modellt}, és \emph{lekérdezés (olvasási, vagy keresési)} és \emph{módosítási (beszúrás, törlés)} műveletekkel láthatóvá és szerkeszthetővé tegye azt a \emph{tervezőmérnöknek}. Ezért a tervezőeszköz különböző \emph{nézeteket} \emph{származtat} a modellekből, amit a mérnök \emph{megjeleníthet} és meghatározott \emph{szerkesztési műveletekkel} változtathat. A modellező eszköz feladata, hogy ezeket a módosításokat visszavezesse a modellbe.
Egy modellhez több nézet is tartozhat, ami különböző részleteit emeli ki a modellnek. 
\remofigscalefixed{modellezoeszkoz-felepites}{Modellező eszközökök alapvető 
felépítése}{0.45}

Az elészült vagy akár félkész modelleken a tervezőmérnök különböző \emph{automatizált műveleteket} kezdeményezhet (tervezési szabályok ellenőrzése, kódgenerálás, modell refactorálása) ami hatására \emph{automatikus transzformációk} hajtódnak végre a modellen. A transzformáció eredménye lehet jelentés (például hibajelentés a tervezési szabályok alapján), újonnan létrehozott dokumentum (például forráskód), vagy egy módosított modell.
A modellezési műveleteket egy fejlett keretrendszernszer magától is elindíthatja fejlesztés közben (például minden mentéskor lefut a tervezési szabályok ellenőrzése), de érdemes ezeket külön meghívhatónak is meghagyni. 

\begin{pelda}
Vegyünk egy a Yakindu tervezőeszközből: a tervezőeszközben megnyithatjuk a \code{.sct} kiterjesztésű fájlokat, melyeket beolvasva elkészíti a program a modell belső adatreprezentációját. Ezután az eszköz készít és megjelenít egy olyan nézetet, amelyben az állapotokat négyzetekkel, az állapotátmeneteket nyilakkal, a triggereket, őrfeltételeket és akciókat pedig szövegesen ábrázolja. A grafikus nézethez szerkesztőfelületet is tartozik, amin keresztül a tervezőmérnök módosíthatja a modellt. A Yakindu tervezőeszköz fontos tulajdonsága, hogy szigorúan ellenőrzi a modelleket és szabályozza modellmódosítások körét. Emiatt a fejlesztés során tipikusan helyes modelleket finomíthatunk egy újfent helyes modellé (ellentétben a forráskóddal, ahol nem ritka hogy órák után fordul újra a program).

Egy yakindu modellből (egy \code{.sgen} kiterjesztésű generátor modell segítségével) kódot is generálhatunk, azaz a modellezésért cserébe komoly fejlesztési feladatokat automatikusan elvégezhetünk. Yakindu esetén egy 10 állapotból álló rendszer is 3000 sor java kódot eredményezhet.
\end{pelda}

% \section{Szintaxis és Szemantika}
% 
% \begin{definicio}
% 	\fogalom{Modellez\'{e}si nyelvnek} nevezzük
% \end{definicio}
% 
% A modelleket 
% 
% \begin{definicio}
% 	\fogalom{Modellez\'{e}si nyelvnek} nevezzük
% \end{definicio}
% 
% 
% \begin{definicio}
% 	A \fogalom{konkret-szintaxis} ...
% \end{definicio}
% 
% \section{Grafikus és szöveges nyelvek}
% 
% \begin{definicio}
% 	A \fogalom{nyelvtan} (\fogalomangolul{nyelvtan})...
% \end{definicio}
% 
% \section{Kódgenerálási Módszerek}
% 
% A \fogalom{kodgeneralas} célja \fogalomragozva{modell}{modellből} futtatható \fogalom{forraskod} generálása. ;-)
% 
% 
% \fogalomragozva{regularis-kifejezes}{reguláris kifejezések}
% 
% \fogalom{fordito}
% 
% \fogalom{nezeti-modell}
% 
% \fogalom{sablon}
% 
% A reguláris kifejezésekkel és a nyelvtanok elméletével bővebben az \algel BSc és a \nyau MSc tárgyak foglalkoznak.
% 
% \section{Eszközök\kieg}
% 
% ANTLR eszköz~\cite{Parr:2013:DAR:2501720}
% 
% Xtext~\cite{Xtext}
% 
% \section{Összefoglalás}
% 
% \fogalom{produktivitas}++
