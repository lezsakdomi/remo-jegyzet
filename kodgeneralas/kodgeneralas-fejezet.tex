% !TeX spellcheck = hu_HU
\topic{Modellező eszközök és kódgenerálási módszerek}
\graphicspath{ {./kodgeneralas/figures/} }

A fejezet célja, hogy megismertessük a korszerű modellező eszközökkel és kódgenerálási technikákkal kapcsolatos fogalmakat, alapvető felépítésésüket és működési elvüket.
A fejezet gyakorlatias megközelítésben tárgyalja a témát, minded pontban bemutatjuk, hogyan és munkával készíthetünk saját, vagy egészíthetünk ki meglévő modellező eszközt.

\section{Modellező eszközök felépítése}
A modellező eszközök célja, hogy különböző modellezési nyelvekhez szerkesztőfelületet nyújtson, és a modellekre épülő automatizált műveletekkel támogassa a fejlesztési folyamatot.

A \emph{modelező eszköz} és a felhasználó \emph{tervező mérnök} közötti interakciókat az \aref{fig:modellezoeszkoz-felepites}.~ábra szemlélteti. A modellező eszköz alapvető feladata, hogy karbantartson egy \emph{modellt}, és \emph{lekérdezés (olvasási, vagy keresési)} és \emph{módosítási (beszúrás, törlés)} műveletekkel láthatóvá és szerkeszthetővé tegye azt a \emph{tervezőmérnöknek}. Ezért a tervezőeszköz különböző \emph{nézeteket} \emph{származtat} a modellekből, amit a mérnök \emph{megjeleníthet} és meghatározott \emph{szerkesztési műveletekkel} változtathat. A modellező eszköz feladata, hogy ezeket a módosításokat visszavezesse a modellbe.
Egy modellhez több nézet is tartozhat, ami különböző részleteit emeli ki a modellnek. 
\remofigscalefixed{modellezoeszkoz-felepites}{Modellező eszközökök alapvető 
felépítése}{0.45}

Az elészült vagy akár félkész modelleken a tervezőmérnök különböző \emph{automatizált műveleteket} kezdeményezhet (tervezési szabályok ellenőrzése, kódgenerálás, modell refactorálása) ami hatására \emph{automatikus transzformációk} hajtódnak végre a modellen. A transzformáció eredménye lehet jelentés (például hibajelentés a tervezési szabályok alapján), újonnan létrehozott dokumentum (például forráskód), vagy egy módosított modell.
A modellezési műveleteket egy fejlett keretrendszernszer magától is elindíthatja fejlesztés közben (például minden mentéskor lefut a tervezési szabályok ellenőrzése), de érdemes ezeket külön meghívhatónak is meghagyni. 

\begin{pelda}
Vegyünk egy a Yakindu tervezőeszközből: a tervezőeszközben megnyithatjuk a \code{.sct} kiterjesztésű fájlokat, melyeket beolvasva elkészíti a program a modell belső adatreprezentációját. Ezután az eszköz készít és megjelenít egy olyan nézetet, amelyben az állapotokat négyzetekkel, az állapotátmeneteket nyilakkal, a triggereket, őrfeltételeket és akciókat pedig szövegesen ábrázolja. A grafikus nézethez szerkesztőfelületet is tartozik, amin keresztül a tervezőmérnök módosíthatja a modellt. A Yakindu tervezőeszköz fontos tulajdonsága, hogy szigorúan ellenőrzi a modelleket és szabályozza modellmódosítások körét. Emiatt a fejlesztés során tipikusan helyes modelleket finomíthatunk egy újfent helyes modellé (ellentétben a forráskóddal, ahol nem ritka hogy órák után fordul újra a program).

Egy yakindu modellből (egy \code{.sgen} kiterjesztésű generátor modell segítségével) kódot is generálhatunk, azaz a modellezésért cserébe komoly fejlesztési feladatokat automatikusan elvégezhetünk. Yakindu esetén egy 10 állapotból álló rendszer is 3000 sor java kódot eredményezhet.
\end{pelda}

\section{Modellek ellenőrzése és ábrázolása}
Ahogy korábban említettük, a helyes modelleket szigorú tervezési szabályok határozzák meg: egy megszabott elemkészlet használva tehetjük össze a modelleket, miközben több szabály be kell tartaniuk: egy fejlett modellezési környezetben a véletlenszerűen összetett diagramok és szövegek szinte mindig hibássak lesznek. Ezeket a szabályokat \emph{szintaxisnak} nevezzük.

\begin{definicio}
\fogalomragozva{szintaxis}{Szintaxisnak} nevezzük a modellekkel szemben támasztott szabályokat. Ha egy modell teljesíti ezeket a szabályokat, akkor azt mondjuk rá hogy szintaktikailag helyes. Amikor modellekről beszélünk, általában szintaktikailag helyes modellekre gondolunk.
\end{definicio}

A modellek elemzése és szerkesztése során külön kezeljük a modellek megjelenítésével és nézeteivel kapcsolatos részleteket a mögöttes logikai felépítéstől. A konkrét ábrázolási módot érintő szabályokat (például állapotgép diagramokban az állapotokat négyzettel ábrázoljuk, a \code{var} kulcsszóval vezetünk be változókat, a szorzásnak magasabb a precedenciája mint az összeadásnak\ldots) \emph{konkrét szintaxisnak} fogjuk nevezni, míg a lényegi mögöttes logikai szerkezetet vizsgáló szabályokat (tranzíciók csak állapotok között mehetnek, tranzíciókon keresztül minden állapotnak elérhetőnek kell lennie) \emph{absztrakt szintaxisnak} mondjuk.

\begin{definicio}
\fogalomragozva{konkret-szintaxis}{Konkrét szintaxisnak} nevezzük a modell ábrázolásával kapcsolatos szabályokat (színek,  alakzatok, kulcsszavak, precedencia, kommentelési szabály). Megkülönböztetünk szöveges és és grafikus szintaxist.
% Ezen kívül konkrét szintaxisnak nevezzük egy modell reprezentáciját magát is (a szöveges ábrázolást vagy a diagramot).
\end{definicio}

Amennyiben egy modell megfelel a konkrét szintaxisnak, a tervezőeszköz felépíti annak logikai strukturáját. A folyamat hasonlít arra, mint ha strukturálisan modelleznénk magukat a modelleket, elhagyván azokból a konkrét megjelenítésből fakadó részleteket (például a koordinátája a modellelemeknek). Ennek a folyamatnak a kimenetele egy olyan struktúramodell lesz, amiben a lényegi elemek szerepelnek.

\begin{definicio}
\fogalomragozva{absztrakt-szintaxis}{Absztrakt szintaxisnak} nevezzük a modellek logikai felépítésére vonatkozó szabályokat, melyek függetlenek minden megjelenítéssel kapcsolatos részlettől. Ezen kívül absztrakt szintaxis modellnek nevezzük magát a logikai vagy számítógépes reprezentációt is.
\end{definicio}

A definíciók szerint absztrakt szintaxis alatt egyaránt értjük a szabályokat és modelleket is. Ez azért van, mert a konkrét és absztrakt szintaxis ellenőrzésével együtt szokott előállni a mögöttes modell is. 

\begin{pelda}
Yakinduban 
\end{pelda}

\remofigscalefixed{concrete}{Állapotgép konkrét szintaxisok Yakindu tervezőeszközben}{0.45}

\remofigscalefixed{abstract}{Kétállapotú állapotgép absztrakt szintaxisa}{0.45}

% \begin{definicio}
% \fogalomragozva{modellezesi-nyelv}{Modellezési nyelvnek} nevezzük az összes megengedett absztrakt szintaxis gráf halmazát.
% \end{definicio}

% 
% \begin{definicio}
% 	\fogalom{Modellez\'{e}si nyelvnek} nevezzük
% \end{definicio}
% 
% A modelleket 
% 
% \begin{definicio}
% 	\fogalom{Modellez\'{e}si nyelvnek} nevezzük
% \end{definicio}
% 
% 
% \begin{definicio}
% 	A \fogalom{konkret-szintaxis} ...
% \end{definicio}
% 
% \section{Grafikus és szöveges nyelvek}
% 
% \begin{definicio}
% 	A \fogalom{nyelvtan} (\fogalomangolul{nyelvtan})...
% \end{definicio}
% 
% \section{Kódgenerálási Módszerek}
% 
% A \fogalom{kodgeneralas} célja \fogalomragozva{modell}{modellből} futtatható \fogalom{forraskod} generálása. ;-)
% 
% 
% \fogalomragozva{regularis-kifejezes}{reguláris kifejezések}
% 
% \fogalom{fordito}
% 
% \fogalom{nezeti-modell}
% 
% \fogalom{sablon}
% 
% A reguláris kifejezésekkel és a nyelvtanok elméletével bővebben az \algel BSc és a \nyau MSc tárgyak foglalkoznak.
% 
% \section{Eszközök\kieg}
% 
% ANTLR eszköz~\cite{Parr:2013:DAR:2501720}
% 
% Xtext~\cite{Xtext}
% 
% \section{Összefoglalás}
% 
% \fogalom{produktivitas}++
