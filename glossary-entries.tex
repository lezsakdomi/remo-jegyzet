% http://mirror.math.ku.edu/tex-archive/macros/latex/contrib/glossaries/
% http://mirror.math.ku.edu/tex-archive/macros/latex/contrib/glossaries/

\newcommand{\fogalomcim}[1]{\gls{#1} (\glsdesc{#1})}
\newcommand{\Fogalomcim}[1]{\Gls{#1} (\glsdesc{#1})}

\newcommand{\fogalom}[1]{\emph{\gls{#1}}}
\newcommand{\Fogalom}[1]{\emph{\Gls{#1}}}

\newcommand{\fogalomangolul}[1]{\emph{\glsdesc{#1}}}
\newcommand{\Fogalomangolul}[1]{\emph{\Glsdesc{#1}}}

\newcommand{\fogalomragozva}[2]{\emph{\glslink{#1}{#2}}}

\newcommand{\rovidites}[1]{\acrshort{#1} (\glsdesc{#1})}
\newcommand{\roviditesmagyarul}[1]{\acrshort{#1} (\glsdesc{#1}, \acrlong{#1})}
\newcommand{\angolrovidites}[1]{(\glsdesc{#1}, \acrshort{#1})}



\IfFileExists{./use-xindy}{
	\makeglossaries
}{
\makenoidxglossaries
}

% a rövidítések megjelenítése a Tárgymutatóban
\newacronymstyle{db-long-short-desc}%
{%
	\GlsUseAcrEntryDispStyle{long-short}%
}%
{%
	\GlsUseAcrStyleDefs{long-short}%
	\renewcommand*{\GenericAcronymFields}{}%
	\renewcommand*{\acronymsort}[2]{##1}%
	\renewcommand*{\acronymentry}[1]{%
		\glsentryshort{##1} \textnormal{\glsentrylong{##1};}}%
}
\setacronymstyle{db-long-short-desc}

\newcommand{\kiejtes}[1]{{\ipa [#1]}}

% roviditesek
\newacronym[description={International Phonetic Alphabet}]{IPA}{IPA}{nemzetközi fonetikai ábécé}
\newacronym[description={breadth-first search}]{BFS}{BFS}{szélességi keresés}
\newacronym[description={depth-first search}]{DFS}{DFS}{mélységi keresés}
\newacronym[description={Eclipse Modeling Framework}]{EMF}{EMF}{Eclipse modellezési keretrendszer}
\newacronym[description={Unified Modeling Language}]{UML}{UML}{egységesített modellező nyelv}
\newacronym[description={Business Process Model and Notation}]{BPMN}{BPMN}{üzleti folyamatmodell és jelölés}
\newacronym[description={Extensible Markup Language}]{XML}{XML}{kiterjeszthető jelölőnyelv}
%\newacronym[description={}]{}{}{}


% fogalmak
\newglossaryentry{formalizmus}{name=formalizmus, description=formalism \kiejtes{ˈfɔːməlɪz(ə)m}}
\newglossaryentry{modszertan}{name=módszertan, description=methodology \kiejtes{meθəˈdɒlədʒi}}
\newglossaryentry{viselkedes}{name=viselkedés, description=behaviour \kiejtes{bɪˈheɪvjə}}
\newglossaryentry{metamodell}{name=metamodell, description=metamodel \kiejtes{ˈmɛtəmɒdl̩}}
\newglossaryentry{modell}{name=modell, description=model \kiejtes{ˈmɒdl̩}}
\newglossaryentry{diagram}{name=diagram, description=diagram \kiejtes{ˈdaɪ.ə.ɡræm}}
\newglossaryentry{modellezesi nyelv}{name=modellezési nyelv, description=modeling language}
\newglossaryentry{absztrakt szintaxis}{name=absztrakt szintaxis, description=abstract syntax}
\newglossaryentry{konkret szintaxis}{name=konkrét szintaxis, description=concrete syntax \kiejtes{ˈkɒŋkriːt}}
\newglossaryentry{szemantika}{name=szemantika, description=semantics \kiejtes{sɪˈmæntɪks}}
\newglossaryentry{szintaxis}{name=szintaxis, description=syntax \kiejtes{ˈsɪn.tæks}}
\newglossaryentry{kornyezet}{name=környezet, description={environment, context}}
\newglossaryentry{fekete doboz}{name=fekete doboz, description=black box}
\newglossaryentry{feher doboz}{name=fehér doboz, description=white box}
\newglossaryentry{veges}{name=véges, description=finite \kiejtes{ˈfaɪnaɪt}}
\newglossaryentry{vegtelen}{name=végtelen, description=infinite \kiejtes{ˈɪnfɪnɪt}}


% struktura alapu modellezes
\newglossaryentry{rendszer}{name=rendszer, description=system \kiejtes{ˈsɪstəm}}
\newglossaryentry{faktoring}{name=faktoring, description=factoring \kiejtes{fæktərɪŋ}}
\newglossaryentry{helyes-dekompozicio}{name=helyes dekompozíció, description=??}
\newglossaryentry{top-down}{name=felülről lefelé, description=top-down}
\newglossaryentry{bottom-up}{name=alulról felfelé, description=bottom-up}
\newglossaryentry{struktura}{name=struktúra, description=structure \kiejtes{ˈstɹʌktʃə(ɹ)}}
\newglossaryentry{strukturalis-modell}{name=strukturális modell, description=structural model}
\newglossaryentry{dekompozicio}{name=dekompozíció, description=decomposition}
\newglossaryentry{kompozicio}{name=kompozíció, description=composition \kiejtes{ˌkɒmpəˈzɪʃən}}
\newglossaryentry{graf}{name=gráf, description=graph \kiejtes{ɡɹɑːf}}
\newglossaryentry{csomopont}{name=csomópont, description={node, vertex} \kiejtes{noʊd, ˈvɜrtɛks}}
\newglossaryentry{el}{name=él, description={edge, relationship}}
\newglossaryentry{cimke}{name=címke, description=label}
\newglossaryentry{fa-graf}{name=fa gráf, description=tree graph}
\newglossaryentry{erdo}{name=erdő, description=forest}
\newglossaryentry{kormentes}{name=körmentes, description=acyclic \kiejtes{eɪˈsʌɪklɪk}}
\newglossaryentry{reszgraf}{name=részgráf, description=subgraph}
\newglossaryentry{iranyitott-graf}{name=irányított gráf, description=directed graph}
\newglossaryentry{iranyitatlan-graf}{name=irányítatlan gráf, description=undirected graph}
\newglossaryentry{hipergraf}{name=hipergráf, description=hypergraph}
\newglossaryentry{relacio}{name=reláció, description=relation}
\newglossaryentry{attributum}{name=attribútum, description=attribute}
\newglossaryentry{jellemzo}{name=jellemző, description=property}
\newglossaryentry{projekcio}{name=projekció, description=projection \kiejtes{pɹəˈdʒɛkʃən}}
\newglossaryentry{vetites}{name=vetítés, description=projection}
\newglossaryentry{szelekcio}{name=szelekció, description=selection \kiejtes{səˈlɛkʃən}}
\newglossaryentry{szures}{name=szűrés, description=filtering}
\newglossaryentry{tipus}{name=típus, description=type}
\newglossaryentry{osztaly}{name=osztály, description=class}
\newglossaryentry{objektum}{name=objektum, description=object}
\newglossaryentry{referencia}{name=referencia, description=reference}
\newglossaryentry{mezo}{name=mező, description=field}
\newglossaryentry{metodus}{name=metódus, description=method}
\newglossaryentry{peldany}{name=példány, description=instance \kiejtes{ˈɪnstəns}}
\newglossaryentry{peldanya}{name=példánya, description=instance of}
\newglossaryentry{tipusa}{name=típusa, description=type of}
\newglossaryentry{orokles}{name=öröklés, description=inheritance \kiejtes{ɪnˈhɛr ɪ təns}}
\newglossaryentry{alosztaly}{name=alosztály, description=subtype}
\newglossaryentry{ososztaly}{name=ősosztály, description=supertype}
\newglossaryentry{leszarmazott}{name=leszármazott, description=descendant \kiejtes{dɪˈsɛndənt}}
\newglossaryentry{os}{name=ős, description=ancestor \kiejtes{ˈæn.sɛs.tɚ}}
\newglossaryentry{tranzitiv}{name=tranzitív, description=transitive \kiejtes{ˈtrænsɪtɪv}}
\newglossaryentry{tranzitiv-lezaras}{name=tranzitív lezárás, description=transitive closure \kiejtes{ˈkləʊʒə(r)}}
\newglossaryentry{taxonomia}{name=taxonómia, description=taxonomy \kiejtes{tækˈsɒnəmi}}
\newglossaryentry{ontologia}{name=ontológia, description=ontology \kiejtes{ɒnˈtɒlədʒi}}
\newglossaryentry{struktura-alapu-modellezes}{name=struktúra alapú modellezés, description=structural modeling}
\newglossaryentry{teljes-graf}{name=teljes gráf, description=complete graph \kiejtes{kəmˈpliːt}}
\newglossaryentry{paros-graf}{name=páros gráf, description=bipartite graph \kiejtes{bʌɪˈpɑːtʌɪt}}
\newglossaryentry{perfekt-graf}{name=perfekt gráf, description=perfect graph \kiejtes{pəˈfɛkt}}
\newglossaryentry{intervallumgraf}{name=intervallumgráf, description=interval graph \kiejtes{ˈɪntəv(ə)l}}
\newglossaryentry{sikbarajzolhato-graf}{name=síkbarajzolható gráf, description=planar graph \kiejtes{ˈpleɪnə}}
\newglossaryentry{gyoker-csomopont}{name=gyökér csomópont, description=root note}
\newglossaryentry{gyokeres-fa}{name=gyökeres fa, description=rooted tree}
\newglossaryentry{gyokeres-szintezett-fa}{name={gyökeres, szintezett fa}, description=leveled tree}
\newglossaryentry{kor}{name=kör, description=cycle}
\newglossaryentry{ut}{name=út, description=path \kiejtes{pɑːθ}}
\newglossaryentry{seta}{name=séta, description={walk, chain}}
\newglossaryentry{tipusgraf}{name=típusgráf, description=type graph}
\newglossaryentry{relacioalgebra}{name=relációalgebra, description=relational algebra}
\newglossaryentry{tulajdonsag}{name=tulajdonság, description=property}
\newglossaryentry{reflexivitas}{name=reflexivitás, description=reflexivity \kiejtes{ˌriflɛkˈsɪvɪti}}
\newglossaryentry{szimmetria}{name=szimmetria, description=symmetry \kiejtes{ˈsɪmɪtri/}}
\newglossaryentry{refaktoring}{name=refaktoring, description=refactoring}
\newglossaryentry{Dijsktra-algoritmus}{name=Dijkstra algoritmus, description=Dijksta's algorithm \kiejtes{ˈdɛɪkstras ˈælɡəɹɪðm/}}
\newglossaryentry{A-csillag-kereses}{name=A* keresés, description=A* search}
\newglossaryentry{Bellman-Ford-algoritmus}{name=Bellman-Ford algorithmus, description=Bellman--Ford algorithm}
\newglossaryentry{Floyd-algoritmus}{name=Floyd-algoritmus, description=Floyd--Warshall algorithm}
\newglossaryentry{multigraf}{name=multigráf, description=multigraph}
\newglossaryentry{tobbszoros-el}{name=többszörös él, description=multiedge}
\newglossaryentry{hiperel}{name=hiperélek, description=hyperedge}
\newglossaryentry{szures-eltipusra}{name=szűrés éltípusra, description=filtering by edge type}
\newglossaryentry{cimkezett-graf}{name=címkézett gráf, description=labeled graph}
\newglossaryentry{egyedi-azonosito}{name=egyedi azonosító, description=unique identifier}
\newglossaryentry{csucscimkezett-graf}{name=csúcscímkézett gráf, description=vertex-labeled graph}
\newglossaryentry{elcimkezett-graf}{name=élcímkézett gráf, description=edge-labeled graph}
\newglossaryentry{tipusos-graf}{name=típusos gráf, description=typed graph}
\newglossaryentry{sulyozott-graf}{name=súlyozott gráf, description=weighted graph}
\newglossaryentry{tulajdonsaggraf}{name=tulajdonsággráf, description=property graph}

%\newglossaryentry{}{name=, description=}

%\item \fogalom{} (ejtsd: ,,á-csillag keresés'')
%\item \fogalom{}
%\item \fogalom{Floyd-algoritmus}


% allapot alapu modellezes
\newglossaryentry{determinisztikus}{name=determinisztikus, description=deterministic}
\newglossaryentry{allapot}{name=állapot, description=state \kiejtes{steɪt}}
\newglossaryentry{allapotter}{name=állapottér, description=state space}
\newglossaryentry{allapot-alapu-modellezes}{name=állapot alapú modellezés, description=state modeling}
\newglossaryentry{mealy-automata}{name=Mealy-automata, description=Mealy automaton}
\newglossaryentry{teljesseg}{name=teljesség, description=completeness}
\newglossaryentry{kizarolagossag}{name=kizárólagosság, description=(mutual) exclusiveness}
\newglossaryentry{kezdoallapot}{name=kezdőállapot, description=initial state}
\newglossaryentry{tranzicio}{name=tranzíció, description=transition \kiejtes{tɹænˈzɪʃən}}
\newglossaryentry{allapotatmenet}{name=állapotátmenet, description=state transition}
\newglossaryentry{tuzeles}{name=tüzelés, description=firing}
\newglossaryentry{akcio}{name=akció, description=action}
\newglossaryentry{diszkret-allapotter}{name=diszkrét állapottér, description=discrete state space}
\newglossaryentry{regio}{name=regió, description=region \kiejtes{ɹiːdʒn̩}}
\newglossaryentry{osszetett-allapot}{name=összetett állapot, description=composite state}
\newglossaryentry{allapotmentes}{name=állapotmentes, description=stateless}
\newglossaryentry{allapotos}{name=állapotos, description=stateful}
\newglossaryentry{szinkron}{name=szinkron, description=synchronous \kiejtes{ˈsɪŋkrənəs}}
\newglossaryentry{aszinkron}{name=aszinkron, description=asynchronous \kiejtes{eɪˈsɪŋkrənəs}}
\newglossaryentry{ortogonalis-allapot}{name=ortogonális állapot, description=orthogonal state}
\newglossaryentry{aszinkron-szorzas}{name=aszinkron szorzás, description=asynchronous product}
\newglossaryentry{szorzatautomata}{name=szorzatautomata, description=product automaton}
\newglossaryentry{allapotter-robbanas}{name=állapottér-robbanás, description=state space explosion}
\newglossaryentry{ortogonalis-regio}{name=ortogonális régió, description=orthogonal region}
\newglossaryentry{versenyhelyzet}{name=versenyhelyzet, description=race condition}
\newglossaryentry{determinizmus}{name=determinizmus, description=determinism}
\newglossaryentry{nemdeterminizmus}{name=nemdeterminizmus, description=nondeterminism}
\newglossaryentry{teljesen-specifikalt}{name=teljesen specifikált, description=fully specified}
\newglossaryentry{holtpontmentes}{name=holtpontmentes, description=deadlock-free}
\newglossaryentry{szimulacio}{name=szimuláció, description=simulation}
\newglossaryentry{allapotcsomopont}{name=állapotcsomópont, description=state node}
\newglossaryentry{szintaktikai-jelentes}{name=szintaktikai jelentés, description=syntactic meaning}
\newglossaryentry{szemantikai-jelentes}{name=szemantikai jelentés, description=semantic meaning}
\newglossaryentry{pillanatnyi-allapot}{name=pillanatnyi állapot, description=current state}
\newglossaryentry{vegrehajtasi-szekvencia}{name=végrehajtási szekvenciái,description=execution trace? \kiejtes{ˌek.sɪˈkjuː.ʃən tɹeɪs}}
\newglossaryentry{elerheto}{name=elérhető, description=reachable}
\newglossaryentry{allapotkonfiguracio}{name=állapotkonfiguráció, description=state configuration}
\newglossaryentry{finomitas}{name=finomítás, description=refinement \kiejtes{rɪˈfʌɪnm(ə)nt}}
\newglossaryentry{absztrakcio}{name=absztrakció, description=abstraction \kiejtes{əbˈstɹæk.ʃn̩}}
\newglossaryentry{valtozo}{name=változó, description=variable}
\newglossaryentry{valtozoertekeles}{name=változóértékelés, description=variable interpretation ?}
\newglossaryentry{utasitas}{name=utasítás, description=instruction}
\newglossaryentry{interfeszvaltozo}{name=interfészváltozó, description=interface variable}
\newglossaryentry{interfesz}{name=interfész, description=interface}

% folyamatmodellezes
\newglossaryentry{folyamatmodell}{name=folyamatmodell, description=business process model}
\newglossaryentry{esemeny}{name=esemény, description=event \kiejtes{ɪˈvɛnt}}
\newglossaryentry{konfliktus}{name=konfliktus, description=conflict \kiejtes{ˈkɒn.flɪkt}}

% modellek ellenőrzése
\newglossaryentry{funkcionalis-kovetelmeny}{name=funkcionális követelmény, description=functional requirement}
\newglossaryentry{nemfunkcionalis-kovetelmeny}{name=nemfunkcionális követelmény, description=non-functional requirement}
\newglossaryentry{biztonsagi-kovetelmeny}{name=biztonsági követelmény, description=safety requirement}
\newglossaryentry{elosegi-kovetelmeny}{name=élőségi követelmény, description=liveness requirement}
%\newglossaryentry{megengedett-viselkedes}{name=megengedett viselkedés, description=permitted behaviour}
%\newglossaryentry{elvart-viselkedes}{name=elvárt viselkedés, description=required behaviour}
\newglossaryentry{verifikacio}{name=verifikáció, description=verification \kiejtes{verɪfɪkeɪʃn}}
\newglossaryentry{validacio}{name=validáció, description=validation \kiejtes{ˌvæl.əˈdeɪ.ʃən}}
\newglossaryentry{teszteles}{name=tesztelés, description=testing}
\newglossaryentry{tesztelendo-rendszer}{name={tesztelendő rendszer},description={system under test (SUT)}}
\newglossaryentry{tesztbemenet}{name=tesztbemenet, description=test input}
\newglossaryentry{teszteset}{name=teszteset, description=test case}
\newglossaryentry{tesztfuttatas}{name=tesztfuttatás, description=test execution}
\newglossaryentry{tesztkeszlet}{name=tesztkészlet, description=test suite}
\newglossaryentry{tesztorakulum}{name=tesztorákulum, description=test oracle \kiejtes{ɒrəkəl}}
\newglossaryentry{modulteszt}{name=modulteszt, description=module testing}
\newglossaryentry{komponensteszt}{name=komponensteszt, description=component testing}
\newglossaryentry{egysegteszt}{name=egységteszt, description=unit testing}
\newglossaryentry{integracios-teszt}{name=integrációs teszt, description=integration testing}
\newglossaryentry{rendszerteszt}{name=rendszerteszt, description=system testing}
\newglossaryentry{regresszios-teszt}{name=regressziós teszt, description=regression testing}
\newglossaryentry{allapotfedettseg}{name=állapotfedettség, description=state coverage}
\newglossaryentry{atmenetfedettseg}{name=átmenetfedettség, description=transition coverage}
\newglossaryentry{utasitasfedettseg}{name=utasításfedettség, description=statement coverage}
\newglossaryentry{jolstrukturalt-folyamatmodell}{name={jólstrukturált folyamatmodell},description={well-structured process model}}
\newglossaryentry{szimbolikus-vegrehajtas}{name={szimbolikus végrehajtás},description={symbolic execution}}
\newglossaryentry{formalis-verifikacio}{name={formális verifikáció},description={formal verification}}
\newglossaryentry{modellellenorzes}{name={modellellenőrzés},description={model checking}}
\newglossaryentry{holtpont}{name={holtpont},description={deadlock}}
\newglossaryentry{holtpontmentesseg}{name={holtpontmentesség},description={deadlock freedom}}
\newglossaryentry{livelock}{name={livelock},description={livelock}}
\newglossaryentry{ellenpelda}{name={ellenpélda},description={counterexample}}
\newglossaryentry{debugolas}{name={debugolás},description={debugging}}
\newglossaryentry{invarians}{name={invariáns},description={invariant} \kiejtes{ɪnˈvɛəriənt}}

% teljesitmenymodellezes
\newglossaryentry{felhasznaloi-keres}{name=felhasználói kérés, description=user request}
\newglossaryentry{eroforras}{name=erőforrás, description=resource}
\newglossaryentry{atlag}{name=átlag, description={average, mean}}
\newglossaryentry{tranzakcio}{name=tranzakció, description=transaction}
\newglossaryentry{keres}{name=kérés, description=request \kiejtes{ɹɪˈkwɛst}}
\newglossaryentry{rendszer-hatara}{name=rendszer határa, description=system boundary \kiejtes{ˈbaʊndɹi}}
\newglossaryentry{erkezesi-rata}{name=érkezési ráta, description=arrival rate}
\newglossaryentry{atbocsatas}{name=átbocsátás, description=throughput}
\newglossaryentry{valaszido}{name=válaszidő, description=service time}
\newglossaryentry{rendszerben-levo-keresek-atlagos-szama}{name=rendszerben lévő kérések átlagos száma, description=??}
\newglossaryentry{egyensulyi-allapot}{name=egyensúlyi állapot, description=}
\newglossaryentry{Little-torveny}{name=Little-törvény, description=Little's Law}
\newglossaryentry{szolgaltatasigeny}{name=szolgáltatásigény, description=}
\newglossaryentry{kihasznaltsag}{name=kihasználtság, description=}
\newglossaryentry{kihasznaltsag-torvenye}{name=kihasználtság törvénye, description=}
\newglossaryentry{forced-flow-torveny}{name=Forced Flow törvény, description=Forced Flow Law}
\newglossaryentry{atbocsatokepesseg}{name=átbocsátóképesség, description=}
\newglossaryentry{szuk-keresztmetszet}{name=szűk keresztmetszet, description=bottleneck \kiejtes{ˈbɒtlˌnɛk}}
\newglossaryentry{szolgaltatasigeny-torvenye}{name=szolgáltatásigény törvénye, description=}
\newglossaryentry{meresi-ido}{name=mérési idő, description=}
\newglossaryentry{tranzakciok-szama}{name=tranzakciók száma, description=}
\newglossaryentry{foglaltsagi-ido}{name=foglaltsági idő, description=busy time}
\newglossaryentry{latogatasok-atlagos-szama}{name=látogatások átlagos száma, description=}
\newglossaryentry{eroforrasigeny}{name=erőforrásigény, description=}

% EDA
\newglossaryentry{szoras}{name=szórás, description=standard deviation}
\newglossaryentry{szorasnegyzet}{name=szórásnégyzet, description=variance}
\newglossaryentry{median}{name=medián, description=median}
\newglossaryentry{varhato-ertek}{name=várható érték, description=expected value}
\newglossaryentry{valoszinusegi-valtozo}{name=valószínűségi változó, description=random variable}
\newglossaryentry{megfigyeles}{name=megfigyelés, description=sample}
\newglossaryentry{tapasztalati-atlag}{name=tapasztalati átlag, description=sample mean}
\newglossaryentry{korrigalt-tapasztalati-szoras}{name=korrigált tapasztalati szórás, description=unbiased sample standard deviation}

%\newglossaryentry{}{name=, description=}
