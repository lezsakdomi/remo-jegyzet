% !TeX spellcheck = hu_HU
\topic{Modellek ellenőrzése}\label{sec:modellek-ellenorzese}

\newcommand{\kiegeszitoanyag}{\textsuperscript{($\ast$)}}
\newcommand{\modellekEllenorzeseAllapotgepScale}{0.5}

A korábbiakban láthattuk, hogy modelleket számos célból készíthetünk. Függetlenül attól, hogy a modellek felhasználási célja a dokumentáció, a kommunikáció segítése, az analízis vagy a megvalósítás származtatása, a modellek minősége fontos kérdés. Egy hibás modell könnyen vezethet hibás megvalósításhoz, amely pedig a felhasználási területtől függően katasztrofális következményekkel is járhat.



\section{Követelmények modellekkel szemben}

Fontos megjegyezni, hogy a modellek ``helyessége'' önmagában nem egy értelmes, vizsgálható kérdés. Ennek vizsgálatához fontos tudni, hogy mi a modell célja, kontextusa, mik a követelmények vele szemben.

\begin{megjegyzes}
	Gondoljunk a BKK egyszerűsített, sematikus metróhálózati térképére\footnote{\url{http://www.bkk.hu/apps/docs/terkep/metro.pdf}}! Ez tekinthető a metróhálózat egy (gráfalapú) modelljének. Ha a követelményünk a metróhálózat sematikus reprezentálása, ami alatt például az állomások megfelelő sorrendjét, illetve a helyes átszállási kapcsolatokat értjük, akkor ez a térkép (gráf) helyes. Hibás akkor lenne, ha például a Nyugati pályaudvar az M4-es metró egyik állomásaként lenne feltüntetve. Ha viszont a követelmény az, hogy a térképről leolvashatók legyenek a metróállomások közti távolságok, ez a térkép hibás, hiszen a térkép alapján az Újbuda-központ és Móricz Zsigmond körtér állomások és a Móricz Zsigmond körtér és Szent Gellért tér állomások közti távolság azonos, míg a valóságban az egyik távolság a másiknak közel a duplája. 
\end{megjegyzes}

Látható, hogy önmagában nincs értelme egy modell helyességéről beszélni, csak arról beszélhetünk, hogy bizonyos meghatározott követelményeknek megfelel-e vagy sem.
Ebben a fejezetben áttekintjük, milyen követelményeket támaszthatunk a modellekkel szemben, hogyan csoportosíthatjuk és hogyan ellenőrizhetjük ezen követelményeket.

\paragraph{Követelmények funkcionalitás szerint.}
Az egyik leggyakoribb felosztás a követelményeket aszerint különbözteti meg, hogy a rendszer elsődleges funkcióját írják le vagy sem. \fogalomragozva{funkcionalis-kovetelmeny}{Funkcionális követelményeknek} nevezzük azokat a követelményeket, amelyek egy rendszer(összetevő) által ellátandó funkciót definiálnak \cite{IEEE-24765}.
\fogalomragozva{nemfunkcionalis-kovetelmeny}{Nemfunkcionális követelményeknek} (vagy extrafunkcionális követelményeknek) nevezzük az ezeken kívül eső követelményeket, amelyek a rendszer minőségére vonatkoznak, például megbízhatóságra, teljesítményre vonatkozó kritériumok \cite{IEEE-24765}.

\begin{megjegyzes}
Tehát a \fogalomragozva{funkcionalis-kovetelmeny}{funkcionális követelmények} meghatározzák, \emph{mit} fog a rendszer csinálni. A \fogalomragozva{nemfunkcionalis-kovetelmeny}{nemfunkcionális követelmények} arról szólnak, \emph{hogyan} kell a rendszernek ezeket a funkciókat ellátnia.
\end{megjegyzes}

\paragraph{Biztonsági és élőségi követelmények.}
A követelmények egy másik klasszikus kategorizálása alapján biztonsági és élőségi követelményeket különböztetünk meg \cite{Lamport:1977}.
A \fogalomragozva{biztonsagi-kovetelmeny}{biztonsági követelmények} a megengedett viselkedést definiálják: megadják, hogy milyen viselkedés engedélyezett és mely viselkedések tiltottak. Ezek univerzális követelmények, melyeknek a rendszerre minden időpillanatban teljesülniük kell.
Az \fogalomragozva{elosegi-kovetelmeny}{élőségi követelmények} az elvárt viselkedést definiálják.
Ezek egzisztenciális követelmények, amelyek szerint a rendszer megfelelő körülmények közt előbb-utóbb teljesíteni képes bizonyos elvárásokat.
Bizonyos követelmények biztonsági és élőségi követelmények keverékei, így -- különösen komplex esetekben -- nem feltétlen lehetséges valamelyik kategóriába sorolni a követelményt.

\begin{megjegyzes}
Biztonsági követelmény például az, hogy egy jelzőlámpán egyszerre sosem világíthat a piros és a zöld fény. Élőségi követelmény, hogy a lámpa (előbb-utóbb) képes legyen zöldre váltani.
%Biztonsági követelmény például az, hogy egy jegyautomatánál a visszajáró összeg mindig nemnegatív. Élőségi követelmény, hogy folyamatban lévő tranzakció esetén a tranzakció mindig megszakítható és az addig bedobott pénz visszaadásra kerül.
\end{megjegyzes}


\paragraph{Gyakori általános követelmények.}
Itt kitérünk néhány olyan gyakori általános követelményre, amelyek általában a modell által leírt folyamattól, vagy a megvalósított rendszer érdemi funkcionalitásától lényegében függetlenek.
Az egyik ilyen általános követelmény a holtpontmentesség. \fogalomragozva{holtpont}{Holtpontnak} (deadlocknak) nevezzük azt az állapotot egy rendszerben, amelyben a végrehajtás megáll, a rendszer többé nem képes állapotot váltani, és nem mutat semmilyen viselkedést. A holtpont egy gyakran előforduló oka, ha a rendszerben két vagy több folyamat egymásra várakozik. Ez egy olyan állapot, amelyből külső (nem modellezett) beavatkozás nélkül nem lehet kilépni. Emiatt párhuzamos rendszereknél egy gyakori követelmény a \fogalom{holtpontmentesseg}, a holtpontok lehetőségének hiánya\footnote{Természetesen előfordulhat olyan eset is, hogy a holtpont nem tiltott, hanem megfelelő körülmények közt egyenesen elvárt. Emlékezzünk arra, hogy egy folyamatpéldány a dolga végeztével semmilyen további viselkedést nem mutat. Ilyenkor inkább az lehet követelmény, hogy a folyamat csak úgy kerülhessen holtpontba, ha már befejeződött.}. Sajnos ez egy olyan probléma, amely kifinomult eszközök nélkül igen nehezen vizsgálható, gyakran a holtpont létrejöttéhez a körülmények ritka, különleges együttállása szükséges.

Hasonló fogalom a livelock. \Fogalom{livelock} esetén az érdemi végrehajtás ugyanúgy megáll, mint holtpont esetén. Ennek viszont nem az az oka, hogy a rendszer nem képes állapotot váltani, hanem az, hogy a livelockban résztvevő komponensek egy végtelen ciklusba ragadnak, amelyben nem végeznek hasznos tevékenységet.

\begin{megjegyzes}
A klasszikus példa livelock-ra a való életből az, amikor két ember szembetalálkozik, és mindketten udvariasan kitérnének egymás elől, de azt mindig megegyező irányba teszik. Ilyenkor az emberek tudnak mozogni (``képesek állapotot váltani''), de nem haladnak előre (``nem végeznek hasznos tevékenységet''). Holtpontról akkor beszélnénk, ha kölcsönösen nem tudnának megmozdulni a másik személy miatt. Tehát a holtpont esetén a problémát az ``örök várakozás'', livelock esetén pedig egy végtelen ciklus okozza.
\end{megjegyzes}

Állapotgépek esetén gyakori általános követelmények például a determinizmus és a teljesen specifikált működés. Ezekről bővebben \iflabeldef{sec:allapot-alapu-modellezes}{\aref{sec:allapot-alapu-modellezes}.~szakaszban}{az Állapotalapú modellezés c. segédanyagban} szólunk.

\paragraph{Vizsgálatok fajtái.}
Ha rendelkezésre állnak a követelmények, már lehetséges a modellek helyességének vizsgálata. Azonban a ``helyességvizsgálat'' nem egy precíz fogalom. 

\begin{megjegyzes}
Képzeljük el, hogy egy kereszteződést szerelünk fel jelzőlámpákkal. A leszállított rendszert ellenőriztük, megfelel a specifikációnak: a fények megfelelő sorrendben, megfelelő időzítéssel követik egymást és mindig csak az engedélyezett irányok kapnak egyidejűleg szabad jelzést. Az átadáskor a megrendelő megkérdezi: ``-- És hol lehet átkapcsolni villogó sárgára?'' Mivel ez nem volt a specifikáció része, ilyen funkció nem is került a jelzőlámpák vezérlésébe. Mi meg vagyunk győződve arról, hogy a rendszer helyes, mivel teljesíti a specifikáció minden elemét. Ugyanakkor a megrendelő biztos abban, hogy a rendszer hibás, hiszen nem megfelelő a számára.
\end{megjegyzes}

Annak érdekében, hogy a helyességvizsgálatról pontosabban beszéljünk, két új fogalmat vezetünk be. \fogalomragozva{verifikacio}{Verifikációnak} nevezzük, amikor azt vizsgáljuk, hogy az implementáció (az elkészített modell vagy rendszer) megfelel-e a specifikációnak. Ekkor a kérdés az, hogy helyesen fejlesztjük-e a rendszert, megfelel-e az az előírt kívánalmaknak. \fogalomragozva{validacio}{Validációnak} nevezzük azt a folyamatot, amelyben a rendszert a felhasználói elvárásokhoz hasonlítjuk, azaz azt vizsgáljuk, hogy a megfelelő rendszert fejlesztjük-e. Mint ahogyan azt a korábbi példán láthattuk, a sikeres verifikáció nem feltétlen jár együtt sikeres validációval. A verifikáció és a validáció közti különbséget illusztrálja \aref{fig:verifikacio_vs_validacio}.~ábra.

\begin{figure}[h]
	\centering
	\input{modellek-ellenorzese/figures/verifikacio_vs_validacio.pdf_tex}
	
	\caption{A verifikáció és a validáció közti különbség illusztrációja}
	\label{fig:verifikacio_vs_validacio}
\end{figure}

A modellek vagy rendszerek ellenőrzésére többféle módszer is rendelkezésre áll, ezeket mutatjuk be a fejezet hátralévő részében. Először a fontosabb statikus ellenőrzési technikákat ismertetjük (\ref{sec:statikus-ellenorzes}.~szakasz), amelyekhez a rendszert nem szükséges futtatni. Utána a tesztelést mutatjuk be (\ref{sec:teszteles}.~szakasz), amely egy dinamikus, a rendszert futás közben, de még fejlesztési időben megfigyelő módszer. Ha a rendszert normál üzemű futás közben is meg szeretnénk figyelni, futásidejű verifikációról beszélünk, amelyről a \ref{sec:futasideju-verifikacio}.~szakaszban szólunk. A fejezetet a formális ellenőrzési módszerekre történő kitekintéssel zárjuk (\ref{sec:formalis-verifikacio}.~szakasz).



\section{Statikus ellenőrzés}\label{sec:statikus-ellenorzes}
Statikus ellenőrzés során a vizsgált rendszert vagy modellt annak végrehajtása, szimulációja nélkül elemezzük. Bizonyos hibák ``ránézésre látszanak'', könnyen felismerhetők, ezekre célszerű statikus ellenőrzési módszereket alkalmazni. Ilyenkor a statikus vizsgálat alapvető előnye, hogy gyorsan és könnyen szolgáltat eredményt. Ráadásul a statikus ellenőrzési módszerek gyakran a hiba helyét is pontosan behatárolják, míg például dinamikus módszereknél gyakran a hiba felismerése és annak okának megtalálása két külön feladat. Statikus ellenőrzési technikákat akkor is használunk, amikor szintaktikai hibákat keresünk, ilyenkor a rendszer tipikusan nem is futtatható.

Az alábbiakban részletesen foglalkozunk a statikus ellenőrzés technikáival és használatával. Először a szintaktikai hibák vizsgálatát tekintjük át (\ref{sec:statikus-ellenorzes-szintaktikai-hibak}.~szakasz), majd a szemantikai hibák vizsgálatára koncentrálunk (\ref{sec:statikus-ellenorzes-szemantikai-hibak}.~szakasz). 

\subsection{Szintaktikai hibák vizsgálata}\label{sec:statikus-ellenorzes-szintaktikai-hibak}

Szintaktikai hibáknak nevezzük azokat a hibákat, amelyek következtében egy modell nem felel meg a metamodelljének, vagy egy program nem felel meg a használt programozási nyelv formai megkötéseinek. Ilyen lehet például egy állapotokhoz nem kötött állapotátmenet egy állapotgépben vagy egy hiányzó zárójel egy programban.

Grafikus modellek esetén tipikus, hogy a szerkesztő megakadályozza a szintaktikai hibák elkövetését és betartatja a strukturális helyességet, de szöveges leírások esetén ezek elkerülhetetlenek a fejlesztés folyamán. A modern fejlesztőeszközök általában már a beírás során jelzik a szintaktikai hibákat a fejlesztő számára, így azok azonnal javíthatók. Ilyenkor az eszköz beépített szintaktikai statikus ellenőrzőjét láthatjuk működni. Más esetekben (például egyszerű szöveges szerkesztő használata esetén) az esetleges szintaktikai hibákra csak fordítás vagy végrehajtás során derül fény.

Általánosan igaz a szintaktikai hibákra, hogy ezek nagy biztonsággal kimutathatók (legkésőbb futtatás vagy végrehajtás során) és ritka, hogy egy statikus ellenőrző helyes kódot vagy modellt szintaktikailag hibásnak értékelne.

\subsection{Szemantikai hibák vizsgálata}\label{sec:statikus-ellenorzes-szemantikai-hibak}

Szemantikai hibákról akkor beszélünk, ha a fejlesztés alatt álló rendszer szintaktikailag helyes, ugyanakkor valószínűsíthetően nem értelmes vagy nem az elvárt módon fog viselkedni. Ha egy C programban leírjuk, hogy \code{x = y / 0;}, akkor az szintaktikailag helyes (feltéve, hogy az \code{x} és \code{y} változók definiáltak és megfelelő kontextusban szerepel a fenti értékadás), ugyanakkor triviálisan nullával való osztáshoz vezet, amely a legritkábban kívánatos egy programban. 

Gyakran a szemantikai problémák nem olyan egyértelműek, mint a fenti nullával osztás. Például az \code{if (x = 1) { ... }} C kódban gyanús az értékadás, feltehetően a fejlesztő szándéka az \code{x} változó értékétől függő feltételes elágazás implementálása volt, ugyanakkor ezt biztosan nem tudhatjuk. Az ilyen gyanús kódrészleteket angolul \emph{code smell}nek hívjuk, és a statikus ellenőrző eszközök tipikusan ezekre is felhívják a figyelmet.

Hasonló probléma folyamatmodellek esetén az alábbi ábrán látható:

\remofigscale{modellek-ellenorzese/figures/folyamatmodell-decision-join}{Folyamatmodell decision és join elemmel \textbf{TODO: kicserélni egy konzisztens stílusú ábrára}}{\modellekEllenorzeseAllapotgepScale}

A fenti folyamatmodell szintaktikailag helyes, azonban ha közelebbről megvizsgáljuk látszik, hogy valószínűleg szemantikailag helytelen. A decision elem miatt vagy a \allapot{Kártyás fizetés}, vagy a \allapot{Kasszakezelés} tevékenység lesz aktív. Mivel a join mindkét bemeneten tokent fog várni, sosem léphet tovább (tehát holtpontra jut), és így a \allapot{Készletcsökkentés} tevékenység sosem fut le.

Hasonló a helyzet az alábbi folyamatmodellnél.
\remofigscale{modellek-ellenorzese/figures/folyamatmodell-ket-terminalo}{Folyamatmodell két termináló csomóponttal \textbf{TODO: kicserélni egy konzisztens stílusú ábrára}}{\modellekEllenorzeseAllapotgepScale}

A fenti folyamatmodell szintén helyes szintaktikailag, viszont amint a \allapot{Pontokat kihirdet} vagy a \allapot{Jegyeket beír} tevékenység befejeződik, a termináló csomópont leállítja a \emph{teljes} folyamatot, így a másik tevékenység nem fog tudni lefutni.

\paragraph{Védekezés szemantikai hibák ellen.}
Két egyszerűbb módszert ismertetünk itt a fenti hibák kivédése érdekében. Az egyik módszerrel a hibák felismerhetők, a másik módszerrel pedig megelőzhetők.

\begin{itemize}
	\item Ha azonosítottuk a fenti szituációk közül a leggyakrabban előfordulókat, \emph{hibamintákat} fogalmazhatunk meg rájuk. Ez után a statikus ellenőrző eszköz ezeket a hibamintákat keresi a modellben vagy a forráskódban. Például ha egy decision elem egy join elemmel áll párban egy folyamatmodell vagy a kódban egy \code{if (<változó> = <érték>)} minta található, erre felhívhatja a felhasználó figyelmét, aki ezután javíthatja a modellt vagy figyelmen kívül hagyhatja a jelzést.

	\item Lehetőségünk van ezeket a hibákat megelőzni, ha a modellezési vagy programozási nyelv szintaktikájánál önként erősebb megkötéseket alkalmazunk. Ilyen megkötések a kódolási szabályok (például a mutatók használatának tiltása C-ben) vagy a jólstrukturált folyamatmodellek. \fogalomragozva{jolstrukturalt-folyamatmodell}{Jólstrukturált folyamatmodellek} esetén kikötjük, hogy a folyamatmodell kizárólag \aref{fig:jolstrukturalt-folyamatmodellek-mintai}.~ábrán látható mintákból állítható elő: üres folyamat, elemi tevékenység, szekvencia, ciklus, döntés, párhuzamosság. Ezzel a szintaxist úgy kötjük meg, hogy a tipikus szemantikai hibák ne állhassanak elő. A jólstrukturált folyamatmodellekről bővebben \iflabeldef{sec:jolstrukturalt-folyamatok}{\aref{sec:jolstrukturalt-folyamatok}.~szakaszban}{a Folyamatmodellezés c. segédanyagban} szólunk.
\end{itemize}

\begin{figure}[h]
	\centering
	\textbf{TODO -- megrajzolni a többivel egységes formában}
	
	\caption{Jólstrukturált folyamatmodellek megengedett mintái}
	\label{fig:jolstrukturalt-folyamatmodellek-mintai}
\end{figure}

Az, hogy mit tartunk szemantikai hibának függ a használt modellezési vagy programozási nyelvtől, de függhet az alkalmazási területtől vagy a konkrét felhasználástól is. Bizonyos alkalmazási területeken további \emph{tervezési szabályokat} definiálunk, amelyek tovább korlátozhatják a modellezés vagy programozás szabadságát. Például biztonságkritikus programok esetén gyakran tiltott a dinamikus memóriafoglalás. Ilyen esetekben kiegészíthetjük a hibaminták készletét a \code{malloc} és hasonló konstrukciókkal.

\paragraph{Szimbolikus végrehajtás.}
Bizonyos esetekben a szemantikai hibák kiszűrése bonyolultabb feladat. Gondoljunk például az \code{x = y / z;} kódrészletre. Előfordulhat itt nullával osztás? A válasz nem egyértelmű, függ \code{z} értékétől.

\begin{megjegyzes}
Tekintsük például az alábbi C kódot:
\begin{lstlisting}[language=C]
int foo(int z) {
	int y;
	
	y = z + 10;
	if (y != 10) {
		x = y / z;
	} else {
		x = 2;
	}
	return x;
}
\end{lstlisting}

Lehet \code{z} értéke 0? Természetesen, hiszen \code{z} egy bementi változó. Vizsgáljuk a \code{z}-vel osztás előtt annak az értékét? Nem, csak \code{y} változót vizsgáljuk. Lehetséges a nullával osztás a kódban? Nem. Amikor \code{z} értéke nulla lenne, akkor \code{y} értéke 10 lesz, és így az osztást tartalmazó utasítás nem fut le.
\end{megjegyzes}

Amikor végrehajtunk egy programot, mindig a változók bizonyos konkrét értékei mellett tesszük ezt (pl. \code{foo(5)}). \Fogalom{szimbolikus-vegrehajtas} esetén konkrét értékek helyett szimbolikus értékekkel ``imitáljuk'' a végrehajtást, azaz a változókat matematikai változóként fogjuk fel. Emellett a belső elágazások által támasztott feltételeket is összegyűjtjük, majd ezen információk alapján következtetünk az egyes változók értékeire a program adott pontjain, vagy egyes programrészek elérhetőségére.

\begin{megjegyzes}
Ha \code{z}-t egy $z$ matematikai változónak tekintjük, akkor tudjuk, hogy a feltételes elágazás \emph{igaz} ágában a \code{y = z + 10;} utasítás miatt $y = z + 10$ igaz, valamint az elágazási feltételben szereplő \code{y != 10} miatt $y \neq 10$ igaz. A kettőből együttesen $z + 10 \neq 10$, azaz $z \neq 0$. Így bizonyíthatjuk, hogy sosem osztunk nullával a fenti példakódban.
\end{megjegyzes}

\begin{megjegyzes}
Milyen esetekben ad a fenti példakód 2-t eredményül? Erre szintén választ kaphatunk szimbolikus végrehajtás segítségével, míg konkrét végrehajtással minden egyes lehetséges \code{z}-re le kellene futtatnunk a függvényt. Ha az elágazás feltétele teljesült tudjuk, hogy a program végére $x = \frac{z+10}{z} \wedge z \neq 0$, azaz \code{x} akkor lesz 2, ha \code{z} értéke 10. Ha az elágazás feltétele nem teljesült, akkor tudjuk, hogy a program végére $x = 2 \wedge z = 0$, azaz ha \code{z} értéke 0, a kimenet $2$ lesz. Tehát \code{z=0} és \code{z=10} esetén kaphatunk eredményként 2-t.
\end{megjegyzes}




\section{Tesztelés}\label{sec:teszteles}

\Fogalom{teszteles} alatt olyan tevékenységet értünk, amely során a rendszert (vagy egy futtatható modelljét) bizonyos meghatározott körülmények közt futtatjuk (vagy szimuláljuk), majd az eredményeket összehasonlítjuk az elvárásainkkal\footnote{Ennél a nagyon általános és gyakran használt fogalomnál kivételesen érdemes az eredeti, angol nyelvű, az IEEE által adott definíciót is elolvasni: ``[Testing is an] activity in which a system or component is executed under specified conditions, the results are observed or recorded, and an evaluation is made of some aspect of the system or component''. \cite{IEEE-24765}} \cite{IEEE-24765}. A tesztelés célja a vizsgált rendszer minőségének felmérése és/vagy javítása azáltal, hogy hibákat azonosítunk. 

Láthatjuk az első szembetűnő különbséget a tesztelés és a statikus ellenőrzés közt: utóbbi esetben a vizsgált rendszert nem futtatjuk, nem hajtjuk végre. Ugyanakkor attól, hogy a rendszert annak vizsgálata céljából végrehajtjuk, még nem beszélünk tesztelésről. Ahogyan a definíció is mutatja, meghatározott körülmények közt futtatjuk a rendszert, azaz nem ``próbálgatásról'' van szó, hanem a fejlesztés egy alaposan megtervezett részfolyamatáról.


\subsection{A tesztelés alapfogalmai}
Már a definíció alapján is látszik, hogy a teszteléshez nem elég önmagában a rendszer. Tesztek végrehajtásához legalább a következő komponensek szükségesek: 
\begin{itemize}
\item a \fogalom{tesztelendo-rendszer} (system under test, SUT), amelyet a teszt során futtatni fogunk;
\item a \fogalomragozva{tesztbemenet}{tesztbemenetek}, amelyek megadják a tesztelendő rendszer számára biztosítandó bemeneti adatokat; és
\item a \fogalom{tesztorakulum}, amely alapján a végrehajtott tesztről eldönthető annak eredménye.
\end{itemize}

Összefoglaló néven \fogalomragozva{teszteset}{tesztesetnek} hívjuk azon adatok összességét, amelyek egy adott teszt futtatásához és annak értékeléséhez szükségesek. Tehát a teszteset ``bemeneti értékek, végrehajtási előfeltételek, elvárt eredmények és végrehajtási utófeltételek halmaza, amelyeket egy konkrét célért vagy a tesztért fejlesztettek'' \cite{HTB-glossary,IEEE-24765}. Tesztesetek egy adott halmazát \fogalomragozva{tesztkeszlet}{tesztkészletnek} hívjuk.

A \fogalom{tesztfuttatas}, azaz egy vagy több teszteset végrehajtása után \cite{IEEE-24765} -- az orákulum segítségével -- megtudjuk a teszt eredményét, amely lehet sikeres (pass), sikertelen (fail) vagy hibás (error). Utóbbi esetben a tesztről nem tudjuk eldönteni, hogy sikeres-e vagy sem. Ilyen lehet például, ha a tesztrendszerben történt hiba, emiatt a SUT helyességéről nem tudunk nyilatkozni. 
Általában a teszt eredménye a kapott és a tesztesetben megfogalmazott elvárt kimenetek összehasonlításával kapható meg, de származhat egy referenciaimplementációval összehasonlításból, vagy ellenőrizhetünk implicit elvárásokat, például azt, hogy a kód nem dob kivételt.

A tesztelés általános elrendezése látható \aref{fig:teszteles-elrendezes-orakulum}.~ábrán.

\begin{figure}[h]
	\centering
	\input{modellek-ellenorzese/figures/teszteles-elrendezes-orakulum.pdf_tex}
	
	\caption{A tesztelés általános sémája}
	\label{fig:teszteles-elrendezes-orakulum}
\end{figure}

Legegyszerűbb esetben a teszteset közvetlenül tartalmazza az adott tesztbemenetekre elvárt kimeneteket (referenciakimeneteket), így az orákulum feladata mindössze a SUT kimenetének összevetése a tesztesetben leírt elvárt kimenetekkel, ahogyan az az alábbi ábrán is látható:

\begin{figure}[h]
	\centering
	\input{modellek-ellenorzese/figures/teszteles-elrendezes-elvart-kimenet.pdf_tex}
	
	\caption{A tesztelés menete ismert referenciakimenet esetén}
	\label{fig:teszteles-elrendezes-elvart-kimenet}
\end{figure}

\Aref{fig:teszteles-elrendezes-elvart-kimenet}.~ábrán vázolt elrendezésről van szó például akkor, ha állapotgépeket tesztelünk, és a teszteset tartalmazza a bemeneti eseménysort (tesztbemenetként) és az elvárt akciókat, eseményeket (elvárt kimenetként). 

Nincs feltétlen lehetőség azonban referenciakimenet megadására olyan teszteseteknél, amelyek deklaratív követelményeket ellenőriznek, vagy többféle kimenetet is megengednek. Ilyenkor speciális orákulum szükséges. Például egy prímtesztelő eljárással szembeni követelmény lehet, hogy amennyiben a bemeneten kapott szám összetett, akkor bizonyítékul a kimeneten be kell mutatni a bemenetként kapott szám egyik valódi osztóját. Ilyenkor a tesztorákulumnak többféle kimenetet is el kell fogadnia. Ehhez például megvizsgálhatja a bemenet oszthatóságát a SUT által adott kimenettel.


\subsection{A tesztek kategorizálása\kiegeszitoanyag}
Tesztelést a szoftverfejlesztési életciklus számos fázisában használhatunk. Attól függően, hogy a rendszer mekkora részét vizsgáljuk, különböző tesztelési módszereket különböztethetünk meg.

\begin{itemize}
\item \fogalomragozva{modulteszt}{Modultesztnek} (másként \fogalom{komponensteszt} vagy \fogalom{egysegteszt}) nevezzük azt a tesztet, amely csak egyes izolált komponenseket tesztelnek \cite{HTB-glossary}.
\item \fogalomragozva{integracios-teszt}{Integrációs tesztnek} nevezzük azt a tesztet, ``amelynek célja az integrált egységek közötti
interfészekben, illetve kölcsönhatásokban lévő hibák megtalálása''  \cite{HTB-glossary}.
\item \fogalomragozva{rendszerteszt}{Rendszertesztnek} hívjuk azt a tesztet, amelyben a teljes, integrált rendszert vizsgáljuk annak érdekében, hogy ellenőrizzük a követelményeknek való megfelelőséget \cite{HTB-glossary}.
\end{itemize}

Ezek a különféle tesztelési módszerek általában egymást követik a fejlesztési ciklusban: először az egyes modulok tesztelése történik meg, később a modulok integrációja után az integrációs tesztek, majd végül a rendszerteszt kerül elvégzésre.

Amennyiben módosítást végzünk a rendszerünkön, a korábbi tesztek eredményeit már nem fogadhatjuk el, hiszen a rendszer megváltozott. Ha ismerjük, hogy az egyes tesztesetek a rendszer mely részeit vizsgálják, elegendő azokat újrafuttatnunk, amelyek a megváltoztatott részt (is) vizsgálják. Az ilyen, változtatások utáni (szelektív) újratesztelést hívjuk \fogalomragozva{regresszios-teszt}{regressziós tesztnek}. Fontos megjegyezni, hogy ez a korábbiakhoz képest egy ortogonális kategória, és egységeket vagy teljes rendszert is vizsgálhatunk regressziós teszttel. 

\subsection{A tesztelés metrikái}
Ahogyan a fejezet elején írtuk, a tesztelés általános célja a vizsgált rendszer minőségének javítása hibák megtalálásán és javításán keresztül. Nyilvánvaló, hogy ez a végtelenségig nem folytatható, egy idő után az összes hibát megtaláljuk és kijavítjuk. Ennél az utópisztikus nézetnél kissé pragmatikusabban azt is mondhatjuk, hogy egy idő után a tesztelés folytatása nem célszerű (nem gazdaságos), mert a vizsgált rendszer minősége már ``elég jó''. % Igen, az utópisztikust tényleg így írják.
De honnan tudhatjuk, hogy elértük ezt a szintet?

Az egyik gyakran használt módszer a tesztkészlet fedésének mérése. A tesztfedettség alapötlete az, hogy a tesztkészlet egyik tesztesete sem látogat meg egy adott állapotot, akkor az az állapot biztosan nem lesz vizsgálva, így annak minőségéről következtetést nem vonhatunk le. Ugyanez elmondható például egy metódus hívásával kapcsolatban is.

Ha viszont a tesztkészletünk meglátogat minden állapotot vagy meghív minden metódust, elmondhatjuk, hogy mindent megvizsgáltunk? Sajnos korántsem. Például abból, hogy minden állapotot bejár egy tesztkészlet nem következik, hogy minden állapotátmenetet is érint. Attól, hogy egy tesztkészlet minden metódust meghív, nem feltétlenül érint minden utasítást. Látható, hogy számos fedettségi metrikát lehet bevezetni. Mi itt mindössze az alábbi három alapvető fedettségi metrikára szorítkozunk.
\begin{itemize}
	\item Egy állapotgépben az \fogalom{allapotfedettseg} (vagy állapotlefedettség) egy adott tesztkészlet által érintett (bejárt) állapotok és az összes állapotok arányát adja meg.
	\item Egy állapotgépben az \fogalom{atmenetfedettseg} (vagy átmenetlefedettség) egy adott tesztkészlet által érintett (bejárt) állapotátmenetek és az összes átmenetek arányát adja meg.
	\item Egy vezérlési folyamban (programban) az \fogalom{utasitasfedettseg} (vagy utasításlefedettség) egy adott tesztkészlet által érintett (bejárt) utasítások és az összes utasítások arányát adja meg.	
\end{itemize}

\begin{megjegyzes}
Példa. \textbf{TODO -- ábra, a többivel egységes formában. Vezérpélda?}
\end{megjegyzes}

A fenti áttekintésből az is látszik, hogy egy magas fedettségi arány csak szükséges, de nem elégséges feltétele a jó minőségű rendszer fejlesztésének. Gyakran ez a szám félrevezető is lehet, illetve rossz irányba viheti a teszttervezést.


\section{Tesztelés futásidőben (futásidejű verifikáció)}\label{sec:futasideju-verifikacio}
Ebben a fejezetben a futásidejű ``öntesztelés'' vagy monitorozás alapötletét mutatjuk be. Bizonyos esetekben kiemelkedően magas minőségi elvárásaink vannak a rendszerünkkel szemben (pl. biztonságkritikus alkalmazási területek). Más esetekben olyan külső komponenseket használunk, amelyek minőségéről nem tudunk alaposan meggyőződni (pl. egy lefordított, más által fejlesztett alkalmazást csak korlátozottan tudunk tesztelni). Ilyenkor az elvárásaink egy részét elhelyezzük magában a megvalósított rendszerben és folyamatosan vizsgáljuk.

A monitorozás általános elrendezését szemlélteti a következő ábra.

\begin{figure}[h]
	\centering
	\input{modellek-ellenorzese/figures/monitorozas-elrendezes.pdf_tex}
	
	\caption{A monitorozás általános elrendezése}
	\label{fig:monitorozas-elrendezes}
\end{figure}

A monitorozás két fő lépésből áll:
\begin{itemize}
\item \emph{bemenetek ellenőrzéséből}, amely során a bemeneti adatok megfelelőségét vizsgáljuk a definiált bemeneti invariánsok (előfeltételek) alapján, és/vagy
\item \emph{hihetőségvizsgálatból}, amely során a kimeneti adatok megfelelőségét vizsgáljuk a bemeneti adatok és a definiált kimeneti invariánsok (utófeltételek) alapján.
\end{itemize}

Egyes esetekben az invariánsok igen egyszerűek (például egy valós számok négyzetre emelést megvalósító függvény végén vizsgálhatjuk, hogy a kapott eredmény negatív-e; a negatív eredményt hibásnak minősítjük). Ilyenkor tipikusan az implementáció is három részre tagolódik, követve \aref{fig:monitorozas-elrendezes}.~ábrán látható elrendezést:
\begin{itemize}
\item Először az \emph{előfeltételt} vizsgáljuk. Ha ez nem teljesül, kivételről beszélünk. Ez egy normálistól eltérő, váratlan helyzet, aminek a kezelését máshol valósítjuk meg (ilyen körülmények közt az implementációnk helyességét nem követeljük meg). Ha az előfeltétel nem teljesül, annak oka a rendszer hibás használata (nem megfelelő bemeneti adatokat kapott).
\item Amennyiben az előfeltétel teljesült, megtörténik az érdemi logika \emph{végrehajtása}.
\item A végrehajtás után az utófeltétel vizsgálatára kerül sor. Amennyiben az utófeltétel nem teljesül, olyan hibás állapotba került a rendszer, amely kezelésére nincs felkészítve. Ennek oka lehet a hibás implementáció vagy futásidejű hiba.
\end{itemize}

\begin{megjegyzes}
Az alábbi példakód egy másodfokú egyenlet gyökeit számolja ki:

\begin{lstlisting}[language=C++]
void Roots(float a, b, c, float &x1, &x2) {
    float d = sqrt(b*b-4*a*c);

    x1 = (-b+d)/(2*a);
    x2 = (-b-d)/(2*a);
}
\end{lstlisting}

Tudhatjuk, hogy ez a kód nem működik helyesen minden esetben. Feltételezzük, hogy a diszkrimináns ($D=b^2-4\cdot a\cdot c$) nemnegatív, különben a gyökvonást negatív számon végezzük el. Tudjuk azt is, hogy a kiszámított $x_1$ és $x_2$ értékeknek zérushelyeknek kell lennie, azaz elvárt, hogy $ax_1^2 + bx_1 + c = 0$ és $ax_2^2 + bx_2 + c = 0$. Ezekkel az elő- és utófeltételekkel kiegészíthetjük az implementációt is az alábbiak szerint:

\lstset{literate=*{[Ho]}{ő}{1}}
\lstset{literate=*{[guilleft]}{oooooo}{1}{[guilright]}{\guillemotright{}}{1},}

\lstset{
    inputencoding=utf8,
    extendedchars=true,
    literate={ő}{{\Ho}}1 {ã}{{\~a}}1 {é}{{\'e}}1,
}

\begin{lstlisting}[language=C++]
void RootsMonitor(float a, b, c, float &x1, &x2) {
    // elofeltétel
    float D = b*b-4*a*c;
    if (D < 0)
        throw "Invalid input!";

    // végrehajtás
    Roots(a, b, c, x1, x2);

    // utófeltétel
    assert(a*x1*x1+b*x1+c == 0 && a*x2*x2+b*x2+c == 0);
}
\end{lstlisting}
\end{megjegyzes}

Monitorozást nem csak ilyen egyszerű esetekben lehet használni, összetett monitorok is elképzelhetők. Például állapotgépek esetén készíthetünk egy monitor régiót, ami a rendszer megvalósításával párhuzamosan fut és detektálja a hibás vagy tiltott állapotokat, akciókat.







\section{Formális verifikáció\kiegeszitoanyag}\label{sec:formalis-verifikacio}
\Fogalom{formalis-verifikacio} alatt olyan módszereket értünk, amelyek segítségével adott modellek vagy programok helyességét matematikailag precíz eszközökkel vizsgálhatjuk. Három fontosabb formális verifikációs módszert (családot) említünk meg:
\begin{itemize}
\item Modellellenőrzés;
\item Automatikus helyességbizonyítás, amely során axiómarendszerek alapján tételbizonyítás segítségével próbáljuk a helyességet belátni;
\item Konformanciavizsgálat, amely során adott modellek közt bizonyos konformanciarelációk teljesülését vizsgáljuk, így beláthatjuk, hogy különböző modellek viselkedése megegyező vagy eltérő az adott relációk szerint.
\end{itemize}

Jelen jegyzetben kitekintésként a modellellenőrzést mutatjuk be röviden. Bővebben a formális verifikációról a Formális Módszerek (BMEVIMIM100) MSc tárgy\footnote{\url{https://inf.mit.bme.hu/edu/courses/form}} keretei közt szólunk.

\paragraph{Modellellenőrzés.}
A \fogalom{modellellenorzes} egy olyan módszer, amelynek során egy adott modellen vagy implementáción egy követelmény teljesülését vizsgáljuk. A modellellenőrzés egyik előnye, hogy amennyiben a követelmény nem teljesül, lehetséges egy ellenpéldát adni. Az \fogalom{ellenpelda} egy olyan futási szekvencia, amely megmutatja, hogyan lehetséges a vizsgált követelményt megsérteni. Ez nagyban segíthet a hibás működés okának meghatározásában.

A modellellenőrzés -- a teszteléssel szemben -- egy \emph{teljes} módszer, azaz az adott modell vizsgálata kimerítő. Ennek következtében lehetőség van a helyes működés bizonyítására is, míg ez teszteléssel nem lehetséges. Ugyanakkor a modellellenőrzés igen nagy számítási igényű, ezért használhatósága korlátozott.