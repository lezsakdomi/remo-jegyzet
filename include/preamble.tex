\usepackage[a4paper,top=2cm,bottom=2cm,left=2cm,right=2cm]{geometry}
%add the twoside option for printig

% http://tex.stackexchange.com/questions/48753/obtaining-the-default-section-spacing-into-the-titlespacing-parameters
\usepackage[raggedright]{titlesec}
%\titlespacing*{\chapter} {0pt}{50pt}{40pt}
\titlespacing*{\chapter} {0pt}{36pt}{0pt}
%\titlespacing*{\section} {0pt}{3.5ex plus 1ex minus .2ex}{2.3ex plus .2ex}
\titlespacing*{\section} {0pt}{3ex plus 0.5ex minus 1ex}{1ex plus .5ex minus .5ex}
%\titlespacing*{\subsection} {0pt}{3.25ex plus 1ex minus .2ex}{1.5ex plus .2ex}
\titlespacing*{\subsection} {0pt}{2.4ex plus 0.5ex minus 1ex}{.8ex plus .5ex minus .5ex}
%\titlespacing*{\subsubsection}{0pt}{3.25ex plus 1ex minus .2ex}{1.5ex plus .2ex}
\titlespacing*{\subsubsection}{0pt}{2ex plus 0.5ex minus 1ex}{.6ex plus .5ex minus .5ex}
%\titlespacing*{\paragraph} {0pt}{3.25ex plus 1ex minus .2ex}{1em}
%\titlespacing*{\subparagraph} {\parindent}{3.25ex plus 1ex minus .2ex}{1em}

\usepackage[hang]{caption}
%\captionsetup{belowskip=0ex,aboveskip=0.5ex}

\usepackage{wrapfig}
\usepackage{changes}
\usepackage{fontspec}
\usepackage{setspace}
\usepackage{amssymb,amsmath,amsthm}
\usepackage{graphicx,xcolor}
\usepackage{listings}
\usepackage{hyperref}
\usepackage[magyar]{babel}
\usepackage{multicol}
\usepackage{multirow}
\usepackage{tabularx}
\usepackage{colortbl}
\usepackage{booktabs}
\usepackage{placeins}
\usepackage{algorithm2e}
\usepackage{array}
\usepackage{adjustbox}
\usepackage{subcaption}
\usepackage{float}
\usepackage[normalem]{ulem}
\usepackage{enumitem}
%\usepackage{wasysym}
\usepackage{hhline}
\usepackage{tabto}
\usepackage{needspace}
\usepackage{lscape}
%\IfFileExists{./use-xindy}{
	\usepackage[xindy,acronym,nopostdot,shortcuts,nomain,toc]{glossaries}
%}{
%	\usepackage[acronym,nopostdot,shortcuts,nomain,toc]{glossaries}
%}
\usepackage{skull}
\usepackage{fancyhdr}
\usepackage{todonotes}
\usepackage{etoolbox}

\usepackage{xcolor}
\usepackage{mdframed}
\usepackage{glossary-mcols}
\renewcommand*{\glsmcols}{2}
\setglossarystyle{mcolindex}

% Valtozasok kovetesehez:
% \added[id=szarnyas]{szoveg}
% \deleted[id=szarnyas]{szoveg}
% \replaced[id=szarnyas,remark={megjegyzest is irhatsz mindharom parancsba}]{new}{old}
\definechangesauthor[name={Gabor Szarnyas}, color=purple]{szarnyas}
\definechangesauthor[name={Vince Molnar}, color=brown]{molnar}
\definechangesauthor[name={Gabor Bergmann}, color=blue{bergmann}
\definechangesauthor[name={Andras Voros}, color=red]{voros}
\definechangesauthor[name={Daniel Darvas}, color=orange]{darvas}
\definechangesauthor[name={Laszlo Gonczy}, color=orange]{gonczy}
\definechangesauthor[name={Agnes Salanki}, color=orange]{salanki}
\definechangesauthor[name={Csaba Debreceni}, color=orange]{debreceni}
\definechangesauthor[name={Marton Bur}, color=orange]{bur}
\definechangesauthor[name={Tamas Toth}, color=orange]{toth}

\setlist{itemsep=0.2ex,topsep=0ex,parsep=0.2ex}

% nagyobb sorköz
\setstretch{1.15}
\singlespacing

\sloppy % margón túllógó sorok tiltása
%\brokenpenalty10000\relax % oldalhatáron átnyúló elválasztás elkerülése
%\widowpenalty10000 \clubpenalty10000 % a fattyú- és árvasorok elkerülése
%\def\hyph{-\penalty0\hskip0pt\relax} % kötőjeles szavak elválasztásának engedélyezése

% http://tex.stackexchange.com/questions/4152/how-do-i-prevent-widow-orphan-lines
\clubpenalty=9996
\widowpenalty=9999
\brokenpenalty=4991
\predisplaypenalty=10000
\postdisplaypenalty=1549
\displaywidowpenalty=1602

% http://tex.stackexchange.com/questions/2644/how-to-prevent-a-page-break-before-an-itemize-list
\makeatletter
\@beginparpenalty=7000
 
\newcommand\mynobreakpar{\par\nobreak\@afterheading} 
\makeatother

%\usepackage[hang]{footmisc}
%\setlength{\footnotemargin}{3ex}

%\setcounter{chapter}{0}
%\setcounter{secnumdepth}{5}

\graphicspath{ {./figures/}, {./sources/} }

\pagestyle{plain}
\setlength{\parindent}{0pt} % angol nyelvű dokumentumokban jellemző
\setlength{\parskip}{0.5em}    % angol nyelvű dokumentumokban jellemző
%\nofrenchspacing
%\setlength{\parindent}{2em} % magyar nyelvű dokumentumokban jellemző
%\setlength{\parskip}{0em}   % magyar nyelvű dokumentumokban jellemző
\frenchspacing



\setlist[enumerate,2]{label=\alph*),before=\normalfont,after=\normalfont}
\setlist[itemize,1]{label={--}}
\setlist[itemize,2]{label={$\circ$}}%\textperiodcentered}}

\setlength{\itemsep}{0em}

%\usepackage{tocloft}
%\setlength{\cftchapnumwidth}{1.8em}
%\setlength{\cftsecnumwidth}{2.8em}
%\setlength{\cftsecindent}{1.8em}
%\setlength{\cftsubsecindent}{4.6em}
	
\newcommand{\code}[1]{{\footnotesize\ttfamily #1}}

%\usepackage{soul}
\definecolor{lightgray}{gray}{0.95}


\usepackage{listings}
%\usepackage{color}
\definecolor{listinggray}{rgb}{0.9,0.9,0.9}
\definecolor{keywordcolor}{rgb}{0.5,0,0.1}
\definecolor{commentcolor}{rgb}{0,0.3,0.1}
\definecolor{stringcolor}{rgb}{0,0,1}
\definecolor{lightgray}{rgb}{0.985,0.985,0.985}
\lstset{
	basicstyle=\footnotesize\ttfamily, % print whole listing small
	keywordstyle=\bfseries, % bold black keywords
	identifierstyle=, 					% nothing happens
	commentstyle=\color{gray},
	stringstyle=\footnotesize, 			
	showstringspaces=false,     % no special string spaces
	aboveskip=0.5em,
	belowskip=0.5em,
	columns=flexible,
	backgroundcolor=\color{lightgray},
	escapeinside={(*@}{@*)},
	inputencoding=ansinew,
	extendedchars=true,
	literate={á}{{\'a}}1 {é}{{\'e}}1 {í}{{\'i}}1 {ó}{{\'o}}1 {ú}{{\'u}}1
	{Á}{{\'A}}1 {É}{{\'E}}1 {Í}{{\'I}}1 {Ó}{{\'O}}1 {Ú}{{\'I}}1
	{ö}{{\"o}}1 {ő}{{\H{o}}}1 {ü}{{\"u}}1 {ű}{{\H{u}}}1
	{Ö}{{\"O}}1 {Ő}{{\H{O}}}1 {Ü}{{\"U}}1 {Ű}{{\H{U}}}1,
}
\lstset{
	xleftmargin=1em,
	framexleftmargin=1em,
	framextopmargin=1em,
	framexbottommargin=1em,
	frame=tb,
	framerule=0pt
	}
\lstset{escapeinside={(*@}{@*)}}
\def\lstlistingname{Forrás}	

\newcommand{\alignListing}{\lstset{xleftmargin=5mm, framexleftmargin=5mm, numbers=left}}
	% forráskódok stílusai
\newcommand{\remofigscale}[3]{
	\begin{figure}[htb]
		\centering
		\includegraphics[scale=#3,center]{#1}
		\caption{#2}
		\label{fig:#1}
	\end{figure}}
	
\newcommand{\remofigscaleframed}[3]{
	\begin{figure}[H]
		\centering
		\includegraphics[scale=#3,center]{#1}
		\caption{#2}
		\label{fig:#1}
	\end{figure}}


\newcommand{\kieg}{*}
	
\newcommand{\halalfejes}{$\skull$}
\newcommand{\allapot}[1]{\textsf{#1}}
\newcommand{\esemeny}[1]{\textsf{#1}}
\newcommand{\regio}[1]{\textsf{#1}}
\newcommand{\valtozo}[1]{\textsf{#1}}
\newcommand{\tipus}[1]{\textsf{#1}}

\newcommand{\cpp}{C\texttt{++}\xspace}

\newcommand{\targy}[1]{\emph{#1}\@\xspace}
\newcommand{\progegy}{\targy{A programozás alapjai 1.}}
\newcommand{\progketto}{\targy{A programozás alapjai 2.}}
\newcommand{\progharom}{\targy{A programozás alapjai 3.}}
\newcommand{\szofttech}{\targy{Szoftvertechnológia}}
\newcommand{\sznikak}{\targy{Szoftvertechnikák}}
\newcommand{\adatb}{\targy{Adatbázisok}}
\newcommand{\bsz}{\targy{Bevezetés a számításelméletbe}}
\newcommand{\bszegy}{\targy{Bevezetés a számításelméletbe 1.}}
\newcommand{\bszketto}{\targy{Bevezetés a számításelméletbe 2.}}
\newcommand{\adatblab}{\targy{Adatbázisok laboratórium}}
\newcommand{\algel}{\targy{Algoritmuselmélet}}
\newcommand{\halok}{\targy{Kommunikációs hálózatok}}
\newcommand{\halokegy}{\targy{Kommunikációs hálózatok 1.}}
\newcommand{\halokketto}{\targy{Kommunikációs hálózatok 2.}}
\newcommand{\mestersegesintelligencia}{\targy{Mesterséges intelligencia}}
\newcommand{\rendszertervezes}{\targy{Informatikai rendszerek tervezése}}
\newcommand{\mdsd}{\targy{Modellvezérelt rendszertervezés}}
\newcommand{\szore}{\targy{Szoftver- és rendszerellenőrzés}}
\newcommand{\eat}{\targy{Eclipse alapú technológiák}}
\newcommand{\form}{\targy{Formális módszerek}}

%\usepackage{tcolorbox}
\tcbuselibrary{breakable,skins}
\usetikzlibrary{decorations}

\newtcolorbox{definicioTCB}[1][]
{
    enhanced jigsaw,
%    show bounding box,
    parbox=false,
    colframe=black,
    opacityframe=1.0,
    opacityback=0.0,
    boxrule=0.1pt,
    leftrule=5pt,
    grow to left by=20pt,
    grow to right by=15pt,
    right=15pt,
    left=12pt,
    top=-0.6\parskip,
    enlarge bottom by=\parskip,
%    bottom=3\parskip,
    breakable,
    #1,
}

\newtcolorbox{megjegyzesTCB}[1][]
{
    enhanced jigsaw,
    parbox=false,
    colframe=black,
    opacityframe=1.0,
    opacityback=0.0,
    boxrule=0.1pt,
    grow to left by=15pt,
    grow to right by=15pt,
    right=15pt,
    left=15pt,
    top=-0.6\parskip,
    enlarge bottom by=\parskip,
    breakable,
    #1,
}

\newtcolorbox{tetelTCB}[1][]
{
	enhanced jigsaw,
	parbox=false,
	colframe=black,
	opacityframe=1.0,
	opacityback=0.0,
	boxrule=0.1pt,
	toprule=3pt,
	bottomrule=3pt,
	grow to left by=15pt,
	grow to right by=15pt,
	right=15pt,
	left=12pt,
    top=0pt, %\parskip,
    enlarge bottom by=\parskip,
	breakable,
	toprule at break=0pt,
	bottomrule at break=0pt,
	#1,
}

\makeatletter
\newtheoremstyle
	{dbdef} % name
	{0pt} % spaceabove
	{0pt} % spacebelow
	{} % bodyfont
	{0pt} % indentamt
	{} % headfont
	{.} % headpunct
	{5pt plus 1pt minus 1pt} % headspace
	{\textbf{\thmname{#1}}{\@ifnotempty{#3}{ }}\thmnote{{-- \emph{#3}}}%\textbf{.}\hspace{1em} % Definíció, szóköz, ha kell (definiált fogalom).
	}
\newtheoremstyle
	{dbmegj} % name
	{0pt} % spaceabove
	{0pt} % spacebelow
	{} % bodyfont
	{0pt} % indentamt
	{} % headfont
	{.} % headpunct
	{5pt plus 1pt minus 1pt} % headspace
	{\textbf{\thmname{#1}}{\@ifnotempty{#3}{ }}\thmnote{{\emph{(#3)}}}%\textbf{.}\hspace{1em} % Definíció, szóköz, ha kell (definiált fogalom).
	}
\newtheoremstyle
	{dbpl} % name
	{\parskip} %\parskip} % spaceabove
	{1.5\parskip} % spacebelow
	{} %\sffamily} % bodyfont
	{0pt} % indentamt
	{} % headfont
	{.} % headpunct
	{5pt plus 1pt minus 1pt} % headspace
	{\textbf{\thmname{#1}}{\@ifnotempty{#3}{ }}\thmnote{{\emph{(#3)}}}%\textbf{.}\hspace{1em} % Definíció, szóköz, ha kell (definiált fogalom).
	}
\newtheoremstyle
	{dbbiz} % name
	{1.5\parskip} % spaceabove
	{\parskip} % spacebelow
	{} % bodyfont
	{0pt} % indentamt
	{} % headfont
	{.} % headpunct
	{5pt plus 1pt minus 1pt} % headspace
	{\textbf{\thmname{#1}}{\@ifnotempty{#3}{ }}\thmnote{{-- \emph{#3}}}%\textbf{.}\hspace{1em} % Definíció, szóköz, ha kell (definiált fogalom).
	}
\makeatother

\theoremstyle{dbdef}
\newtheorem{definicioT}{Definíció}
\newtheorem{algoritmusT}{Algoritmus}
\newtheorem{tetelT}{Tétel}

\theoremstyle{dbbiz}
\newtheorem{bizonyitas}{Bizonyítás}

\theoremstyle{dbmegj}
\newtheorem{megjegyzesT}{Megjegyzés}
\newtheorem{megjegyzesekT}{Megjegyzések}

\theoremstyle{dbpl}
\newtheorem{pelda}{Példa}

\newenvironment{definicio}[1][]%
	{\begin{definicioTCB}\begin{definicioT}[#1]}%
	{\end{definicioT}\end{definicioTCB}}

\newenvironment{algoritmus}[1][]%
{\begin{definicioTCB}\begin{algoritmusT}[#1]}%
		{\end{algoritmusT}\end{definicioTCB}}

\newenvironment{tetel}[1][]%
	{\begin{tetelTCB}\begin{tetelT}[#1]}%
	{\end{tetelT}\end{tetelTCB}}

\newenvironment{megjegyzes}[1][]%
	{\begin{megjegyzesTCB}\begin{megjegyzesT}[#1]}%
	{\end{megjegyzesT}\end{megjegyzesTCB}}

\newenvironment{megjegyzesek}[1][]%
	{\begin{megjegyzesTCB}\begin{megjegyzesekT}[#1]}%
	{\end{megjegyzesekT}\end{megjegyzesTCB}}

\newenvironment{aprobetu}[1]{\begin{footnotesize}}{\end{footnotesize}}
	% definíciók, tételek stílusai -- előtte szükséges tcolorbox import
\usepackage{tikz}
\usetikzlibrary{automata,arrows,positioning,trees,fit}
\tikzset{
	%Define standard arrow tip
	>=stealth',
	%Define style for boxes
	punkt/.style={
		rectangle,
		rounded corners,
		draw=black,
		%		text width=6.5em,
		minimum height=2.5em,
		text centered},
	% Define arrow style
	pil/.style={
		->,>=stealth',shorten >=1pt,
		text width = 2.5cm,
		align=center,},
	%font=\clearsanslight,
}
%--------------------------------------------------------------------------------------
%	Setup hyperref package
%--------------------------------------------------------------------------------------
\hypersetup{
    bookmarks=true,           % show bookmarks bar?
    unicode=false,            % non-Latin characters in Acrobat's bookmarks
    pdftitle={\title{}},      % title
    pdfauthor={\author{}},    % author
    pdfsubject={\title{}},    % subject of the document
    pdfcreator={\author{}},   % creator of the document
    pdfproducer={},           % producer of the document
    pdfkeywords={},
	                          % list of keywords (separate then by comma)
    pdfnewwindow=true,        % links in new window
    colorlinks=false,         % false: boxed links; true: colored links
    %linkcolor=black,          % color of internal links
    %citecolor=black,          % color of links to bibliography
    %filecolor=black,          % color of file links
    %urlcolor=black,           % color of external links
    %pdfborderstyle={/S/U/W 1}
}

\newenvironment{megjegyzes}{\begin{mdframed}[backgroundcolor=gray!20] }{\end{mdframed}}
\newenvironment{tipp}{\begin{mdframed}[backgroundcolor=green!20] }{\end{mdframed}}
\newenvironment{feladat}{\begin{mdframed}[backgroundcolor=red!20] }{\end{mdframed}}
\newenvironment{definicio}{\begin{mdframed}[backgroundcolor=yellow!20] }{\end{mdframed}}

% \iflabeldef parancs
\newcommand{\iflabeldef}[3]{\makeatletter\ifcsdef{r@#1}{#2}{#3}\makeatother}
% Használata: \iflabeldef{sec:whatever}{Erről bővebben \aref{sec:whatever}.~fejezetben írunk.}{Erről bővebben a Whatever leírásban írunk.}

% listings beállításai
\definecolor{backgroundcolor}{HTML}{F5F5F5}
\definecolor{keywordcolor}{HTML}{295F94}
\definecolor{commentcolor}{HTML}{AD95AF}
\definecolor{stringcolor}{HTML}{317ECC}

\lstset{
	numbers=none,
	numberstyle=\footnotesize\ttfamily,
	stepnumber=1,
	numbersep=5pt,
	%
	basicstyle=\footnotesize\ttfamily,
	backgroundcolor=\color{backgroundcolor},
	keywordstyle=\color{keywordcolor}\bfseries,
	commentstyle=\color{commentcolor},
	stringstyle=\color{stringcolor},
	identifierstyle=, % nothing happens
	%
	showstringspaces=false, % no special string spaces
	aboveskip=3pt,
	belowskip=3pt,
	columns=flexible,
	keepspaces=true,
	breaklines=true,	
	frameround=tttt,
	captionpos=b,
	tabsize=2,
	frame=tb,
	framerule=0pt,
%	framexleftmargin=0.25em,
}


\loadglsentries{glossary-entries.tex}

\title{Rendszermodellezés}
\author{Hibatűrő Rendszerek Kutatócsoport}
\date{2016}
