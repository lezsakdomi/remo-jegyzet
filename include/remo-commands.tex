\newcommand{\remofigscale}[3]{
	\begin{figure}[htb]
		\centering
		\includegraphics[scale=#3]{#1}
		\caption{#2.}
		\label{fig:#1}
	\end{figure}}

\newcommand{\remofigscaleframed}[3]{
	\begin{center}
	\begin{samepage}
	\includegraphics[scale=#3]{#1}
	\captionof{figure}{#2.}
	\label{fig:#1}
	\end{samepage}
	\end{center}}
	
\newcommand{\halalfejes}{$\skull$}
\newcommand{\allapot}[1]{\textsf{#1}}
\newcommand{\esemeny}[1]{\textsf{#1}}
\newcommand{\regio}[1]{\textsf{#1}}
\newcommand{\valtozo}[1]{\textsf{#1}}
\newcommand{\tipus}[1]{\textsf{#1}}

\newcommand{\cpp}{C\texttt{++}\xspace}

\newcommand{\targy}[1]{\emph{#1}}
\newcommand{\progegy}{\targy{A programozás alapjai 1.\xspace}}
\newcommand{\progketto}{\targy{A programozás alapjai 2.\xspace}}
\newcommand{\progharom}{\targy{A programozás alapjai 3.\xspace}}
\newcommand{\szofttech}{\targy{Szoftvertechnológia\xspace}}
\newcommand{\sznikak}{\targy{Szoftvertechnikák}}
\newcommand{\adatb}{\targy{Adatbázisok\xspace}}
\newcommand{\bsz}{\targy{Bevezetés a számításelméletbe}}
\newcommand{\bszegy}{\targy{Bevezetés a számításelméletbe 1.}}
\newcommand{\bszketto}{\targy{Bevezetés a számításelméletbe 2.}}
\newcommand{\algel}{\targy{Algoritmuselmélet}}
