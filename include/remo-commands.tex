\newcommand{\remofigscale}[3]{
	\begin{figure}[htb]
		\centering
		\includegraphics[scale=#3]{#1}
		\caption{#2.}
		\label{fig:#1}
	\end{figure}}
	
\newcommand{\halalfejes}{$\skull$}
\newcommand{\allapot}[1]{\textsf{#1}}
\newcommand{\esemeny}[1]{\textsf{#1}}
\newcommand{\regio}[1]{\textsf{#1}}
\newcommand{\valtozo}[1]{\textsf{#1}}
\newcommand{\tipus}[1]{\textsf{#1}}



%\newcommand{\egyedhalmaznev}[1]{{\ttfamily #1}}
%\newcommand{\kapcsolat}[1]{{\ttfamily\itshape #1}}
%\newcommand{\kapcsolategyedek}[2]{{\ttfamily\itshape #1}\texttt{:} \egyedhalmaznev{#2}}
%\newcommand{\kulcs}[1]{{\ttfamily \uline{#1}}}
%\newcommand{\ekattributum}[1]{{\ttfamily #1}}
%\newcommand{\egyedhalmaz}[2]{{\ttfamily #1}\ekattributum{(#2)}}
%\newcommand{\idegenkulcs}[1]{{\ttfamily \uline{\uline{#1}}}}
%
%\mathchardef\breakingcomma\mathcode`\,
%{\catcode`,=\active
%	\gdef,{\breakingcomma\discretionary{}{}{}}
%}
%
%\newcommand{\nev}[1]{\mathit{#1}}
%\newcommand{\sema}[2]{#1(#2)}
%\newcommand{\semanev}[1]{\nev{#1}}
%
%% kapcsolódó linkek:
%% - http://tex.stackexchange.com/questions/19094/allowing-line-break-at-in-inline-math-mode-breaks-citations/
%% - http://tex.stackexchange.com/questions/1959/allowing-line-break-at-in-inline-math-mode
%\newcommand{\mathlist}[1]{\mathcode`\,=\string"8000 #1}
%
%\newcommand{\attr}[1]{\it \mathlist{#1}}
%\newcommand{\relacio}[1]{\mathit{#1}}
%\newcommand{\felbontas}[2]{#1(\mathlist{#2})}
%\newcommand{\semakulcs}[1]{\underline{\smash{\textit{#1}}}}
%\newcommand{\semaidegenkulcs}[1]{\underline{\underline{\smash{\textit{#1}}}}}
%
%\newcommand{\ureshalmaz}{\emptyset}
%\newcommand{\unio}{\cup}
%\newcommand{\metszet}{\cap}
%\newcommand{\minusz}{\setminus}
%\newcommand{\join}{\Join}
%\newcommand{\vagy}{\vee}
%\newcommand{\es}{\wedge}
%\newcommand{\nem}{\neg}
%
%% kalkulus
%% kvantor: kvantor, valtozo, predikatum
%\newcommand{\kvantor}[2]{\left(#1 #2\right) \!} % \! hogy surubb legyen a kifejezes
%\newcommand{\letezik}[2]{\kvantor{\exists}{#1}}
%\newcommand{\minden}[2]{\kvantor{\forall}{#1}}
%\newcommand{\sorv}[2]{\mathit{#1}^{(#2)}}
%\newcommand{\krel}[4]{\relacio{#1}^{(#2)}\left( \sorv{#3}{#4} \right)}
%\newcommand{\kalkulus}[2]{\left\{#1 \Big| #2\right\}}
%
%% relacios tervezes
%% project-join mapping
%\newcommand{\pjm}[2]{m_{#1} \left(#2\right)}
%% $a \implikacio b$ jelentése: "a" implikálja "b"-t
%\newcommand{\implikacio}{\Rightarrow}
%% $X \ff Y$ jelentése: X-től funkcionálisan függ Y
%\newcommand{\ff}{\rightarrow}
%% $X \ff Y$ jelentése: X-től nem függ funkcionálisan Y
%\newcommand{\nff}{\not\rightarrow}
%% $X \tf Y$ jelentése: X-től többértékűen függ Y
%\newcommand{\tf}{\twoheadrightarrow}
%
%\newcommand{\lezart}[1]{\left(#1\right)^+}
%\newcommand{\lezartf}[2]{#1^+\left(#2\right)}
%
%\renewcommand{\models}{\vDash}
%\newcommand{\yields}{\vdash}
%
%% példatár
%\newcommand{\reszletesmegoldas}{$\CIRCLE$\xspace}
%\newcommand{\csakvegeredmeny}{$\LEFTcircle$\xspace}
%\newcommand{\nincsmegoldas}{$\Circle$\xspace}
%
%\newcommand{\exercisefigure}[1]{
%\begin{figure}[H] 
%	\centering
%	\includegraphics[scale=0.65]{figures/#1}
%\end{figure}}
