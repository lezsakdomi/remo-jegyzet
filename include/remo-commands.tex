% képek
\newcommand{\remofigscale}[3]{
	\begin{figure}[htb]
		\centering
		\includegraphics[scale=#3,center]{#1}
		\caption{#2}
		\label{fig:#1}
	\end{figure}}

\newcommand{\remofigscalefixed}[3]{
	\begin{figure}[H]
		\centering
		\includegraphics[scale=#3,center]{#1}
		\caption{#2}
		\label{fig:#1}
	\end{figure}}

\newcommand{\simplefigscale}[2]{
	\centering
	\includegraphics[scale=#2,center]{#1}}

% környezetek
\definecolor{SpringGreen}{HTML}{C6DC67}
\newenvironment{tipp}{\begin{mdframed}[backgroundcolor=SpringGreen] \textbf{Tipp.}}{\end{mdframed}}

\newenvironment{feladat}{\begin{mdframed}[backgroundcolor=blue!20] \textbf{Feladat.}}{\end{mdframed}}

\newenvironment{definicio}{\begin{minipage}{\textwidth}\begin{mdframed}[backgroundcolor=white, linecolor=black] \textbf{Definíció.}}{\end{mdframed}\end{minipage}}

\newenvironment{kisdefinicio}{\begin{mdframed}[backgroundcolor=white, linecolor=gray] \footnotesize \textbf{Formális definíció ($*$).}}{\end{mdframed}}

\newenvironment{kisdefiniciok}{\begin{mdframed}[backgroundcolor=white, linecolor=gray] \footnotesize \textbf{Formális definíciók ($*$).}}{\end{mdframed}}

\newenvironment{pelda}{\begin{minipage}{\textwidth}\begin{mdframed}[backgroundcolor=white,linecolor=orange] \textbf{Példa.}}{\end{mdframed}\end{minipage}}

\newenvironment{figyelmeztetes}{\begin{minipage}{\textwidth}\begin{mdframed}[backgroundcolor=white,linecolor=red] \textbf{Figyelem!}}{\end{mdframed}\end{minipage}}

\newenvironment{megjegyzes}{\footnotesize \textbf{Megjegyzés.}}{}


\newcommand{\kieg}{\textsuperscript{$\ast$}}
\newcommand{\kiegeszitoanyag}{\textsuperscript{($\ast$)}} %lábjegyzetek, "bónusz bekezdések" végére  %

\newcommand{\halalfejes}{$\skull$}
\newcommand{\allapot}[1]{\textsf{#1}}
\newcommand{\esemeny}[1]{\textsf{#1}}
\newcommand{\regio}[1]{\textsf{#1}}
\newcommand{\valtozo}[1]{\textsf{#1}}
\newcommand{\tipus}[1]{\textsf{#1}}

\newcommand{\cpp}{C\texttt{++}\xspace}

\newcommand{\targy}[1]{\emph{#1}\@\xspace}
\newcommand{\progegy}{\targy{A programozás alapjai 1.}}
\newcommand{\progketto}{\targy{A programozás alapjai 2.}}
\newcommand{\progharom}{\targy{A programozás alapjai 3.}}
\newcommand{\szofttech}{\targy{Szoftvertechnológia}}
\newcommand{\sznikak}{\targy{Szoftvertechnikák}}
\newcommand{\adatb}{\targy{Adatbázisok}}
\newcommand{\bsz}{\targy{Bevezetés a számításelméletbe}}
\newcommand{\bszegy}{\targy{Bevezetés a számításelméletbe 1.}}
\newcommand{\bszketto}{\targy{Bevezetés a számításelméletbe 2.}}
\newcommand{\adatblab}{\targy{Adatbázisok laboratórium}}
\newcommand{\algel}{\targy{Algoritmuselmélet}}
\newcommand{\halok}{\targy{Kommunikációs hálózatok}}
\newcommand{\halokegy}{\targy{Kommunikációs hálózatok 1.}}
\newcommand{\halokketto}{\targy{Kommunikációs hálózatok 2.}}
\newcommand{\mestersegesintelligencia}{\targy{Mesterséges intelligencia}}
\newcommand{\rendszertervezes}{\targy{Informatikai rendszerek tervezése}}
\newcommand{\mdsd}{\targy{Modellvezérelt rendszertervezés}}
\newcommand{\szore}{\targy{Szoftver- és rendszerellenőrzés}}
\newcommand{\eat}{\targy{Eclipse alapú technológiák}}
\newcommand{\form}{\targy{Formális módszerek}}
\newcommand{\malo}{\targy{Matematikai logika}}
\newcommand{\nyau}{\targy{Nyelvek és automaták}}
\newcommand{\deklapo}{\targy{Deklaratív programozás}}
\newcommand{\rendszerelmelet}{Rendszerelmélet}
\newcommand{\szgarch}{Számítógép}
