\newcommand{\remofigscale}[3]{
	\begin{figure}[htb]
		\centering
		\includegraphics[scale=#3]{#1}
		\caption{#2.}
		\label{fig:#1}
	\end{figure}}
	
\newcommand{\remofigscaleframed}[3]{
	\begin{center}
		\begin{samepage}
			\includegraphics[scale=#3]{#1}
			\captionof{figure}{#2.}
			\label{fig:#1}
		\end{samepage}
	\end{center}}
	
% for fbox around remofigscaleframedwhite
\newcommand{\remofigscaleframedwhite}[3]{
	\begin{center}
		\begin{samepage}
			\setlength{\fboxsep}{0pt}\fbox{\setlength{\fboxsep}{15pt}\colorbox{white}{\includegraphics[scale=#3]{#1}}}
			\captionof{figure}{#2.}
			\label{fig:#1}
		\end{samepage}
	\end{center}}
	
\newcommand{\halalfejes}{$\skull$}
\newcommand{\allapot}[1]{\textsf{#1}}
\newcommand{\esemeny}[1]{\textsf{#1}}
\newcommand{\regio}[1]{\textsf{#1}}
\newcommand{\valtozo}[1]{\textsf{#1}}
\newcommand{\tipus}[1]{\textsf{#1}}

\newcommand{\cpp}{C\texttt{++}\xspace}

\newcommand{\targy}[1]{\emph{#1}\@\xspace}
\newcommand{\progegy}{\targy{A programozás alapjai 1.}}
\newcommand{\progketto}{\targy{A programozás alapjai 2.}}
\newcommand{\progharom}{\targy{A programozás alapjai 3.}}
\newcommand{\szofttech}{\targy{Szoftvertechnológia}}
\newcommand{\sznikak}{\targy{Szoftvertechnikák}}
\newcommand{\adatb}{\targy{Adatbázisok}}
\newcommand{\bsz}{\targy{Bevezetés a számításelméletbe}}
\newcommand{\bszegy}{\targy{Bevezetés a számításelméletbe 1.}}
\newcommand{\bszketto}{\targy{Bevezetés a számításelméletbe 2.}}
\newcommand{\adatblab}{\targy{Adatbázisok laboratórium}}
\newcommand{\algel}{\targy{Algoritmuselmélet}}
\newcommand{\halok}{\targy{Kommunikációs hálózatok}}
\newcommand{\halokegy}{\targy{Kommunikációs hálózatok 1.}}
\newcommand{\halokketto}{\targy{Kommunikációs hálózatok 2.}}
\newcommand{\mestersegesintelligencia}{\targy{Mesterséges intelligencia}}
\newcommand{\rendszertervezes}{\targy{Informatikai rendszerek tervezése}}
\newcommand{\mdsd}{\targy{Modellvezérelt rendszertervezés}}
\newcommand{\szore}{\targy{Szoftver- és rendszerellenőrzés}}
\newcommand{\eat}{\targy{Eclipse alapú technológiák}}
