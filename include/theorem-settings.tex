\usepackage{tcolorbox}
\tcbuselibrary{breakable,skins}
\usetikzlibrary{decorations}

\newtcolorbox{definicioTCB}[1][]
{
    enhanced jigsaw,
%    show bounding box,
    parbox=false,
    colframe=black,
    opacityframe=1.0,
    opacityback=0.0,
    boxrule=0.1pt,
    leftrule=5pt,
    grow to left by=20pt,
    grow to right by=15pt,
    right=15pt,
    left=12pt,
    top=-0.6\parskip,
    enlarge bottom by=\parskip,
%    bottom=3\parskip,
    breakable,
    #1,
}

\newtcolorbox{megjegyzesTCB}[1][]
{
    enhanced jigsaw,
    parbox=false,
    colframe=black,
    opacityframe=1.0,
    opacityback=0.0,
    boxrule=0.1pt,
    grow to left by=15pt,
    grow to right by=15pt,
    right=15pt,
    left=15pt,
    top=-0.6\parskip,
    enlarge bottom by=\parskip,
    breakable,
    #1,
}

\newtcolorbox{tetelTCB}[1][]
{
	enhanced jigsaw,
	parbox=false,
	colframe=black,
	opacityframe=1.0,
	opacityback=0.0,
	boxrule=0.1pt,
	toprule=3pt,
	bottomrule=3pt,
	grow to left by=15pt,
	grow to right by=15pt,
	right=15pt,
	left=12pt,
    top=0pt, %\parskip,
    enlarge bottom by=\parskip,
	breakable,
	toprule at break=0pt,
	bottomrule at break=0pt,
	#1,
}

\makeatletter
\newtheoremstyle
	{dbdef} % name
	{0pt} % spaceabove
	{0pt} % spacebelow
	{} % bodyfont
	{0pt} % indentamt
	{} % headfont
	{.} % headpunct
	{5pt plus 1pt minus 1pt} % headspace
	{\textbf{\thmname{#1}}{\@ifnotempty{#3}{ }}\thmnote{{-- \emph{#3}}}%\textbf{.}\hspace{1em} % Definíció, szóköz, ha kell (definiált fogalom).
	}
\newtheoremstyle
	{dbmegj} % name
	{0pt} % spaceabove
	{0pt} % spacebelow
	{} % bodyfont
	{0pt} % indentamt
	{} % headfont
	{.} % headpunct
	{5pt plus 1pt minus 1pt} % headspace
	{\textbf{\thmname{#1}}{\@ifnotempty{#3}{ }}\thmnote{{\emph{(#3)}}}%\textbf{.}\hspace{1em} % Definíció, szóköz, ha kell (definiált fogalom).
	}
\newtheoremstyle
	{dbpl} % name
	{\parskip} %\parskip} % spaceabove
	{1.5\parskip} % spacebelow
	{} %\sffamily} % bodyfont
	{0pt} % indentamt
	{} % headfont
	{.} % headpunct
	{5pt plus 1pt minus 1pt} % headspace
	{\textbf{\thmname{#1}}{\@ifnotempty{#3}{ }}\thmnote{{\emph{(#3)}}}%\textbf{.}\hspace{1em} % Definíció, szóköz, ha kell (definiált fogalom).
	}
\newtheoremstyle
	{dbbiz} % name
	{1.5\parskip} % spaceabove
	{\parskip} % spacebelow
	{} % bodyfont
	{0pt} % indentamt
	{} % headfont
	{.} % headpunct
	{5pt plus 1pt minus 1pt} % headspace
	{\textbf{\thmname{#1}}{\@ifnotempty{#3}{ }}\thmnote{{-- \emph{#3}}}%\textbf{.}\hspace{1em} % Definíció, szóköz, ha kell (definiált fogalom).
	}
\makeatother

\theoremstyle{dbdef}
\newtheorem{definicioT}{Definíció}
\newtheorem{algoritmusT}{Algoritmus}
\newtheorem{tetelT}{Tétel}

\theoremstyle{dbbiz}
\newtheorem{bizonyitas}{Bizonyítás}

\theoremstyle{dbmegj}
\newtheorem{megjegyzesT}{Megjegyzés}
\newtheorem{megjegyzesekT}{Megjegyzések}

\theoremstyle{dbpl}
\newtheorem{pelda}{Példa}

\newenvironment{definicio}[1][]%
	{\begin{definicioTCB}\begin{definicioT}[#1]}%
	{\end{definicioT}\end{definicioTCB}}

\newenvironment{algoritmus}[1][]%
{\begin{definicioTCB}\begin{algoritmusT}[#1]}%
		{\end{algoritmusT}\end{definicioTCB}}

\newenvironment{tetel}[1][]%
	{\begin{tetelTCB}\begin{tetelT}[#1]}%
	{\end{tetelT}\end{tetelTCB}}

\newenvironment{megjegyzes}[1][]%
	{\begin{megjegyzesTCB}\begin{megjegyzesT}[#1]}%
	{\end{megjegyzesT}\end{megjegyzesTCB}}

\newenvironment{megjegyzesek}[1][]%
	{\begin{megjegyzesTCB}\begin{megjegyzesekT}[#1]}%
	{\end{megjegyzesekT}\end{megjegyzesTCB}}

\newenvironment{aprobetu}[1]{\begin{footnotesize}}{\end{footnotesize}}
