\topic{Tartalmi sablon}

\section{Bevezető}

1-2 bekezdéses bevezető, ami pozicionálja a témakört.

\section{Motiváció}

Vezérpélda megfelelő aspektusának bemutatása. Milyen műszaki kihívások vannak? Erre alkalmas \fogalom{modell}, \fogalom{formalizmus} vagy \fogalom{modszertan} bevezetése.

\subsection{Egy tartalmi blokk}

\begin{mdframed}
	\begin{enumerate}
	\item Modellezési példa (diagram, formalizmus)
	
	\item Hogyan értelmezhető ez? Hogyan szimuláljuk?
	
	\item Formális bevezető. Pl. ez egy halmaz, ahol \ldots; ez egy függvény \ldots.
	\end{enumerate}		
\end{mdframed}

\subsection{Következő tartalmi blokk}

\ldots

\section{Plusz példák}

\ldots

\section{Csillagos anyagok}

\ldots

\section{Kapcsolódó témák}

Kapcsolódó anyagrészek/tárgyak (ha eddig nem kerültek említésre). Egy példa irodalomhivatkozás~\cite{DBLP:books/cs/Ullman88}

Szabványok, nyílt forráskódú és fizetős eszközök

\section{Motiváció (ismét)}

Most, hogy már értem az anyagot, miért fontos ez? Miért jó ez, ha programozó, DBA, \ldots leszek.

\section{Tanulást segítő anyagok}

Ellenőrző kérdések

Kulcsfogalmak (a szójegyzéken kívül a legfontosabb fogalmak)

Szójegyzék