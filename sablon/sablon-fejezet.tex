\topic{Tartalmi sablon}

\section{Bevezető}

Indítsunk egy 1-2 bekezdéses bevezetővel, ami pozicionálja a témakört.

\section{Motiváció}

Mutassuk be a vezérpélda megfelelő aspektusát. Milyen műszaki kihívások merülnek fel itt? Ezekre a műszaki kihívásokra van egy alkalmas \fogalom{modell}, \fogalom{formalizmus} vagy \fogalom{modszertan} -- informálisan vezessük be ezt.

\section{Törzsanyag}

A javasolt tartalmi felépítés az alábbi.

\subsection{Egy tartalmi blokk}

\begin{mdframed}
	\begin{enumerate}
	\item Modellezési példa (diagram, formalizmus)
	
	\item Hogyan értelmezhető ez? Hogyan szimuláljuk?
	
	\item Formális bevezető. Pl. ez egy halmaz, ahol \ldots; ez egy függvény \ldots.
	\end{enumerate}		
\end{mdframed}

\subsection{Következő tartalmi blokk}

Az előzőhöz hasonló felépítéssel.

\section{További példák}

Ha maradtak még példáink, amit a korábbi részbe nem tudtunk beleszőni, azokat gyűjtsük össze itt.

\section{Kiegészítő anyagok}

Ezeket az anyagrészeket nem kérjük számon, de kiegészítik, mélyítik a korábbi ismereteket.

\section{Kapcsolódó témák}

Kapcsolódó anyagrészek/tárgyak (ha eddig nem kerültek említésre). Egy példa irodalomhivatkozás~\cite{DBLP:books/cs/Ullman88}

Itt jelenhetnek meg kapcsolódó szabványok, nyílt forráskódú, valamint fizetős eszközök is.

\section{Gyarkorlati jelentőség}

(Ha az eddigi szövegbe nem sikerült beleszőni.) Most, hogy már értem az anyagot, miért fontos ez? Miért jó ez, ha programozó, DBA, \ldots\ leszek.

\section{Tanulást segítő anyagok}

Ellenőrző kérdések.

Kulcsfogalmak (a tárgymutatóban szereplő \emph{legfontosabb} fogalmak).

Tárgymutató és szótár (a \texttt{glossary-entries.tex} alapján).