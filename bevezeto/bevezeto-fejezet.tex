\topic{Bevezető}

Vajon mennyi programkódot képes átlátni egy programozó? Hány sorból áll a Windows 7 forráskódja?\footnote{Körülbelül \href{http://www.fastcodesign.com/3021256/infographic-of-the-day/infographic-how-many-lines-of-code-is-your-favorite-app\#5}{40 millió sorból}.} Hogyan lehetséges mégis kézben tartani ekkora szoftverek fejlesztését? Hogyan garantálhatjuk, hogy a végső termék nem lesz hibás, megfelelő lesz a teljesítménye, és a rajta dolgozó mérnökökön kívül később más is képes lesz megérteni és módosítani a programot? Ezekre a kérdésre ad választ a Rendszermodellezés tárgy a komplex rendszerek modellezésének alapvető aspektusainak bemutatásával.

A \emph{modellezés} egy jól bevált technika a bonyolult rendszerek komplexitásának kezelésében. A tárgy célja ezért a hallgatók megismertetése a szoftver- és hardverrendszerek modellezésének eszközeivel és módszereivel. A modellezés alapfogalmainak bevezetése után a modellek két nagy csoportjával, a struktúra-alapú és viselkedés-alapú modellekkel foglalkozunk. Szó lesz a modellek fejlesztéséről, illetve az elkészült modellek ellenőrzéséről, különböző felhasználásairól is.

\begin{pelda}
	Vállalatunk a kezdeti sikereken felbuzdulva úgy dönt, hogy belép a közösségi taxizás piacára egy saját szolgáltatás indításával. Ehhez ki kell fejleszteni egy nagyméretű, elosztott rendszert, amelynek részét képezi majd a cég belső gépeire telepített központi szerver és a felhasználók okostelefonjain futó mobilalkalmazás, valamint kapcsolatban fog állni egy bank rendszerével is a fizetések lebonyolításához.
	
	A szenior szoftvermérnök a kezdetektől modellalapú fejlesztésben gondolkodik, hiszen a rendszer mérete és összetettsége több szempontból is indokolja a modellek használatát. Először is a tervezés első lépései során nem lenne szerencsés alacsony szintű technikai problémákkal foglalkozni. Egy megfelelően \emph{absztrakt} modell segítségével a részletkérdések kezdetben elhanyagolhatók, így a mérnökök elsőként a legfontosabb jellemzők kialakítására fókuszálhatnak.
	
	Mivel a megcélzott terület a cég számára új, mindenekelőtt a kapcsolódó fogalmak és a köztük húzódó összefüggések feltárása lesz a cél. A rendszer elosztottsága miatt szükség lesz az egyes összetevők (szerver, mobiltelefon, bank, hálózat stb.) kapcsolatainak modellezésére is. Az ezekhez szükséges eszközöket a \emph{struktúra alapú modellezés} nyújtja.
	
	Amint elkészül a rendszer architektúrája, a mérnökök nekiláthatnak az egyes komponensek viselkedésének tervezéséhez. Itt két kapcsolódó feladat is felmerül: meg kell tervezni a rendszer \emph{folyamatait} (pl. autórendelés, regisztráció, egyenleg feltöltése stb.), illetve ennek megvalósításához az egyes komponensek belső \emph{állapotait} és lehetséges állapotváltozásait. Az ezekhez szükséges eszközöket az \emph{viselkedés alapú modellezés} nyújtja.
	
	Természetesen a vezetőség szeretné elkerülni a tervezési hibák miatti többletköltségeket. A szenior mérnök javaslatára a megszokott \emph{tesztelésen} kívül már a rendszer modelljein is \emph{ellenőrzéseket} fognak végrehajtani, így a hibák azelőtt kiderülhetnek, hogy a tényleges implementációt (drága emberi erőforrás felhasználásával) elkészítenék.
	
	A szoftverek hibátlan működése mellett legalább ilyen fontos, hogy a rendszer \emph{teljesítménye} megfelelő legyen. Ha a szolgáltatásunk nem elérhető vagy lassan válaszol, a felhasználók minden bizonnyal a konkurenciához menekülnek majd, a vállalkozásunk pedig kudarcba fullad. Szerencsére a leendő rendszer teljesítménye is jól becsülhető a modellek alapján, ehhez a \emph{teljesítménymodellezés} nyújt majd eszköztárat.
	
	Részben ide tartozik az is, hogy a legtöbb viselkedésmodell \emph{szimulálható}. Ez több szempontból is a hasznunkra válik majd, hiszen egyrészt már a fejlesztés korai fázisában ki lehet próbálni az alapfunkciókat, másrészt a szimulációk eredményét is felhasználhatjuk a teljesítmény becslésére.
	
	Akár a szimulációkból, akár később az elkészült rendszer monitorozásából rengeteg információ is a rendelkezésünkre áll majd. Ahhoz, hogy ezekből használható tudást állítsunk elő, és adott esetben problémákat, vagy a rendszer rejtett tulajdonságait felfedezhessük, a \emph{felderítő adatelemzés} eszköztárát is bevethetjük majd.
	
	Korábban szóba került már a mindenek felett álló költséghatékonyság és a drága emberi erőforrás is. Szerencsére a szenior mérnöknek erre is van válasza: a kellően precíz modellekből akár \emph{kódot is generálhatunk}, amivel rengeteg mérnökórát spórolhatunk meg.
	
	Nincs más hátra, mint elküldeni a cégünk egyébként kiváló programozóit egy rendszermodellezés témájú tanfolyamra -- vagy felbérelni egy fejvadász céget az ilyen téren (is) jól képzett műegyetemisták levadászására...
\end{pelda}

%TODO: ezt az ígéretet lehet, hogy felül kell majd vizsgálni.
Ebben a jegyzetben a tárgyhoz tartozó legfontosabb témákat mutatjuk be. Az alapfogalmak definícióit releváns mérnöki példákkal illusztráljuk és magyarázzuk, illetve gyakorló feladatokkal segítjük a bemutatott módszerek megértését.
Ehhez különböző színű keretes írásokat fogunk használni. Amint az előzőekben látszott, a példák narancssárga háttérrel jelennek meg. A definíciók sárga háttérrel kerülnek kiemelésre, míg a kiegészítő megjegyzések szürke színt kapnak. A jegyzetben szerepelnek majd feladatok is piros háttérrel, valamint hasznos tippek zöld keretben.

\begin{definicio}
	A \emph{definíció} egy fogalom pontos, precíz meghatározása.
\end{definicio}

\begin{megjegyzes}
	Időnként megjegyzésekkel fogjuk árnyalni a bemutatott anyagot.
\end{megjegyzes}

\begin{feladat}
	Gyakorlásképp oldja meg az egyes fejezetekhez tartozó feladatokat!
\end{feladat}

\begin{tipp}
	A feladatokhoz hasonló kérdések előfordulhatnak a számonkéréseken is.
\end{tipp}

\section*{A jegyzetről}

A jegyzet \LaTeX\ nyelvű forráskódja elérhető a \url{https://github.com/FTSRG/remo-jegyzet} oldalon.

A jegyzet jelenleg fejlesztés alatt áll. Amennyiben az elkészült részekben hibát vagy hiányosságot találtok, kérjük jelezzétek a GitHubon egy új \emph{issue} felvételével. Amennyiben javítási ötletetek van, örömmel fogadjuk a \emph{pull requesteket} is.
