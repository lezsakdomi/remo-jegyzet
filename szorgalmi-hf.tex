% !TeX spellcheck = hu_HU
% !TeX encoding = UTF-8
\documentclass[a4paper]{article}

\usepackage{ifxetex}
\ifxetex
	\usepackage{fontspec}
\else
	\usepackage[T1]{fontenc}
	\usepackage[utf8]{inputenc}
	\usepackage[lighttt]{lmodern}
\fi
\usepackage[magyar]{babel}
\usepackage{geometry}
\geometry{margin=1in}

\frenchspacing

\title{Tesztfedettség számítása}
\author{Rendszermodellezés szorgalmi feladat}
\date{2016. tavaszi félév}

\pagenumbering{gobble}

\begin{document}
	
\maketitle

\section{Áttekintő}

A szorgalmi feladat a tárgy során elkészített házi feladathoz kapcsolódik. A házi feladat során elkészült
a sakkóra állapot alapú modellje, amelyre a kiadott tesztesetek hibamentesen futottak. 
A szorgalmi feladat során a tesztkészlet fedettségét fogjuk meghatározni és szükség esetén bővíteni.
A feladat megoldásához szükséges mind az elkészült Yakindu állapotgépmodell, 
mind az ellenőrzésére kiadott tesztkészlet.

\section{Elvégzendő feladatok}

A ,,Modellek ellenőrzése és tesztelése'' című gyakorlaton tanult (állapotgépekre vonatkozó) tesztfedettségi metrikákat fogjuk a szorgalmi feladatban használni. Ezek a következők:

\begin{itemize}
	\item Állapot lefedettség
	\item Átmenet lefedettség
\end{itemize}

A tesztfedettség fogalmáról bővebb információ olvasható a tárgyhonlapon publikált segédanyagok között, a ,,Modellek ellenőrzése'' c. előadásban.

\subsection{Fedettség számítása}
Számítsa ki a kiadott tesztkészlet által megvalósított (állapot, átmenet) fedettséget! Válaszát indokolja! Mely állapotok 
és átmenetek nem kerültek fedésre? Mik azok az állapotok, amelyeket több teszt is lefed?

\subsection{Tesztkészlet bővítése}
A kiadott tesztkészletet bővítse úgy, hogy teljes, azaz 100\%-os állapot és átmenet lefedettséget érjünk el. A tesztekhez definiálja a következőket (táblázatos formában):

\begin{itemize}
	\item Tesztbemenet (a házi feladat kiírásban kiadotthoz hasonló formában: akciók sorozata)
	\item Elvárt kimenet (az állapotváltozók értékei időbeli sorrend szerint)
\end{itemize}

Igazolja a tesztfedés teljességét! Az értékelés során figyelembe vesszük, hogy milyen ,,kompakt'' tesztesetek készültek el (azaz minél kevesebb input felhasználásával javuljon a fedettség.)

%Vizsgálja meg az elkészült teszt szekvenciá(ka)t méret szempontjából: lehetne-e a fedettséget növelni kevesebb/rövidebb tesztek felhasználásával? Válaszát indokolja!

\section{Dokumentáció}
A feladat megoldását dokumentálja szövegesen, a döntések indoklásával! A tesztkészlet bővítésekor elég szövegesen leírni a kigondolt újabb teszteket, nem kell őket megvalósítani a házi feladat projektben. A dokumentáció viszont legyen olyan részletességű, hogy az alapján egy fejlesztő elkészíthesse a tesztek implementációját. 

\end{document}
